\section{Ricerca della costante elastica della molla: metodo dinamico}

\subsection{Apparato sperimentale}

L'apparato sperimentale e la strumentazione utilizzata per compiere questo secondo esperimento sono gli stessi utilizzati per la procedura del metodo statico, eccezion fatta per il cronometro. Il cronometro da noi utilizzato è un normale cronometro con risoluzione di misura di 0.01 s.

\subsection{Studio qualitativo e analisi dimensionale}

Come prima cosa abbiamo verificato in maniera qualitativa le relazioni che intercorrono tra
costante elastica della molla, massa applicata, ampiezza dell'oscillazione e periodo di oscillazione.
Per fare ciò, abbiamo fatto oscillare molle diverse (che significa costanti elastiche differenti), abbiamo variato
le masse applicate alle molle ed abbiamo variato l'ampiezza delle oscillazioni. I risultati sono stati
i seguenti:

\begin{itemize}
    \item{A parità di altre condizioni, all'aumentare della massa il periodo dell'oscillatore aumenta.}
    \item{A parità di altre condizioni, le molle più ``morbide'' (con costante elastica più bassa), hanno periodi più lunghi
        di molle più rigide (con costante elastica maggiore).}
    \item{A parità di altre condizioni, l'ampiezza dell'oscillazione non fa variare il periodo della molla. Questo significa
        che il periodo non dipende dall'ampiezza dell'oscillazione, almeno non in prima approssimazione.}
\end{itemize}

\paragraph{Analisi dimensionale\\}

Con un analisi dimensionale è possibile ricavare una formula che metta in relazione periodo, massa e costante elastica della molla.
Per quanto visto sopra, supponiamo che il periodo $\mathcal{T}$ dipenda dalla massa $m$ appesa alla molla, dalla costante elastica $k$
e dall'ampiezza $A$ dell'oscillazione (e ci aspettiamo già che non dipenda dall'ampiezza).
Si ha quindi che $\mathcal{T} \propto m^\alpha k^\beta A^\gamma$. Facendo un analisi dimensionale:

\begin{equation*}
    [\mathcal{T}]^1 \,=\, [M]^{\alpha+\beta} [L]^\gamma [T]^{-2\beta} \quad \implies \quad \gamma = 0; \,\, \beta = -1/2; \,\, \alpha = 1/2
\end{equation*}
%
da cui si segue che

\begin{equation}
	\mathcal{T} \,=\, \mathcal{C} \, \sqrt{\frac{m}{k}}
    \label{eq:pp}
\end{equation}
%
dove $\mathcal{C}$ rappresenta una costante che non può essere determinata mediante l'analisi dimensionale. Come ci si poteva aspettare
dall'analisi qualitativa, il periodo non dipende dall'ampiezza dell'oscillazione.

\subsection{Procedura di acquisizione dei dati}

Per calcolare la costante elastica mediante il metodo dinamico ci siamo avvalsi della relazione (\ref{eq:pp})
ricavata dall'analisi dimensionale.

Per le misurazioni sono state usate le stesse masse e le stesse combinazioni di cilindri scelte nell'esperimento precedente, con una sola differenza: è stata aggiunta una nuova configurazione di pesi di massa complessiva nominale 135 g. Il numero totale delle configurazioni è quindi salito a 14.
Nella formula (\ref{eq:pp}), $m$ indica la massa complessiva che viene agganciata alla molla, quindi in questo caso si è dovuto tener conto della massa del piattello portapesi. Per evitare di dover propagare gli errori, abbiamo quindi misurato nuovamente la massa delle combinazioni
con l'aggiunta del piattello. I valori nominali e quelli ricavati con la bilancia elettronica sono riportati nella tabella \ref{tab:masse_dinamico}.

Abbiamo quindi appeso le masse scelte alla molla e la abbiamo fatta oscillare.
Per ottenere il periodo di oscillazione abbiamo agito come segue:

\begin{itemize}
	\item{Si è deciso di cronometrare dieci oscillazioni della molla
        poichè abbiamo ritenuto di non essere in grado di rilevare con una precisione
        accettabile il periodo di una singola oscillazione. Questo metodo ha inoltre
        il pregio di ridurre di un fattor dieci la risoluzione dello strumento.}

	\item{Per ogni massa sono state rilevate 15 misure di periodo.
        Ogni componente del gruppo ha cronometrato cinque cicli di dieci oscillazioni. 
        In questo modo gli errori sistematici dovuti alla prontezza di riflessi degli
        operatori dovrebbero essere misurabili o quantomeno visibili.}
\end{itemize}

I valori di lettura del cronometro sono stati divisi per dieci per ottenere i valori reali di periodo dell'oscillatore.
I dati rilevati sono riportati nella tabella \ref{tab:periodi}.

\begin{table}
    \centering
    \scriptsize
    \begin{tabular}{l | c c c c c c c c c c c c c c}
        \multicolumn{15}{c}{\small \textbf{Masse delle combinazioni di cilindri [g]}} \\[1mm]
        \toprule
        Nominali & 30 & 40 & 50 & 60 & 70 & 80& 90 & 100 & 110 & 120 & 130 & 140 & 150 & 160 \\
        Pesate & 30.3 & 40.3 & 50.2 & 60.3 & 70.4 & 80.3 & 90.5 & 100.3 & 110.4 & 120.4 & 130.4 & 140.4 & 150.5 & 160.5 \\
        \bottomrule
    \end{tabular}
    \caption{La tabella riporta i valori nominali e i valori rilevati con la bilancia della massa delle combinazioni di pesi scelte per
    l'esperimento. Ogni valore include la massa del piattello portapesi e della combinazione di cilindri scelta. Le misure sono state fatte
    con una bilancia di risoluzione di 0.1 g.}
    \label{tab:masse_dinamico}
\end{table}

\begin{table}
    \centering
    \scriptsize
    \begin{tabular}{l | c c c c c c c c c c c c c c c}
        \multicolumn{16}{c}{\small \textbf{Periodi}} \\[1mm]
        \toprule
        {\footnotesize Massa [g]} & \multicolumn{15}{c}{\footnotesize Periodi di oscillazione della molla [s]} \\
        \midrule
		30.3 & 3.84 & 3.90 & 3.95 & 3.95 & 3.99 & 3.91 & 3.91 & 3.93 & 3.90 & 3.94 & 4.04 & 3.92 & 4.00 & 3.98 & 3.95 \\
		40.3 & 4.38 & 4.43 & 4.43 & 4.45 & 4.41 & 4.54 & 4.45 & 4.49 & 4.50 & 4.56 & 4.49 & 4.52 & 4.47 & 4.49 & 4.41 \\
		50.2 & 4.84 & 4.76 & 4.85 & 4.92 & 4.87 & 4.93 & 4.88 & 4.78 & 4.83 & 4.82 & 4.95 & 5.02 & 4.93 & 4.99 & 4.78 \\
		60.3 & 5.29 & 5.33 & 5.27 & 5.24 & 5.30 & 5.27 & 5.30 & 5.21 & 5.25 & 5.27 & 5.30 & 5.34 & 5.34 & 5.31 & 5.34 \\
		70.4 & 5.65 & 5.65 & 5.64 & 5.63 & 5.59 & 5.63 & 5.64 & 5.63 & 5.66 & 5.63 & 5.81 & 5.67 & 5.63 & 5.63 & 5.65 \\
		80.3 & 5.98 & 5.98 & 5.95 & 5.98 & 5.95 & 6.01 & 6.08 & 6.02 & 5.98 & 5.95 & 5.98 & 5.99 & 6.06 & 6.09 & 6.07 \\
		90.5 & 6.33 & 6.34 & 6.32 & 6.34 & 6.32 & 6.33 & 6.26 & 6.31 & 6.31 & 6.31 & 6.34 & 6.35 & 6.37 & 6.34 & 6.34 \\
		100.3 & 6.58 & 6.63 & 6.56 & 6.59 & 6.56 & 6.63 & 6.69 & 6.64 & 6.58 & 6.63 & 6.63 & 6.68 & 6.63 & 6.57 & 6.69 \\
		110.4 & 6.86 & 6.96 & 6.93 & 6.94 & 6.84 & 6.91 & 6.86 & 6.99 & 6.96 & 6.94 & 6.99 & 6.84 & 6.94 & 6.95 & 6.88 \\
		120.4 & 7.24 & 7.20 & 7.22 & 7.23 & 7.20 & 7.18 & 7.23 & 7.13 & 7.24 & 7.23 & 7.24 & 7.21 & 7.27 & 7.26 & 7.24 \\
		130.4 & 7.53 & 7.51 & 7.45 & 7.52 & 7.56 & 7.53 & 7.52 & 7.57 & 7.52 & 7.52 & 7.56 & 7.55 & 7.53 & 7.46 & 7.50 \\
		140.4 & 7.81 & 7.78 & 7.76 & 7.74 & 7.74 & 7.73 & 7.79 & 7.82 & 7.81 & 7.74 & 7.79 & 7.79 & 7.73 & 7.78 & 7.79 \\
		150.5 & 8.02 & 7.94 & 8.04 & 8.06 & 8.07 & 8.12 & 8.02 & 8.05 & 7.98 & 8.04 & 7.99 & 7.93 & 8.02 & 7.99 & 8.02 \\
		160.5 & 8.30 & 8.27 & 8.15 & 8.29 & 8.25 & 8.24 & 8.35 & 8.31 & 8.41 & 8.27 & 8.36 & 8.34 & 8.34 & 8.31 & 8.31 \\
        \bottomrule
    \end{tabular}
    \caption{Periodi di oscillazione misurati con masse diverse. Ogni riga contiene i valori di lettura del cronometro per le misure di 10 periodi di oscillazione relativi alla massa riportata nella prima colonna. I valori sono stati ottenuti da tre operatori, ognuno 5 misure, per un totale di 15. Il cronometro aveva una risoluzione di 0.01 s.}
    \label{tab:periodi}
\end{table}

\subsection{Elaborazione dei dati}

\begin{SCtable}
    \centering
	\begin{tabular}{l c  c  c  c }
        \multicolumn{5}{c}{\textbf{Medie periodi}} \\
        \toprule
        $i$ & $\mathcal{T}_i$ [s] & $\delta\mathcal{T}_i$ [s] & $\mathcal{T}_i^2$ [$s^2$] & $\delta(\mathcal{T}_i^2)$ [$s^2$] \\

        \midrule

        1 & 0.394		& 0.001		& 0.155		& 0.001	\\
        2 & 0.447		& 0.001		& 0.200		& 0.001	\\
        3 & 0.488		& 0.002		& 0.238		& 0.002	\\
        4 & 0.529		& 0.001		& 0.280		& 0.001	\\
        5 & 0.565		& 0.001		& 0.319		& 0.001	\\
        6 & 0.600		& 0.001		& 0.361		& 0.002	\\
        7 & 0.633		& 0.001		& 0.400		& 0.001	\\
        8 & 0.662		& 0.001		& 0.438		& 0.002	\\
        9 & 0.692		& 0.001		& 0.479		& 0.002	\\
        10 & 0.722		& 0.001		& 0.521		& 0.001	\\
        11 & 0.752		& 0.001		& 0.566		& 0.001	\\
        12 & 0.777		& 0.001		& 0.604		& 0.001	\\
        13 & 0.802		& 0.001		& 0.643		& 0.002	\\
        14 & 0.830		& 0.002		& 0.689		& 0.003	\\

        \bottomrule

	\end{tabular}
    \caption{MANCA IL CAPTIOOOON
    CONTROLLARE PROPAGAZIONE ERRORI (ES: ERRORE CASUALE + ERRORE RISOLUZIONE)}
    \label{tab:calcolati}
\end{SCtable}

\subsubsection{Studio quantitativo del moto oscillatorio}

Le considerazioni sull'incertezza della massa sono le stesse già discusse nell'esperimento precedente,
quindi l'incertezza sulla massa, uguale per tutte le combinazioni di cilindri, vale:

\begin{equation*}
    \delta m = 0.03 \,\, g
\end{equation*}

Per comodità notazionale da qui in avanti indicheremo con $T_i$ i periodi relativi alla massa $m_i$, con $i$ che va da 1 a 14 ed
indica i pesi in ordine crescente di massa, come sono riportati in tabella \ref{tab:masse_dinamico}.
Per ogni massa utilizzata, è stata calcolata la media dei periodi $\mathcal{T}_i \equiv m^*[T_i]$ e la deviazione tipo sulla distribuzione
delle medie $\sigma[\mathcal{T}_i]$.

È doveroso notare che tutte le misure del periodo sono affette dall'errore di risoluzione standard.
Dal momento che andremo ad operare con le medie dei 15 periodi l'incertezza complessiva delle medie è data dalla composizione
dell'errore di risoluzione standard con l'incertezza statistica. Cioè, per ogni massa $i$-esima:

\begin{equation*}
    \delta \mathcal{T}_i = \sqrt{\sigma[\mathcal{T}_i]^2 + \sigma_{ris}[\mathcal{T}]^2}
\end{equation*}
%
dove $\sigma_{ris}[\mathcal{T}]$ indica l'errore di risoluzione tipo pari a 0.0003 s.

I valori calcolati sono elencati nelle prime
due colonne della tabella \ref{tab:calcolati}.
In figura \ref{fig:periodo_massa} è mostrato il grafico del periodo di oscillazione $\mathcal{T}_i$ della molla al variare della massa $m_i$ ad essa appesa. Si vede chiaramente
che i punti sul grafico non sono allineati lungo una retta. Infatti, come si può notare dalla formula (\ref{eq:pp}),
il periodo di oscillazione della molla non è proporzionale a $m$, bensì a $\sqrt{m}$.

Al fine di ottenere un andamento lineare abbiamo considerato i valori $\mathcal{T}_i^2$. Le incertezze $\delta \mathcal{T}_i^2$
sui valori di $\mathcal{T}_i^2$ sono state ottenute usando la regole di propagazione dell'incertezza, con la formula:

\begin{equation*}
    \delta(\mathcal{T}_i^2) = 2 \,\, \mathcal{T}_i \,\, \delta \mathcal{T}_i
\end{equation*}

I valori ottenuti con questa procedura sono riportati nella stessa tabella \ref{tab:calcolati}, nelle ultime due colonne e sono
graficati in figura \ref{fig:periodo2_massa}. In questo caso è evidente l'andamento lineare dei punti riportati in figura.

\begin{SCfigure}
    \centering
    \includegraphics[width=120mm]{immagini/periodo_massa.pdf}
    \caption{Il grafico mostra il periodo di oscillazione della molla al variare della massa ad essa appesa.
        Le barre d'errore su massa e periodo non sono mostrate in quanto molto piccole e difficilmente leggibili.
        Come si vede chiaramente il periodo non dipende dalla massa in modo lineare, al contrario, segue una legge del tipo
        $\sqrt{a + bm}$, con coefficienti a e b, come quella tracciata in figura.}
    \label{fig:periodo_massa}
\end{SCfigure}

\begin{SCfigure}
    \centering
    \includegraphics[width=120mm]{immagini/periodo2_massa.pdf}
    \caption{Il grafico mostra la relazione tra il periodo di oscillazione della molla al quadrato e la massa ad essa appesa.
        Anche in questo caso le barre d'errore non sono mostrate in quanto troppo piccole per una chiara lettura.
        Si può notare che ${\mathcal{T}_i}^2$ e $m_i$ sono legati da una relazione lineare. Si nota inoltre che la retta di fit
        interseca l'asse delle ascisse nel punto $-m_e$.}
    \label{fig:periodo2_massa}
\end{SCfigure}

\subsubsection{Massa efficace della molla}

Il sistema molla-massa studiato è un sistema reale che non rispecchia con fedeltà assoluta la legge (\ref{eq:pp}) ricavata in precenza, valida solo in prima approssimazione. Infatti, per ricavare la (\ref{eq:pp}) con analisi dimensionale
si devono fare delle ipotesi sulle grandezze che influenzano il periodo, e si può facilmente mancare di considerare qualche
fattore importante. Per esempio non sono state prese in considerazione le caratteristiche proprie della molla stessa
al di fuori della sua costante elastica. La molla possiede infatti una massa ed altre caratteristiche
come una dissipazione interna, che sono state trascurate. Il principale fattore che è stato trascurato
è l'inerzia (derivante dalla massa) della molla stessa. Questo effetto si può notare nel grafico in figura \ref{fig:periodo2_massa},
dove si vede che la retta che meglio interpreta i dati non passa per l'origine.

Questi effetti apparentemente complicati, possono essere semplificati introducendo un fattore correttivo $m_e$ detto
\emph{massa efficace} della molla, che ne tiene conto. Questo fattore correttivo va a modificare l'equazione (\ref{eq:pp}) che diventa

\begin{equation}
    \mathcal{T} \, = \, \mathcal{C} \, \sqrt{\frac{M}{k}}
    \label{eq:ppme}
\end{equation}
%
dove $M = m_e + m$. Quindi nella figura \ref{fig:periodo2_massa} la retta interseca l'asse delle ascissa
nel punto $m_e$.

La massa efficace non è la massa della molla stessa, ma la massa fittizia che occorrerebbe
appendere ad una molla ideale per rendere le sue oscillazioni identice a quelle della molla reale.

Troviamo ora la relazione che collega la massa efficace $m_e$ della molla alla sua massa reale $\mu$.
Immaginiamo di far oscillare la molla di massa $\mu$ e lunghezza $L$ mantenendo fisso un estremo.
Consideriamo un pezzetto infinitesimo della molla a distanza $\ell$ dall'estremo fisso. La massa del pezzo
varrà $\mu \, d\ell / L$. Ipotizzando che la velocità di tale pezzo sia proporzionale alla distanza dall'estremo
fisso, si ottiene che la sua velocità è $v \, \ell / L$, dove $v$ è la velocità dell'estremo libero.
Calcoliamo l'energia cinetica tramite l'integrale:

\begin{equation*}
    E_k \, = \, \int_0^L \frac{1}{2} \frac{\mu v^2}{L^3} \ell^2 \, d\ell \, = \,
    \frac{1}{2} \frac{\mu v^2}{L^3} \int_0^L \ell^2 \, d\ell = \frac{1}{6} \mu v^2
\end{equation*}
%
Tuttavia, per una massa ideale con le stesse oscillazioni, vale:

\begin{equation*}
    E_k \, = \, \frac{1}{2} m_e v^2
\end{equation*}
%
Eguagliando le due equazioni si ottiene $m_e = \mu/3$ ovvero $\mu = 3\,\,m_e$. 

\subsubsection{Calcolo dei parametri $\mathcal{C}$ e $m_e$ mediante il metodo della regressione lineare}
Considerando l'equazione (\ref{eq:ppme}) e analizzandola tenendo conto del nuovo valore della massa $M = m_e + m$, otteniamo:

\begin{equation*}
	\mathcal{T} \,=\, \mathcal{C} \, \sqrt{\frac{M}{k}}	\,\,\,\Longrightarrow\,\,\, \mathcal{T} \,=\, \mathcal{C} \, \sqrt{\frac{m_e + m}{k}}
\end{equation*}
%
pertanto elevando tutti i membri al quadrato otteniamo:

\begin{equation*}
	\mathcal{T}^2 \,=\, \mathcal{C}^2 \,\, \frac{(m_e + m)}{k}
\end{equation*}
%
che si può riscrivere come segue:

\begin{equation}
	\mathcal{T}^2 \,\,=\,\, \frac{\mathcal{C}^2}{k} m_e \,+\, \frac{\mathcal{C}^2}{k} m
	\label{eq:mCspezzati}
\end{equation}
%
ed inoltre, la si può rendere più leggibile, scivendola nel seguente modo:

\begin{equation}
	y \,=\, A \,+\, Bx
	\label{eq:pppar}
\end{equation}
%
dove $x \equiv m$ (massa applicata) $y \equiv \mathcal{T}^2$, $A \equiv \frac{\mathcal{C}^2}{k} \, m_e$ e $B \equiv \frac{\mathcal{C}^2}{k}$.\\
Vogliamo quindi determinare i valori di A e B che minimizzano la discrepanza mediante la tecnica della regressione lineare pertanto procediamo con il metodo utilizzato anche nella sezione due.
\begin{itemize}
\item{La funzione da minimizzare che misura la discrepanza è:
		\begin{equation*}
			\sum_{i=1}^{N} \frac{(y_i - A - Bx_i)^2}{(\delta y_i)^2}
		\end{equation*}
		%
		ricordando che l'incertezza sull'asse delle ascisse è la stessa incertezza sulla massa, ovvero $\delta m_i$, che in questa analisi poniamo uguale a $\delta x_i$.}
\item{Prima di procedere con l'analisi è necessario verificare che l'incertezza relativa alle masse sia trascurabile rispetto a quella dei periodi al quadrato. Per accertarlo è stato necessario fare una stima preliminare del parametro B. Dal momento che dal grafico è impossibile individuare le rette di massima e minima pendenza abbiamo deciso di ricavare il parmetro B valutando il coefficiente angolare della retta utilizzando il classico rapporto tra due punti del grafico, ovvero:

	\begin{equation*}
		\text{B} \,\,=\,\, \frac{\mathcal{T}_{14} - \mathcal{T}_{1}}{m_{14} - m_1} \,\,=\,\, 4.09 \,\, s^2 \,\,\,[verificare \,\,\,cifre]
	\end{equation*}
	%
	quindi grazie a questo valore di B siamo in grado di stimare l'incertezza trasferita, ovvero: $\sigma (m_{tras}) \,=\, B \, \sigma (m)$ dove $\sigma (m)$ rappresenta l'errore tipo di misura sulla massa. Abbiamo notato che questo valore è 10 volte minore rispetto all'incertezza sul periodo quadrato, e quindi poichè l'incertezza totale risulta essere la seguente
	
	\begin{equation*}
		\delta y_i \,=\, \sqrt{(\delta \mathcal{T}_i^2)^2 + \sigma (m_{tras})^2} 
	\end{equation*}	
	%
	possiamo notare che sotto radice il termine $\sigma (m_{tras})^2$ risulta essere 100 volte minore di $(\delta \mathcal{T}_k^2)^2$ quindi lo si può considerare trascurabile rispetto all'incertezza rispetto al quadrato del periodo.
	}
\item{quindi per quanto studiato in classe abbiamo che:

		\begin{equation*}
			A \,=\, \frac{(\sum_i w_i x_i^2)(\sum_i w-i y_i) - (\sum_i w_i x_i)(\sum_i w_i x_i y_i)}{\Delta} \,=\, 0.033655 \,\, s^2
		\end{equation*}
		%
		\begin{equation*}
			B \,=\, \frac{(\sum_i w_i)(\sum_i w-i x_i y_i) - (\sum_i w_i y_i)(\sum_i w_i x_i)}{\Delta} \,=\, 4.0603 \,\, s^2 / kg
		\end{equation*}
		%
		dove:
		\begin{equation*}
			\Delta \,=\, (\sum_i w_i)(\sum_i w_i x_i^2) - (\sum_i w_i x_i)^2 \,\,\,\,\,\,\, e \,\,\,\,\,\,\,
			w_i \,=\, \frac{1}{(\delta y_i)^2}
		\end{equation*}}
\item{di conseguenza abbiamo che le incertezze relative su A e B sono:

		\begin{equation*}
			(\delta A)^2 \,=\, \frac{\sum_i w_i x_i^2}{\Delta}  \,\,\,\,\, e \,\,\,\,\,
			(\delta B)^2 \,=\, \frac{\sum_i w_i}{\Delta} 
		\end{equation*}}
\end{itemize} 
Quindi possiamo riassumere i risultati di questa procedura in questo modo:

\begin{equation*}
	A \,\pm\, \delta A \,=\, (0.034 \,\, \pm \,\, 0.001) \,\,s^2 \,\,\,\,\, e \,\,\,\,\,
	B \,\pm\, \delta B \,=\, (4.06 \,\, \pm \,\, 0.01) \,\,s^2 \,/\, kg
\end{equation*}
%
Perciò, noti questi parametri, possiamo risalire ai valori di $\mathcal{C}$ e $m_e$ risolvendo l'equazione in (\ref{eq:pppar}) e utilizzando come valore di k quello trovato nell'analisi dati della sezione precedente, ovvero $K_0 \,=\, 9.64 \, N/m$, otteniamo quanto segue:

\begin{equation*}
	\mathcal{C} \,\pm\, \delta \mathcal{C} \,\,=\,\, (6.256 \, \pm \, 0.007)
\end{equation*}
%
\begin{equation*}
	m_e \, \pm \, \delta m_e \,\,=\,\, (0.0083 \,\, \pm \,\, 0.0002) \,\, kg
\end{equation*}

\subsubsection{Test del chi quadro}
Procediamo ora a verificare che le equazioni (\ref{eq:mCspezzati}) e (\ref{eq:pppar}) i valori sopra ottenuti siano compatibili con i dati sperimentali mediante il test del chi quadro. Ricordiamo che il numero di gradi di libertà in questo caso non è più N - 1, ma risulta essere N - 2 in qunto due dati sono stati utilizzati per calcolare i  valori sperimentali A e B. Pertanto ci aspettiamo che:

\begin{equation*}
	\chi_{oss}^2 \,=\, \sum_{i=1}^{N} \frac{(y_i - A - Bx_i)^2}{(\delta y_i)^2} \,\simeq\, N - 2 \quad \text{ricordando che} \quad \chi_{teo}^2 \,\,:=\,\, N - 2
\end{equation*}
%
il nostro valore del chi quadro osservato è:

\begin{equation*}
	\chi_{oss}^2 \,\,=\,\, 21 \quad \text{che risulta essere compatibile con il} \quad \chi_{teo}^2
\end{equation*}
%
in quanto $\chi_{teo}^2 \, \pm \, \delta \chi_{teo}^2 \,\,=\,\, 12 \, \pm \, 5$ pertanto la differenza tra i due chi quadro risulta essere la seguente:

\begin{equation*}
	|R| \,\,=\,\, (\chi_{oss}^2 - \chi_{teo}^2) \,\,=\,\, 9.1
\end{equation*}
%
mentre, posto a priori un fattore di copertura k = 3, l'errore sul resto è:

\begin{equation*}
	k \, \sigma_{R} \,\,=\,\, 14.2
\end{equation*}
%
e quindi risulta che $|R| \, \leq \, k \, \sigma_{R}$.\\

Dal momento che il chi quadro osservato, pur essendo compatibile con la predizione teorica, risulta essere molto maggiore del chi quadro teorico, abbiamo deciso di rifare il fit per ottenere una migliore stima delle inceretezze sul periodo che probabilmente sono state sottostimate.\\
Pertanto rifacendo la procedura del fit vogliamo aggiustare il valore delle incertezze in modo da eguagliare i due chi quadro, ovvero quello teorico e quello sperimentale. Pertanto:

\begin{equation*}
	\chi_{oss}^2 \,=\, \sum_{i=1}^{N} \frac{(y_i - A - Bx_i)^2}{(\delta y_i)^2} \,=\, \chi_{teo}^2 
\end{equation*}
%
quindi sfruttando questa relazione vogliamo trovare un coefficiente che moltiplicato per il precedente $\delta y_i$ ci restituisce una nuova varianza grazie alla quale i due chi risulteranno uguali. Perciò otteniamo quanto segue:
	
\begin{equation*}
	\chi_{oss}^2 \,\,=\,\, \frac{1}{(\delta y)^2} \sum_{1}^{N} (y_i - A - Bx_i)^2  \,\,=\,\, \chi_{teo}^2
\end{equation*}
%
ricordando che il termine $(\delta y)^2$ la nuova stima dell'incertezza sul periodo. Sapendo che valgono:
	
\begin{equation*}
	\chi_{teo}^2 \,=\, \nu \quad \quad con \quad \quad
	\nu \,=\, N - 2 \quad \quad e \quad \quad
	\delta y_{posteriori}^2 \,\,=\,\, \frac{\chi_{oss}^2}{\nu} \, \delta y_{priori}^2
\end{equation*}
%
con $\nu$ che rappresenta i gradi di libertà. Otteniamo:

\begin{equation*}
	\delta x_{posteriori} \,=\, \bigotimes \,s
\end{equation*}	 

\subsubsection{Determinazione dinamica della costante elastica K}
Grazie alla procedura di regressione lineare adottata sopra siamo riusciti a determinare i 2 parametri A e B, pertanto ora, sfruttando la teoria dell'oscillatore armonico, si trova che $\mathcal{C}$ ha il valore di $2 \, \pi$, per cui assumendo che la nostra molla sia perfettamente elastica, otteniamo la relazione:

\begin{equation}
	\mathcal{T} \,\,=\,\, 2 \, \pi \, \sqrt{\frac{M}{k}} \quad \text{dove} \quad M \,=\, m_e\,+\,m
\end{equation}
%
Se noi imponiamo $\mathcal{C} \,=\, 2\,\pi$ siamo in grado di stimare nuovamente i parametri $k \,e\, m_e$ a partire dalla regressione lineare, in quanto:
\begin{itemize}
	\item{dalla definizione di B sappiamo che:
	
			\begin{equation*}
				B \,\,=\,\, \frac{\mathcal{C}^2}{k} \quad \Longrightarrow \quad k \,\,=\,\, \frac{\mathcal{C}^2}{B} \,\,=\,\, 9.72 \, N/m
			\end{equation*}}
	\item{mentre dalla definizione di A sappiamo che:
	
			\begin{equation*}
				A \,\,=\,\, \frac{\mathcal{C}^2}{k} \, m_e \,\, = \,\, \frac{B\,k}{k} \, m_e \,\, = \,\, B\,m_e \quad \Longrightarrow \quad m_e \,=\, \frac{A}{B} \,\,=\,\, \bigotimes
			\end{equation*}}
\end{itemize}
Invece per quanto riguarda gli errori relativi a questi risultati abbiamo che:
\begin{itemize}
	\item{per stimare l'errore relativo alla costante elastica si può procedere come segue:
	
			\begin{equation*}
				\delta k_0 \,\,=\,\, \frac{\delta B}{\mathcal{C}^2} \,\,=\,\, 0.02 \, N/m
			\end{equation*}
			%
			}
	\item{mentre per stimare l'errore sulla massa efficace della molla si può sfruttare la regola della propagazione dell'errore per divisoni e ottenere:
	
			\begin{equation*}
				\delta m_e \,\,=\,\, m_e \, \sqrt{(\frac{\delta A}{A})^2 + (\frac{\delta k_0}{k})^2} \,\,=\,\, \bigotimes
			\end{equation*}
			%
			}
\end{itemize}
Abbiamo dunque ottenuto che la costante elastica trovata mediante la procedura dinamica risulta essere:

\begin{equation*}
	k_0 \, \pm \, \delta k_0 \,\,=\,\, (9.72 \, \pm \, 0.02) \, N/m
\end{equation*}
%
e la massa efficace della molla:

\begin{equation*}
	m_e \, \pm \, \delta m_e \,\,=\,\, \bigotimes
\end{equation*}
%
Quindi non ci rimane che verificare se la costante elastica trovata con questa procedura è compatibile con quella trovata grazie alla procedura statica mediante la tecnica della regressione lineare.\\
Come prima cosa andiamo a calcolare il resto $|R|$ dato dalla differenza dei due $k_0$:

\begin{equation*}
	|R| \,\,=\,\, (k_{dinamico} \,-\, k_{statico}) \,\,=\,\, 0.08 \, N/m
\end{equation*}
%
Posto a priori un fattore di copertura $k \,=\, 3$, calcoliamo lo scarto relativo al resto R ($\sigma_R$):

\begin{equation*}
	\sigma_R \,\,=\,\, \sqrt{(\delta k_{dinamico})^2 \,-\, (\delta k_{statico})^2} \,\,=\,\, 0.03 \,\, N/m	
\end{equation*}
%
infine verifichiamo se:
\begin{equation*}
	|R| \,\,\leq\,\, k \, \sigma_R  
\end{equation*}
%
Poiché risulta effettivamente che $|R| \,\leq\, k\,\sigma_R$, le due costanti elastiche trovate con le due procedure sono compatibili.



