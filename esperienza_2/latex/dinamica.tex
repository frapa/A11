\section{Ricerca della costante elastica della molla: metodo dinamico}

\subsection{Apparato sperimentale}
L'apparato sperimentale utilizzato per compiere le misurazioni del periodo di oscillazione della molla elicoidale è lo stesso utilizzato per la procedura del metodo dinamico. La configurazione dei pesi è la stessa utilizzata pecedentemente come anche la molla. Da notare che in questo caso il valore delle masse applicate alla molla è stato calcolato misurando assieme la massa del gancio e della massa applicata. Inoltre per questa procedura avevamo a nostra disposizione un cronometro con risoluzione di misura pari a un centesimo di secondo ovvero: 0.01 s.

\subsection{Procedura di acquisizione dei dati}
Per calcolare la costante elastica mediante il metodo dinamico dobbiamo avvalerci della relazione che sussiste tra il periodo di oscillazione di una molla e il rapporto tra la massa applicata e la costante elastica della stessa il tutto sotto radice quadrata. Ovvero:

\begin{equation}
	\mathcal{T} = {2\,\pi}{\sqrt{\frac{m}{k}}}
\end{equation}
%
Come si può notare il periodo di oscillazione della molla risulta essere in una proporzionalità quadratica iversa rispetto alla massa applicata alla suddetta. Dal momento che la costante elatica della molla (K) è una costane, allora ci si dovrebbe aspettare che il periodo cresca al crescere della massa appesa, ma non con una proporzionalità diretta poichè la massa risulta essere sotto radice.\\
Per ottenere quidi il periodo di oscillazione della molla abbiamo agito come segue:
\begin{itemize}
	\item{abbiamo deciso di cronometrare il perido di dieci oscillazioni della molla poichè abbiamo ritenuto di non essere in grado di rilevare con una precisione accettabile il periodo di una singola oscillazione;}
	\item{ogni componente del gruppo ha cronometrato per ogni massa cinque cicli di dieci oscillazioni, in modo che alla fine per ogni massa ci siano quindici misure indipendeni di un ciclo da dieci oscillazioni. In questo modo abbiamo ritenuto di poter ottenere una precisione maggiore sulla stima del periodo di oscillazione della molla e anche di evitare possibili errori sistematici dovuti all' azione di un unico sperimentatore.}
\end{itemize}

\subsection{Studio qualitativo del moto oscillatorio}
Come ci si aspetta dall'equvalenza (5) provando a mettere in oscillazione le quattro molle elicoidali a nostra disposizione soggette allo stesso carico il periodo di un'oscillazione è abbastanza differente tra l'una e l'altra. Infatti il periodo di oscillazione delle quattro molle, per piccole ampiezze, è differente pechè la loro costante elastica è diversa e questo risulta particolarmente evedente confrontando il periodo di oscillazione tra la molla più dura e quella più morbida. Lo steso discoso vale se alla stessa molla erano applicate masse differenti.
Al contrario se il nostro apparato sperimentle non si potesse semplificare al modello di pendolo semplice per piccole oscillazioni allora si dovrebbe tenere conto anche dell'ampiezza delle oscillazioni e quindi risulterebbe che: $\mathcal{T} \propto m^\alpha k^\beta A^\gamma$. Facendo una semplice analisi dimensionale si otterrebbe il seguente risultato:

\begin{equation*}
	\mathcal{T} \,=\, \mathcal{C} \, \sqrt{\frac{m}{k}}
\end{equation*}
%
dove $\mathcal{C}$ rappresenta una costante differente da $2 \pi$ che non si può determinare mediante una semplice analisi dimensionale, ma che sicuramente terrà conto anche dell'ampiezza.

\subsection{Studio quantitativo del moto oscillatorio}





 




