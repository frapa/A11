\section{Ricerca della costante elastica della molla: metodo statico}

\subsection{Apparato sperimentale}
L'apparato sperimentale utilizzato è costituito da:
	\begin{itemize}
		\item{una base ad A che sostiene un'asta verticale dotata di gancio di sospensione per le molle e asta millimetrata scorrevole. L'asta millimetrata aveva sul lato sinistro una scala con una risoluzione di mezzo millimetro, mentre sul lato destro c'era una scala con una risoluzione di un millimetro.}
		\item{quattro molle elicoidali con costanti elastiche differenti tra di loro: nello specifico una particolarmente ``morbida'', una ``dura'' e due con caratteristiche intermedie.}
		\item{un piattello portapesi di massa $m_{piattello} = 25.2\,\,g \pm\, 0.1\,\,g$, quattro pesi cilindrici neri rispettivamente di massa nominale 5, 10, 25 e 50 grammi e quattro pesi argentei analoghi a quelli neri, con le stesse masse nominali.}
        \item{una bilancia a un piattello con una risoluzione di 0.1 grammi. (???)}
	\end{itemize}

\subsection{Procedura di aquisizione dei dati}

Ritenendo essere noto che una forza F di trazione o compressione applicata, nel nostro caso, ad una molla elicoidale provoca una deformazione del corpo abbiamo deciso di sfruttare come forza agente la forza peso $F_{p}$ delle masse cilindriche a nostra disposizione. Per fare questo ricordiamo che la relazione tra una massa inerziale e la forza peso è la seguente:

\begin{equation}
	F_{p} = mg
\end{equation}
<<<<<<< HEAD
%
<<<<<<< HEAD
=======
>>>>>>> f04481e5181b3e7888386eb641837b13d89a9525

dove g rappresenta l'accelerazione di gravità che assumiamo avere un valore $g = 9.806\,\,m\,s^{-2}$.

Innanzitutto abbiamo deciso di applicare alla molla 13 masse da 5 g a 125 g, suddividendo l'intervallo in modo equispaziato.
Sono quindi state selezionate tredici combinazioni di cilindri aventi le masse scelte. Per evitare di dover propagare gli errori sulla massa delle combinazioni, invece di misurare la massa di ogni singolo cilindro, si è misurata la massa complessiva di ogni combinazione.
Abbiamo annotato quali cilindri facevano parte di ogni combinazione, in modo da poter ricomporre le combinazioni in un secondo momento. Le masse delle combinazioni sono riportate nella tabella \ref{tab:masse}.

\begin{table}[tb]
    \centering
    \small
    \begin{tabular}{l | c c c c c c c c c c c c c}
        \multicolumn{14}{c}{Masse delle combinazioni di cilindri [g]} \\[1mm]
        \toprule
        Nominali & 5 & 15 & 25 & 35 & 45 & 55 & 65 & 75 & 85 & 95 & 105 & 115 & 125 \\
        Pesate & 5.0 & 15.1 & 24.9 & 35.0 & 45.2 & 55.1 & 65.2 & 75.0 & 85.1 & 95.2 & 105.0 & 115.2 & 125.3 \\
        \bottomrule
    \end{tabular}
    \caption{Masse usate per misurare l'allungamento della molla. Le masse erano composte da combinazioni
    di cilindri da 5, 10, 25 e 50 grammi. Nella seconda riga sono riportati i pesi rilevati con
    la bilancia, che in alcuni casi differiscono da quelli nominali, riportate nella prima riga.}
    \label{tab:masse}
\end{table}

Completata la classificazione dei pesi, abbiamo scelto la molla di cui calcolare la
costante elastica. Abbiamo notato che una di esse era molto ``morbida'' per i carichi che saremmo
andati ad applicare e si deformava in visibilmente anche con carichi piccoli, rendendo impossibili
le misure di deformazione dei carichi più grandi per questioni di spazio.
La abbiamo quindi scartata, anche perché sarebbe stata suscettibile di deformazione
plastica che avrebbe compromesso la buona riuscita dell'esperimento.
Un'altra molla è stata scartata in quanto troppo rigida: le sue deformazioni infatti non sarebbero state apprezzabili con carichi leggeri. Per questi motivi la scelta è ricaduta sulla molla tra 
le due restanti che permetteva una migliore lettura della deformazione in relazione ai pesi.

Per effettuare le misure, abbiamo agganciato il sistema molla-piattello al gancio di sospensione dell'asta verticale, lungo l'asta millimetrata. Atteso che l'oscillazione della molla si smorzasse, il bordo inferiore del piattello agganciato alla molla, è stato allineato con la tacca dei 50 cm dell'asta graduata. In questo modo la posizione di equilibrio della molla risulta essere:

\begin{equation*}
	z_{eq} = z_0\,\pm\,\delta z_0 = 50.0\,cm\,\pm\,0.1\,cm
\end{equation*}

Infine abbiamo misurato l'allungamento della molla al variare delle combinazioni di cilindretti scelte. Ad ogni variazione del carico si è aspettato che le oscillazioni del sistema si smorzassero prima di effettuare la lettura dell'asta graduata. La lettura è stata effettuata sul lato dell'asta con risoluzione di un millimetro, poiché non siamo riusciti ad apprezzare i mezzi millimetri a causa delle oscillazioni della molla, che per quanto piccole erano sempre presenti. Così facendo abbiamo ottenuto una serie di misure delle diverse posizioni di equilibrio della molla:

\begin{equation*}
	z_i = z_i\,\pm\,\delta z_i 
\end{equation*}
%
dalle quali possiamo ricavare l'allungamento della stessa:

\begin{equation*}
	x_i = |z_i\,-\,z_0|
\end{equation*}
%
e quindi l’incertezza sugli allungamenti ($\delta x_i$) si ottiene applicando la regola per la propagazione delle incertezze sulla differenza di misure indipendenti:

\begin{equation*}
	\delta x_i = \sqrt{\delta z_i^2\,-\,\delta z_0^2}
\end{equation*}

\subsection{Elaborazione dei dati}

\subsubsection{Considerazioni inziali}
Come detto precedentemente, per misurare la massa dei dischetti a nostra disposizione abbiamo utilizzato una bilancia elettronica con una risoluzione di 0.1 grammi. Abbiamo osservato che lo strumento era abbastanza sensibile da rilevare variazioni nella pressione dell'aria circostante. Lo abbiamo appurato soffiandoci sopra e notando che il valore rilevato dalla bilancia aumentava in modo considerevole. Per questo motivo per trovare le varie masse dei nostri pesi ci siamo assicurati di stare quato più lontani dalla bilancia ci fosse concesso e di evitare di scuotere il tavolo di lavoro.
E' importante sottolineare che la massa dei nostri tredici pesi è data da una composizione dei dischetti a nostra disposizione e ogni dischetto è affetto da un errore sulla sua massa, dovuto principamente alla risoluzione dello strumento:

\begin{equation*}
	\delta m_i = \frac{\Delta m}{2} = 0.05 g 
\end{equation*}
%
abbiamo deciso di non usare l'incertezza tipo sull'errore di misurazione in quanto, come detto sopra, le misure potevano non risultare così fedeli in quanto soggette anche a variazioni repentine dell'ambiente esterno.
Quindi per evitare di dover propagare le incertezze delle singole masse alle loro somme, abbiamo misurato in aticipo la massa di ogni composizione così che risultasse affetta soltanto dall'errore di risoluzione dello strumento.
Dal momento che per calcolare la costante elastica della molla siamo costretti a passare per la relazionie $F_{el} = -k\,x$ e poichè nel nostro caso la $F_{el} = F_{p} = m\,\,g$ allora dobbiamo propagare l'incertezza derivante dalla misura della massa usata anche sul calcolo della forza peso applicata alla molla ottenendo quindi una relazione di questo tipo:

\begin{equation*}
	\delta F_{p}\, =\, g\,\delta m_i
\end{equation*}
<<<<<<< HEAD
%
Per quanto riguarda gli errori relativi all'errore di misura dell'allungamento della molla rispetto alla sua posizione di equilibrio le considerazioni sono quelle esposte nel paragrafo precedente. C'è però da aggiungere che la posizione iniziale della molla è stata calcolata con il piattello portapesi già agganciato ad essa.\\
E' importante sottolineare che abbiamo deciso di rilevare l'allungamento della molla sulla scala millimetrata con risoluzione del millimetro in quanto abbiamo ritenuto di non riuscire a sfruttare la scala con risoluzione di mezzo millimetro. Questo è dovuto principalmente al fatto che nella lettura della riga la molla non era completamente ferma ma era soggetta ad uan leggere oscillazione. Questa continuo moto oscillatorio unito anche al probabile errore di parallasse commesso nella lettura dello strumento ci ha convito a porre come incertezza sulle varie miure dell'allungamento un delta di un millimetro. Questo giustifica anche la stima fatta della posizione di equilibrio della molla $z_0$ che risulta anch'essa affetta da un errore di un millimetro.
=======
Per quanto riguarda gli errori (relativi all'errore) di misura dell'allungamento della molla rispetto alla sua posizione di equilibrio le considerazioni sono quelle esposte nel paragrafo precedente. Bisnogna però aggiungere che la posizione iniziale della molla è stata calcolata con il piattello portapesi già agganciato ad essa.\\
E' importante sottolineare che abbiamo deciso di rilevare l'allungamento della molla sulla scala millimetrata con risoluzione del millimetro in quanto abbiamo ritenuto di non riuscire a sfruttare la scala con risoluzione di mezzo millimetro. Questo è dovuto principalmente al fatto che nella lettura della riga la molla non era completamente ferma ma era soggetta ad uan leggere oscillazione. Questo continuo moto oscillatorio unito anche al probabile errore di parallasse commesso nella lettura dello strumento ci ha convito a porre come incertezza sulle varie miure dell'allungamento un delta di un millimetro. Questo giustifica anche la stima fatta della posizione di equilibrio della molla $z_0$ che risulta anch'essa affetta da un errore di un millimetro.
>>>>>>> f04481e5181b3e7888386eb641837b13d89a9525

\subsubsection{Analisi di dati:}
Nella \ref{tabella allungamenti} sono riportate le masse applicate alla molla con la loro relativa posizione di equilibrio. Stessa cosa vale per la \ref{tabella carico e scarico}.
La \ref{tabella carico e scarico} mostra la posizione di equilibrio per alcune masse applicate ed è stata realizzata con una procedura di carico e scarico della molla. Ovvero: per le masse scelte si è deciso di caricare la molla e rilevare le varie posizioni di equilibrio, stessa cosa si è fatta nello scariare la molla. Questa procedura è stata ripetuta due volte.
Come si può intuire da una prima anlsi superficiale delle tabelle risulta evidente che le misure sono compatibili tra di loro e questo quindi ci porta a dire che la molla non è stata affeta da una deformazione dovuta ad un carico eccessivo. Si può inoltre osservare che i risultati non dipendono dalla procedura con cui è stato misurato l'allungamento (carico o scarico) e che sono riproducibili entro il carico massimo della molla oltre il quale si avrebbe una deformazione permanente della stessa.
Affermiamo questo perchè prendendo per esempio la massa da "55 g" possiamo oservare che le varie misure sono le seguenti:
\begin{itemize}
	\item{primo allungamento $x_{all_1}$ = $(5.7 \pm 0.1) cm$}
	\item{secondo allungamento $x_{all_2}$ = $(5.6 \pm 0.1) cm$}
	\item{terzo allungamento $x_{all_3}$ = $(5.6 \pm 0.1) cm$}
	\item{quarto allungamento $x_{all_4}$ = $(5.5 \pm 0.1) cm$}
	\item{quinto allungamento $x_{all_5}$ = $(5.6 \pm 0.1) cm$}
\end{itemize}
quindi facendo una media gli ultimi quattro allungamenti, che sono stati ricavati dalla proceura di carico e scarico ricaviamo che

\begin{equation*}
	m^*[x_{all_i}] = \frac{1}{4} \sum_{i=2}^{5} (x_{all_i}) = 5.6/, cm
\end{equation*}
%
e l'incertezza su questo valore è la seguente:

\begin{equation*}
	\sigma^*[m^*[x_{all_i}]] = \sqrt{\frac{1}{4} \sum_{i=2}^{5} (x_{all_i} - m^*[x_{all_i}])^2} = 0.1/, cm
\end{equation*}
<<<<<<< HEAD
%
Quindi con questi valori possiamo valutare se la media delle misure ottenute dalla procedura di carico e scarico è compatibile con la misura ottenuta durante il primo ciclo di misurazione. Posto a priori un fattore di copertura k = 1.5 andiamo a verificare che il resto tra la differenza fra $m^*[x_{all_i}]$ e $x_{all_1}$ risulti essere minore dell'errore relativo al resto moltiplicato per il fattore di copertura.
=======
Con questi valori possiamo pertanto valutare se la media delle misure ottenute dalla procedura di carico e scarico è compatibile con la misura ottenuta durante il primo ciclo di misurazione. Posto a priori un fattore di copertura k = 1.5 andiamo a verificare che il resto tra la differenza fra $m^*[x_{all_i}]$ e $x_{all_1}$ risulti essere minore dell'errore relativo al resto moltiplicato per il fattore di copertura.
>>>>>>> f04481e5181b3e7888386eb641837b13d89a9525
Otteniamo quindi:

\begin{equation*}
	R = x_{all_1} - m^*[x_{all_i}] = 0.1 \,\,cm
\end{equation*}

\begin{equation*}
	\sigma_{R} = k\,\, \sqrt{(\sigma[x_{all_1}])^2 + (\sigma[m^*[x_{all_i}]])} = 0.2 \,\,cm
\end{equation*}
%
Da questo quindi possiamo dire che $m^*[x_{all_i}]$ e $x_{all_1}$ risultano essere compatibili, e compatibii sono anche le misure che abbiamo ottenuto per fare la media dei cicli di scarico e carico, sempre utilizzando come fattore di copertura il k scelto in precednza.
Poiché tale procedura si può estendere anche a tutte le altre misure relative alle masse ottenendo le stesse conclusioni, possiamo desumere che le misure sono tra di loro compatibili e non ci sono state deformazioni della molla.
Un'importante osservazione che si può fare analizzando \ref{tabella allungamenti} e \ref{tabella carico e scarico} è che per lo stesso carico applicato alla molla le misure dell'allungamento differiscono tra di loro per 0.1 cm. Abbiamo ipotizzato che questa differenza possa essere dovuta principalmente a due elementi:

\begin{enumerate}
	\item{dal momento che prima di effetuare i cicli di carico e scarico la molla era stata tolta dal supporto assieme al piattello portapesi, è probabile che nel riposizionare l'apparato l'allineamento del piattello portapesi, nella sua posizione di equilibrio, con la tacca dei 50 cm non sia stato perfetto ma sia risultato errato di 0.1 cm;}
	\item{un'altra possibilità è data da eventuali errori di parallasse, che non sono da escdere, in quanto riuscire a traguardare correttamente tra il bordo inferiore e l'asta millimtrata non è un'operazione così elementare come può sembrare;}
\end{enumerate}

\subsubsection{Calcolo della costante elastica K a partire dalla tabella}
Per calcolare la costante elastica della molla abbiamo deciso di procedere in questo modo:

Abbiamo calcolato la $F_{p_{i}}$ delle masse applicate alla molla, come riportato nella quinta colonna della tabella, e anche il rapporto $k_{i} = \frac{F_{p_{i}}}{x_{i}}$. Abbiamo anche stimato l'incertezza $\delta_{k_{i}}$ riguardante la costante elastica relativa ai vari pesi mediante la regola pr la propagazione delle incertezze sui rapporti di misure indipendenti:

\begin{equation*}
	\left(\frac{\delta {k_{i}}}{k_{i}}\right)^2  =  \left(\frac{\delta F_{p_{i}}}{F_{p_{i}}}\right)^2  +  \left(\frac{\delta_{x_{i}}}{x_{i}}\right)^2 
\end{equation*}
<<<<<<< HEAD
%
Queste incertezze sono riportate sulla sesta colonna della tabella e come si può osservare queste non risultano essere tutte uguali. In particolar modo abbiamo che il contributo dell'incertezza relativa al peso [ma è corretto?] sull'incertezza complessiva è trascurabile.\\
Ottenuti quindi tutti questi valori differenti di k per riunirli in un unico valore che riassuma l'esito complessivo di queste misurazioni abbiamo deciso di adottare due procedure differenti e verificare quele ci avrebbe dato il risultato più reciso e perchè.
=======
Queste incertezze sono riportate sulla sesta colonna della tabella [(!) che tabella?] e come si può osservare queste non risultano essere tutte uguali. In particolar modo abbiamo che il contributo dell'incertezza relativa al peso [ma è corretto?] sull'incertezza complessiva è trascurabile.\\
Ottenuti quindi tutti questi valori differenti di k per riunirli in un unico valore che riassuma l'esito complessivo di queste misurazioni abbiamo deciso di adottare due procedure differenti e verificare quele ci avrebbe dato il risultato più Preciso e perchè.
>>>>>>> f04481e5181b3e7888386eb641837b13d89a9525

\paragraph{Primo metodo\\}
Il primo metodo che prendiamo in considerazione è quello di calcolare la media pesata dei valori $k_{i}$ e per farlo ci serviremo delle seguenti formule:

\begin{equation*}
		k_{0} = \frac{\sum (k_i w_i)}{\sum (w_i)} , \,\,\,\,\,\, \delta k_0 = \frac{1}{\sqrt{\sum (w_i)}} , \,\,\,\,dove\,\,\,\, w_i = \frac{1}{(\delta k_i)^2}
\end{equation*}
%
in particolare $k_0$ appresenta il risultato della media pesata delle varie costanti elastiche ($k_i$) ottenute, $\delta k_0$ simboleggia l'errore relativo a $k_0$ e $w_i$ è il peso relativo alla misura $k_i$, ovvero quanto quella misura contribuisce nella media complessiva. E' importente sottolineare che la procedura non è in grdo di tenere conto di eventuali altre sorgenti di incertezza come ad esempio l'errore sistematico.

Quindi con i dati a nostra disposizione abbiamo otteuto i seguenti risultati:
\begin{itemize}
	\item{  $  k_{0} = 9.64 \, N/m  $  }
	\item{  $  \delta k_0 = 0.0143 \,\, N/m  $  }
\end{itemize}
che portano alla misura della costante elastica:
\begin{equation}
		k = 9.640 \,\, \pm \,\, 0.015\,\,N/m
\end{equation}

\paragraph{Secondo metodo\\}
Il secondo metodo che addottiamo per il calcolo della costante elastica della molla elicoidale è quello di considerare la distribuzine dei valori $K_i$, ovvero se ne deve calcolare la media campionaria $m^*_k$ e stimarne lo scarto quadratico medio. In questo modo otteniamo:

\begin{equation*}
	k_0 \,\,=\,\, m^*_k \,\,=\,\, \frac{1}{13}\,\,\sum_{i=1}^{13} (k_i) \,\,=\,\, 9.58 \,\,N/m
\end{equation*}
%
\begin{equation*}
	\delta k_0 \,\,=\,\, \sigma_{k_0} \,=\, \sqrt{\frac{1}{13}\,\,\sum_{i=1}^{13} (k_i - m^*_k)^2} \,\,=\,\, 0.156 \,\,N/m
\end{equation*}
<<<<<<< HEAD
%
Quindi mettendo a confronto queste due procedure e aiutandoci con i dati riportati in tabella possiamo osservare che la media pesata sembrerebbe essere la procedura che meglio approssima il risultato dell'esperimento. Affermiamo questo perchè dalla tabella si può oservare che le prime quattro misurazioni fatte risultano avere un incertezza sulla costante elastica dieci volte superiore alle altre misure effettuate, e quindi ci sembra doveroso che nel calcolo di $k_0$ si tenga in considerazione della differente accuratezza delle misure dando più importanza a quelle con un incertezza inferiore rispetto alle altre.
=======
ottenendo la misura
\begin{equation}
		k = 9.60 \,\, \pm \,\, 0.15 \,\, N/m
\end{equation}

Quindi mettendo a confronto queste due procedure e aiutandoci con i dati riportati in tabella possiamo osservare che la media pesata sembrerebbe essere la procedura che meglio approssima il risultato dell'esperimento. Affermiamo questo perchè dalla tabella si può oservare che le prime quattro misurazioni fatte risultano avere un incertezza sulla costante elastica dieci volte superiore alle altre misure effettuate, e quindi ci sembra doveroso che nel calcolo di $k_0$ si tenga in considerazione della differente accuratezza (o precisione?) delle misure dando più importanza a quelle con un incertezza inferiore rispetto alle altre.
>>>>>>> f04481e5181b3e7888386eb641837b13d89a9525

Una spiegazione al perchè in queste prime quattro misure l'incertezza risulti essere maggiore rispetto alle altre si può trovare nel fatto che i carichi appesi alla molla non fossero sufficientemente importanti, e quindi la molla non [ha lavorato correttamente.]. Nonostante questo però non possiamo dire che la legge di Hooke non sia valida nel regime di carichi "relativamente bassi": infatti per quanto si può osservare dai dati raccolti l'alungamento della molla ha avuto un andamento lineare sin dall'inizio delle misurazioni.

\subsubsection{Calcolo della costante elastica K a partire dal grafico}
Ricordando le seguenti relazioni:

\begin{equation}
	x \,\,=\,\, bF_p \,\,\,\,\,ovvero\,\,\,\,\, F_p \,\,=\,\, kx \,\,\,\,\,pertanto\,\,\,\,\, b \,\,=\,\, \frac{1}{k}
\end{equation}
<<<<<<< HEAD
%
possiamo procedere al calcolo della costante elastica k a partire direttamente dal garfico. Per farlo però e necessario stare attenti al fatto che i punti sperimentali sono affetti da incertezza, pertanto per trovare la retta che meglio approssima il loro andameno i possono seguire due procedure distinte:
=======
possiamo procedere al calcolo della costante elastica k a partire direttamente dal garfico. Per farlo però è necessario stare attenti al fatto che i punti sul grafico sono ricavati sperimentalmente e di conseguenza sono affetti da incertezza, pertanto, per trovare la retta che meglio approssima il loro andamento, si possono seguire due procedure distinte:
>>>>>>> f04481e5181b3e7888386eb641837b13d89a9525

\paragraph{Prima procedura: Metodo della retta minima e massima\\}
Come suggerisce il titolo si tratta di una procedura molto semplice e molto veloce, ma anche abbastanza imprecisa in quanto consiste nel misurare la pendenza della retta massima e della retta minima direttamente dal grafico e successivamente calcolarne la media per ottenere il valore medio $k_0$. Per ottene l'icertezza su $k_0$ basta fare la differenza tra i due valori massimo e minimo e dividere per due.\\
Nel caso in cui le barre d'errore risultassero troppo piccole da rendere impossibile distinguere le due rette - come nel nostro caso - allora il calcolo di $k_0$ risulterebbe immediato in quanto sarebbe la pendenza della retta.\\
Seguendo questa prima procedura otteniamo:

[Vediamo cosa otteniamo] WTF ????

\paragraph{Seconda procedura: Regressione lineare\\}
Questa procedura risulta essere più precisa della precedente e si basa sul metodo dei minimi quadrati.\\
La stima migliore del coefficiente angolare b è data dal valore che rende minima la discrepanza tra i dati sperientali $F_{p_i} \,\,e\,\, x_i$ e la retta $x \,=\, bF_p$.\\
Procediamo operativamente in questo modo:
\begin{itemize}
	\item{misura della discrepanza:
			\begin{equation*}
				discrepanza \,\,=\,\, \sum_{i=1}^{13} \frac{(x_i - bF_{p_i})^2}{(\delta x_i)^2} \,\,=\,\,		
			\end{equation*}
			ricordando che consideriamo le incertezze $\delta F_{p_i}$ trascurabili rispetto alle incertezze $\delta x_i$. [non sono sicuro su quale sia trascurabile] Notiamo inoltre che le incertezze di $x_i$ hanno tutte lo stesso valore.}
	\item{quindi otteniamo le segenti relazioni:
			\begin{equation*}
				b_0  \,=\,  \frac{\sum F_{p_i}  x_i}{\sum F_{p_i}^2} \,=\, 0.104(00) \, m/N \,\,\,\,\,\,\, e \,\,\,\,\,\,\,
				\delta b  =  \frac{\delta x_i}{\sqrt{\sum F_{p_i}^2}} \,=\, 0.00015(4) \, m/N
			\end{equation*}
			e quindi sfruttando la regola della propagazione dell'incertezze sul quoziente abbiamo
			\begin{equation*}
				\frac{\delta k}{k} \,=\, \frac{\delta b}{b}
			\end{equation*}
			in modo da ottenere:
			\begin{equation*}
				k \,=\, 9.640 \pm 0.015 \,\,N/m <-- come hai fatto ad ottenerlo?
			\end{equation*}
			}
\end{itemize}
Possiamo notare che la costante elastica così ottenua è uguale a quella ottenuta con la media pesata. Questo è dovuto al fatto che le due procedure sono equivalenti.

\subsection{Test del chi quadro}
<<<<<<< HEAD
Fino a questo punto noi abbiamo valutato la compatibilità dei valorisperimentali con la legge di Hooke esaminando qualitativamente il grafico. noi possiamo però valutare la compatibilità dei dati anche con un metodo quantitativo che si basa sul test del chi quadro. Questo test confronta punto per punto la discrepanza tra i valori sperimentali e la retta teorica ($(x_i - b_0 F_{p_i})$) e ci si dovrebbe aspettare che questa per ogni punto sia mediamente paragonabile con l'incertezza ($\delta x_i$). Ricordiamo che la discrepanza viene calcolata sui valori dell'ordinata, inoltre questa verifica prsuppone che l'errore sui valori in asissa sia trascurabile.
Pertanto in base alla nostra analisi noi abbiamo che l'icertezza sul peso risulta essere trascurabile rispetto  a quella sull'allungamento, come richiesto dal test, e grazie a (4) sull'ordinata del nostro grafico sono riportati i valori dell'allungamento della molla.\\
=======
Fino a questo punto noi abbiamo valutato la compatibilità dei valori sperimentali con la legge di Hooke esaminando qualitativamente il grafico. Possiamo però valutare la compatibilità dei dati anche con un metodo quantitativo che si basa sul test del chi quadro. Questo test confronta punto per punto la discrepanza tra i valori sperimentali e la retta teorica
(!) ($(x_i - b-0 F_{p_i})$) questo è da sistemare!
 e ci si dovrebbe aspettare che questa, per ogni (punt =? ) dato sperimentale, sia mediamente paragonabile con l'incertezza ($\delta x_i$). Ricordiamo che la discrepanza viene calcolata sui valori dell'ordinata, inoltre questa verifica presuppone che l'errore sui valori in ascissa sia trascurabile.
(!) da riformulare (!) Pertanto in base alla nostra analisi noi abbiamo che l'icertezza sul peso risulta essere trascurabile rispetto  a quella sull'allungamento, come richiesto dal test, e grazie a (4) sull'ordinata del nostro grafico sono riportati i valori dell'allungamento della molla. (!) da riformulare (!)\\
>>>>>>> f04481e5181b3e7888386eb641837b13d89a9525
Procediamo quindi col test del chi quadro:
\begin{itemize}
\item{calcoliamo il chi quadro (osservato):
	\begin{equation*}
		\chi^2 \,=\, \sum_{i=1}^{13} \frac{(x_i - b_0 F_{p_i})^2}{(\delta x_i^2)} = 28.9
	\end{equation*}
	dove $b_0$ rappresenta il coefficiente angolare determinato mediante la regressione lineare}
\item{se il test va a buon fine ci si aspetterebbe che:
	\begin{equation*}
		\chi^2 \,=\, \sum_{i=1}^{13} \frac{(x_i - b_0 F_{p_i})^2}{(\delta x_i^2)} \simeq N - 1 \,=\, 12
	\end{equation*}
	dove N rappresenta l'insieme di tutti i punti sperimentali. Abbiamo N-1 in quanto dobbiamo tener conto che un parametro della curva (retta) teorica, ovvero $b_0$, che è stato ricavato a partire da punti sperimentali}
\end{itemize}
Nel nostro caso il chi quadro non è compatibile con il vaore N-1, anzi risulta essere più del doppio.
Per questo motivo abbiamo come prima cosa verificato che questo non fosse dovuto all'influenza di particolari punti ad esempio il primo o l'ultimo. A tal fine abbiamo controllato che:

\begin{equation*}
	\frac{(x_i - b_0 F_{p_i})^2}{(\delta x_i)^2} \,\,\, non fosse >> 1
\end{equation*}
%
[dal momento che non si è verificata questa eventualita siamo sicuri che tutti i nostri dati sono stati presi con sufficiente accuratezza...]
Per questo motivo le soluzioni possibili sono due: la prima è che idati sperimentali non sono compatibili con la legge di Hooke, caso che ci sentiamo di escludere in quanto è stato provato ripetutamente che la legge in questione è valida. Perciò non rimane che ammettere di aver sottostimato le incertezze della grandezza $x_i$ che rappresenta l'allungamento.\\
Quindi abbiamo deciso di procedere ad aggiustare l'inceretezza relativa a $x_i$. Abbiamo operato nel seguente modo:
\begin{itemize}
\item{Abbiamo deciso di ricalcolare a "posteriori" le incertezze, così facendo cerchiamo di avvicinarci col chi quadro osservato ($\chi_{oss}^2$) al chi quadro teorico ($\chi_{teo}^2$) il più possibile}
\item{Abbiamo deciso di considerare uguali tra di loro tutte le varianze in modo da poter procedere con la seguente operazione:
	\begin{equation*}
		\chi_{oss}^2 \,=\, \sum_{i=1}^{13} \frac{(x_i - b_0 F_{p_i})^2}{(\delta x_i^2)} \,=\, \chi_{teo}^2 
	\end{equation*}
	e così facendo troviamo un coefficiente che moltiplicato per il precedente $\delta x_i$ ci restituische una nuova varianza grazie alla quale i due chi risulteranna uguali. Infatti abbiamo ottenuto:
	\begin{equation*}
		\chi_{oss}^2 \,=\, \frac{1}{\delta x_{i}^2} \sum_{1}^{13} (x_i - b_0 F_{p_i})^2  \,=\, \chi_{teo}^2
	\end{equation*}
	poichè sappiamo che:
	
	\begin{equation*}
		\chi_{teo}^2 \,=\, \nu \,\,\,\,\, con \,\,\,\,\,
		\nu \,=\, N - 1 \,\,\,\,\, e \,\,\,\,\,
		\delta x_{iposteriori}^2 \,=\, \frac{\chi_{oss}^2}{\nu} \, \delta x_{ipriori}^2 \,=\, 0.000633 \,\, m^2
	\end{equation*}
	con $\nu$ ch rappresenta i gradi di libertà.
	Quindi si ottiene che:
	\begin{equation*}
		\delta x_{iposteriori} \,=\, 0.00155 \,\, m \,=\, 1.5 \,\, mm
	\end{equation*}}	 
\end{itemize}
Quindi grazie a tutto questo abbiamo ottento che il $\chi_{oss}^2$ combacia con $\chi_{teo}^2$ soltanto dopo aver modificato l'incertezza relativa lle misure dell'allungamento della molla aumentandola di mezzo millimetro cioè portandola dal doppio dell'incertezza massima di misura pari ad 1 millimetro ad un'incertezza di 1 millimetro e mezzo. Crediamo che questa variazione sia più che accettabile in quanto, pur avendo già cercato di arginare gli errori di tipo A e B di misura, raddoppiando l'incertezza massima di misura sembra non sia stato sufficiente. Infatti come accennato all'inizio dell'analisi dei dati quando si cercava di leggere lo strumento la misura poteva non risultare così precisa a causa delle continuè oscillazioni della molla, che è a tutti gli effetti un oscillatore armonico, e un probabile errore di parallasse[potremmo provare a stimarlo?].\\
Dal momento che abbiamo variato l'errore sulla misura dell'allungamento otteniamo come nuovo valore della costante elastica della molla il seguente risultato:

\begin{equation*}
	k_0 \,\,=\,\, 9.64 \pm 0.0222 \,\, N/m
\end{equation*}
%
dove ovviamente l'unico cambiamento riguarda l'incertezza sulla misura che risuta essere maggiore di quella calcolata in precedenza per ovvi motivi.
<<<<<<< HEAD

=======
>>>>>>> f04481e5181b3e7888386eb641837b13d89a9525
