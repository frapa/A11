\section{Conclusioni}
In questo esperimento abbiamo utilizzato due diversi procedimenti meccanici per la misura di una stessa grandezza: la costante elastica della molla. I due procedimenti si differenziano per il diverso utilizzo della molla: il primo necessita di un uso statico (misurazione dell'elongazione), mentre il secondo di un uso dinamico (misurazione del periodo). Ogni procedimento ha portato all'ottenimento di un valore della costante: $k_{statico}$ e $k_{dinamico}$. Inoltre, ogni valore della costante è stato ottenuto a sua volta sia con procedure grafiche sia con procedure analitiche.
Seguire non una sola procedura, ma entrambe, ci ha permesso una maggiore sicurezza sulla validità del risultato ottenuto. Infatti, come mostrato nel precedente paragrafo, i coefficienti elastici ricavati con il metodo statico e con il metodo dinamico risultano compatibili, ponendo un buon fattore di copertura $k=3$.

In questo stesso esperimento ci interessava, inoltre, verificare la linearità della risposta della molla in funzione della forza ad essa applicata ed la abbiamo appurata in un range adeguato.
\\
\\


\textbf{Disclaimer}: Nonostante i componenti del gruppo B11 ci vogliano affibbiare la colpa del fatto che il loro esperimento non è andato a buon fine, noi decliniamo apertamente ed a gran voce ogni responsabilità.