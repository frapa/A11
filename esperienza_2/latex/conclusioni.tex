\section{Conclusioni}
In questo esperimento abbiamo utilizzato due diversi procedimenti meccanici per la misura di una stessa grandezza: la costante elastica della molla. I due procedimenti si differenziano per il diverso utilizzo della molla: il primo necessita di un uso statico (misurazione dell'elongazione), mentre il secondo di un uso dinamico (misurazione del periodo di oscillazione). Ogni procedimento ha portato all'ottenimento di un valore della costante elastica e, inoltre, ogni valore della costante è stato a sua volta ottenuto sia con procedure grafiche sia con procedure analitiche.
Seguire non una sola procedura, ma entrambe, ci ha permesso una maggiore sicurezza sulla validità del risultato ottenuto. Ogni procedura ha infatti funto da controprova per l'altra e, come mostrato nel precedente paragrafo, i coefficienti elastici ricavati con il metodo statico e con il metodo dinamico risultano compatibili, ponendo come fattore di copertura $k\,=\,3$.

In questo stesso esperimento ci interessava, inoltre, verificare la linearità della risposta della molla in funzione della forza ad essa applicata. Questo obiettivo è stato raggiunto e, come si può vedere dalle tabelle e dai grafici, la linearità è stata appurata in un adeguato range di valori.
È in aggiunta stato verificato che la molla non si deformasse in modo definitivo a causa dei pesi a cui era stata sottoposta, vanificando tutto l'esperimento.
\\
\\


\textbf{Disclaimer}: Nonostante i componenti del gruppo B11 ci vogliano affibbiare la colpa del fatto che il loro esperimento non è andato a buon fine, noi decliniamo apertamente ed a gran voce ogni responsabilità.