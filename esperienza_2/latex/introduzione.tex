\section{Introduzione}
La legge sperimentale che lega una forza applicata ad una molla e la sua deformazione rispetto alla sua posizione di equilibrio fu formulata da Robert Hooke nel 1675 ed è la seguente:
\begin{equation}
	F_{el} = -k\,x
\end{equation}
In questo esperimento lo scopo principale è quello di calcolare la costante elastica di una molla conoscendo, grazie alle misure effettuate, la deformazione della molla in funzione della forza applicata ad essa. Una seconda parte della relazione verterà sullo studio del moto oscillatorio e nell'esecuzione del test del chi al quadrato.
