\section{Introduzione}
La legge sperimentale che lega una forza applicata ad una molla e la sua deformazione rispetto alla sua posizione di equilibrio fu formulata da Robert Hooke nel 1675 ed è la seguente:

\begin{equation}
	F_{el} = -k\,x
	\label{hooke}
\end{equation}

In questo esperimento lo scopo principale è quello di calcolare la costante elastica di una molla con due differenti procedure.
La prima è quella di calcolare la deformazione della molla in funzione della forza applicata ad essa, seguendo l'equazione (1). Inoltre abbiamo verificato che la molla rispondesse linearmente ai carichi applicati (da 5 g a 125 g) e che non fosse avvenuta una deformazione permanente della stessa durante l'eperimento.

Il secondo metodo che adotteremo sarà quello di calcolare la costante elastica conoscendo la relazione che lega il periodo di oscillazione della molla e $\omega$:

\begin{equation}
	\mathcal{T} \,=\, \frac{2\,\pi}{\omega} \,=\, {2\,\pi}{\sqrt{\frac{m}{k}}} \quad\quad\quad dove \,\,  \omega \,=\, \sqrt{\frac{k}{m}}
\end{equation}