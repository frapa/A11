\section{Introduzione}

La legge sperimentale che lega una forza applicata ad una molla e la sua deformazione rispetto alla sua posizione di equilibrio fu formulata da Robert Hooke nel 1675 ed è la seguente:

\begin{equation}
	F_{el} = -k\,x
	\label{eq:hooke}
\end{equation}

Lo scopo principale di questa relazione e degli esperimenti da noi eseguiti,
è quello di calcolare la costante elastica di una molla con due differenti procedure.
La prima è quella di calcolare la deformazione della molla in funzione della forza applicata ad essa, seguendo l'equazione (\ref{eq:hooke}).
Stradafacendo, abbiamo anche verificato che la molla rispondesse linearmente ai carichi applicati (da 5 g a 125 g),
e quindi la corretezza della legge di Hooke, e che non si deformasse permanente durante l'eperimento.

Il secondo metodo che abbiamo adottato è quello di calcolare la costante elastica conoscendo la relazione
che lega il periodo di oscillazione della molla e $\omega$:

\begin{equation}
	\mathcal{T} \,=\, \frac{2\,\pi}{\omega} \,=\, {2\,\pi}{\sqrt{\frac{m}{k}}} \qquad \qquad \text{dove} \quad  \omega \,=\, \sqrt{\frac{k}{m}}
\end{equation}
