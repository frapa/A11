\section{Introduzione}
La legge sperimentale che lega una forza applicata ad una molla e la sua deformazione rispetto alla sua posizione di equilibrio fu formulata da Robert Hooke nel 1675 ed è la seguente:

\begin{equation}
	F_{el} = -k\,x
	\label{hooke}
\end{equation}

In questo esperimento lo scopo principale è quello di calcolare la costante elastica di una molla conoscendo, grazie alle misure effettuate, la deformazione della molla in funzione della forza applicata ad essa. [Inoltre abbiamo verificato la correttezza della legge di Hooke e quindi la linearità della risposta della molla ai carichi applicati, entro l'intervallo di ]

Un secondo metodo che adotteremo per calcolare questa costante elastica sarà quello di calcolarla conoscendo la relazione che lega il periodo di oscillazione della molla e $\omega$ ($\sqrt{\frac{k}{m}}$). La relazione è la seguente:

\begin{equation}
	\mathcal{T} = \frac{2\,\pi}{\omega} = {2\,\pi}{\sqrt{\frac{m}{k}}}
\end{equation}
Una seconda parte della relazione verterà sullo studio del moto oscillatorio e nell'esecuzione del test del chi al quadrato.
