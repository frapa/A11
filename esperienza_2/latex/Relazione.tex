\documentclass[11pt, twoside, a4paper]{article}
\usepackage[italian]{babel}
\usepackage[utf8]{inputenc}
\usepackage{amsmath}
\usepackage{fullpage}
\usepackage{graphicx}
\usepackage{booktabs}
\usepackage{wrapfig}
\usepackage{sidecap}

\begin{document}

\begin{titlepage}
\begin{center}
	\hrule \vspace{0.5cm}
     	\textsc{\LARGE Deformazione elastica e studio del moto di un oscillatore armonico}
	\vspace{0.5cm} \hrule \vspace{2cm}
      	{\large Francesco Pasa, Davide Bazzanella, Andrea Miani\\
		Gruppo A11}\\
	\vspace{0.5cm}
      	{\large 18 marzo 2013 - 1 aprile 2013}
	\vfill
	{\begin{abstract}
Misura della costante elastica di una molla elicoidale mediante due procedure diverse: misurazione statica e
dinamica. Test del chi al quadrato.
	 \end{abstract}}
\end{center}
\end{titlepage}

\newpage

\vspace*{\fill}
\begin{center}
	\tableofcontents
\end{center}
\vspace*{\fill}

\newpage

\section{Introduzione}
La legge sperimentale che lega una forza applicata ad una molla e la sua deformazione rispetto alla sua posizione di equilibrio fu formulata da Robert Hooke nel 1675 ed è la seguente:
\begin{equation}
	F_{el} = -k\,x
\end{equation}
In questo esperimento lo scopo principale è quello di calcolare la costante elastica di una molla conoscendo, grazie alle misure effettuate, la deformazione della molla in funzione della forza applicata ad essa. Una seconda parte della relazione verterà sullo studio del moto oscillatorio e nell'esecuzione del test del chi al quadrato.

\section{Ricerca della costante elastica della molla}

\subsection{Apparato sperimentale}
L'apparato sperimentale utilizzato è costituito da:
	\begin{itemize}
		\item{una base ad A sulla quale sono installati verticalmente un supporto con gancio di sospensione delle molle e asta millimetrata scorrevole. L'asta millimetrata presentava sul lato sinistro una scala millimetrata con una risoluzione di mezzo millimetro, mentre sul lato destro una scala millimetrata con una risoluzione del millimetro.}
		\item{quattro molle elicoidali con caratteristiche elastica differente tra di loro: in particolare una di esse è risultata essere troppo morbida per i carichi che saremmo andati ad applicare e quindi suscettibile di una probabile deformazione che avrebbe compromesso la buona riuscita dell'esperimento. Un'altra molla è stata scartata in quanto troppo poco elastica così che non ci avrebbe permesso di apprezzare una deformazione per carichi troppo leggeri. Per questi motivi la scelta è ricaduta su una delle due molle restanti.}
		\item{un piattello portapesi di massa $m_{piattello} = 25.2\,\,g \pm\, 0.1\,\,g$, quattro pesi cilindrici neri rispettivamente di massa nominale 5, 10, 25 e 50 grammi e quattro pesi argentei identici a quelli neri, con stesse masse nominali.}
		\item{una bilancia a un piattello con una risoluzione di 0.1 grammi}
	\end{itemize}

\subsection{Procedura di aquisizione dei dati}
Ritenendo essere noto che una forza F di trazione o compressione applicata, nel nostro caso, ad una molla elicoidale provoca una deformazione del corpo abbiamo deciso di sfruttare come forza agente la forza peso $F_{p}$ delle masse cilindriche a nostra disposizione. Per fare questo ricordiamo che la relazione tra una massa inerziale e la forza peso è la seguente:
\begin{equation}
	F_{p} = m\,\,g
\end{equation}
dove g rappresenta l'accelerazione di gravità e assumiamo valga: $g = 9.806\,\,m\,s^{-2}$.

Quindi per calcolare l'allungamento di una delle due molle elicoidali mediante il metodo statico abbiamo posizionato il sistema molla e piattello lungo l'asta millimetrata, appendendo un'estremità della molla al gancio di supporto. Atteso che l'oscillazione della molla si smorzasse il più ossibile abbiamo allineato il bordo inferiore del piattello, agganciato alla molla, con la tacca dei 50 cm dell'asta graduata, cercando di evitare eventuali errori di parallasse. In questo modo la posizione di equilibrio della molla risulta essere:
\begin{equation}
	z_{eq} = z_0\,\pm\,\delta(z_0) = 50.0\,cm\,\pm\,0.1\,cm
\end{equation}
Di seguito abbiamo verificato il peso nominale delle otto masse cilindriche a nostra disposizione[riferimento a tabella futura]. Successivamente abbiamo accoppiato i vari pesi a nostra disposizione per formare tredici masse che variano da circa cinque grammi a centoventicinque grammi con intervalli di dieci. Infine abbiamo misurato il differente allungamento della molla man mano che  si variava il peso applicato ad essa attendendo che le oscillazioni del sistema si smorzassero il più possibile. Così facendo abbiamo ottenuto una serie di misure delle varie posizioni di equilibrio della molla:
\begin{equation}
	z_i = z_i\,\pm\,\delta(z_i) 
\end{equation}
dalle quali possiamo ricavare l'allungamento della stessa:
\begin{equation}
	x_i = |z_i\,-\,z_0|
\end{equation}
e quindi l’incertezza sugli allungamenti ($\delta(x_i)$) si ottiene applicando la regola per la propagazione delle incertezze sulla differenza di misure indipendenti:
\begin{equation}
	\delta(x_i) = \sqrt{\delta(z_i)^2\,-\,\delta(z_0)^2}
\end{equation}

\subsection{Elaborazione dei dati}













\end{document}