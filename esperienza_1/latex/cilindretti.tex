\section{Cilindretti}

\subsection{Descrizione della procedura di misura}
Procediamo ora con la descrizione della procedura operativa con cui
abbiamo misurato i 25 cilindri di metallo con i tre strumenti di
misura a nostra disposizione:

\subsubsection{Metro a nastro}
Abbiamo misurato i 25 cilindretti con un metro a nastro la cui risoluzione
di misura è 1 millietro, in quanto abbiamo ritenuto di non poter distinguere
un intervallo di mezzo millimetro. Per questo motivo abbiamo approssimato
le misure alla tacca da un millimetro più vicina al bordo del cilindretto.
Poichè la lunghezza dei cilindretti era di circa 15 mm abbiamo incontrato
delle difficoltà nell'allineare i cilindri con lo 0 dello strumento.
Per questo motivo abbiamo allineato un'estremità di ogni cilindro con la
tacca dei 10 cm del metro e abbiamo sottratto questa quantità al valore
della misurà così ottenuta. E' importante evidenziare che anche in questa
operazione abbiamo riscontrato una certa difficoltà nell'allineare correttamente
i cilindri anche con la tacca dello strumento, ma ci è sembrato che la
misura così ottenuta fosse più precisa di quella riscontrata allineandoli
con lo 0 dello strumento. Ciò nonostante abbiamo tentato di evitare
eventuali errori di parallasse durante la lettura dello stumento.

\subsubsection{Calibro ventesimale}
Il calibro che abbiamo utillizzato ha una risoluzione di misura di 0.05
millimetri. La misura con il calibro ventesimale è stata effettuata avendo
cura di disporre i cilindri in modo perpendicolare ai becchi del calibro,
a tal fine abbiamo sfruttato le scanalature presenti sullo strumento.
In questo modo abbiamo tentato di misurare l'asse del cilindro, tuttavia
non abbiamo potuto evitare nella maniera più assoluta errori di posizione
dei cilindri.
La lettura dei ventesimi di millimetro è stata effettuata cercando quale
tacca della scala mobile ventesimale si allineasse meglio con le tacce
sovrastanti della scala millimetrata. Questa operazione è stata eseguita
da tutti i membri del gruppo per evitare eventuali problemi di parallasse
dovuti ad una singola valutazione della misura.

\subsubsection{Micrometro}
Il micrometro è uno strumento con una risoluzione di misura di 0.01
millimetro. Anche in questo caso la misura dei cilindri è stata effettuata
prestando attenzione che i campioni fossero perpendicolari alle aste di
misurazione, per evitare lo stesso tipo di errore descritto sopra.
Come nel caso del calibro non possiamo essere sicuri del perfetto
allineamento dei cilindri, ma abbiamo tentato di ridurre al minimo
questa problematica.

\begin{table}[tb]
	\footnotesize
	\centering
	\begin{tabular} { c c c c c | c c c c c | c c c c c }
		\toprule
		\multicolumn{15}{c}{Lunghezza dei cilindretti} \\
		\multicolumn{5}{c}{Metro [mm]} & \multicolumn{5}{c}{Calibro [mm]} & \multicolumn{5}{c}{Micrometro [mm]} \\
		\midrule
		14 & 14 & 14 & 14 & 14 & 13.70 & 13.85 & 13.95 & 13.80 & 13.95 & 13.94 & 13.92 & 13.84 & 13.93 & 13.79 \\
		14 & 14 & 14 & 14 & 14 & 13.70 & 13.95 & 13.90 & 13.90 & 13.90 & 13.94 & 13.93 & 13.89 & 13.92 & 13.91 \\
		14 & 14 & 14 & 14 & 14 & 13.70 & 13.95 & 13.90 & 13.95 & 13.85 & 13.93 & 13.93 & 13.84 & 13.90 & 13.74 \\
		14 & 14 & 14 & 14 & 14 & 13.70 & 13.95 & 13.90 & 13.90 & 13.90 & 13.70 & 13.85 & 13.91 & 13.79 & 13.70 \\
		13 & 14 & 14 & 14 & 14 & 13.75 & 13.90 & 13.80 & 13.85 & 13.90 & 13.71 & 13.92 & 13.91 & 13.94 & 13.70 \\
	\bottomrule
	\end{tabular}
	\caption{Misure della lunghezza dei 25 cilindretti ottenute con i tre strumenti a nostra disposizione.
        È riportato il valore di lettura degli strumenti.}
	\label{tab:cilindretti}
\end{table}

\subsection{Analisi dei dati}
La seguente tabella \ref{tab:cilindretti} riporta tutte le 25 misure di
lunghezza dei cilindri effettuate con ognuno dei tre strumenti a nostra
disposizione:

L'istogramma in figura \ref{fig:metro} rappresenta la distribuzione della lunghezza della
popolazione dei cilindretti ottenuta con il metro a nastro. Per costruire
questo istogramma abbiamo deciso di utilizzare come binnaggio la risoluzione
di misura dello strumento e quindi di 1 millimetro, poichè tutte le misure
effettuate sono risultate essere 14 $\pm$ 0.5 mm tranne una che ha valore
13 $\pm$ 0.5 mm. Come si può osservare dal grafico non siamo riusciti ad
apprezzare la distribuzione non uniforme della lunghezza dei cilindri,
in quanto la risoluzione dello strumento non è stata sufficiente per
apprezzare le variazioni di lunghezza da un corpo all'altro.

\begin{SCfigure}[0.6][bt]
	\centering
	\includegraphics[width=100mm]{grafici/Cilindretti_metro.pdf}
	\caption{Istogramma relativo alle misure di lunghezza dei cilindri con il metro a nastro}
	\label{fig:metro}
\end{SCfigure}

I parametri statistici ottenuti relativi alle misure dei cilindri sono
i seguenti:

% se vuoi elenco numerato sostituisci enumerate al posto di itemize
% ovviamente devi cambiare anche \end{itemize}
\begin{itemize}
    \item{Media campionaria:}
        \begin{equation}
        m^*[x] = \frac{1}{N} \sum_{i=1}^{N} (x_i) \simeq \sum_{j=1}^{\mathcal{N}} (x_j p_j^*) = 13.96\,\,mm 
        \end{equation}

    \item{Varianza campionaria:}
        \begin{equation}
        \tilde{D} = \frac{1}{N - 1} \sum_{i=1}^{N} (x_i - m^*[x])^2 = 0.04\,\,mm^2
        \end{equation}

    \item{Deviazione standard campionaria:}
        \begin{equation}
        \tilde{\sigma} = \sqrt{\frac{1}{N - 1} \sum_{i=1}^{N} (x_i - m^*[x])^2} = 0.2\,\,mm
        \end{equation}
\end{itemize}

Gli istogrammi figura \ref{fig:calmic} rappresentano rispettivamente la distribuzione delle
misure della lunghezza dei cilindri ottenute con il calibro ventesimale e con
il micrometro.

\begin{wrapfloat}{figure}{r}{0pt}
	\centering
	\includegraphics[width=100mm]{grafici/Cilindretti_calibro_micrometro.pdf}
	\caption{I due grafici riportano le lunghezze dei cilindri, misurati con calibro
        ventesimale e micrometro. Come si può notare i grafici sono identici a parte
        le ultime due colonne, dove un dato "ha cambiato" bin.}
	\label{fig:calmic}
\end{wrapfloat}

Per l'istogramma relativo al calibro ventesimale la scelta del binning è legata
alla risoluzione dello strumento e quindi la larghezza di ogni singolo
bin è di 0.05 mm. I dati sono infatti abbastanza numerosi per avere un numero
di conteggi sufficienti per ogni colonna.

Per l'istogramma relativo al micrometro, invece, si è scelto di utilizzare lo stesso
binnaggio utilizzato per il calibro. Abbiamo effettuato questa scelta poiché se avessimo
diminuito la larghezza dei bin avremmo ottenuto un numero di colonne troppo alto e per
ognuna di esse un numero di conteggi insufficiente. Al contrario se avessimo scelto
di aumentare la larghezza dei bin allora lo scopo di misurare i 25 cilindretti
con il micrometro sarebbe risultato vano in quanto non si sarebbe nemmeno apprezzata
la buona risoluzione dello strumento. D'altra parte questa scelta rende i grafici 
confrontabili direttamente.

Come si può notare i due istogrammi si assomigliano: infatti in entrambi i casi
siamo riusciti ad apprezzare la presenza di un picco della popolazione in
corrispondenza del valore 13.90 mm. Nonostante le misure risultino essere
più accurate rispetto a quelle eseguite con il metro a nastro non ci sono
abbastanza campioni per poter affermare l'esistenza di più di una popolazione
in quanto non sono distinguibili altri picchi dalle code relative al picco preponderante.
Affermiamo questo perchè l'errore su ogni colonna è $\sqrt{n_j}$ %$\sqrt{\frac{n_j}{N \Delta x}}$
e il numero di campioni per questa misura risulta essere inferiore a 5.

\newpage
Più nel dettaglio possiamo dire che i dati statistici riguardanti le misure con il calibro ventesimale e il micrometro (\ref{fig:calmic}) sono:
\begin{itemize}
    \item{Media campionaria:}
        \begin{equation}
        m^*_{calibro}[x] \simeq 13.xx\,\,mm \qquad
        m^*_{micrometro}[x] \simeq 13.xx\,\,mm 
        \end{equation}

    \item{Varianza campionaria:}
        \begin{equation}
        \tilde{D}_{calibro} = x.xx\,\,mm^2 \qquad
        \tilde{D}_{micrometro} = x.xx\,\,mm^2
        \end{equation}

    \item{Deviazione standard campionaria:}
        \begin{equation}
        \tilde{\sigma}_{calibro} = x.x\,\,mm \qquad
        \tilde{\sigma}_{micrometro} = x.x\,\,mm
        \end{equation}
\end{itemize}
Inoltre abbiamo che:

\begin{itemize}
    \item{Mediana:}
        \begin{equation}
        m_{calibro}[x] \simeq 13.xx\,\,mm \qquad
        m_{micrometro}[x] \simeq 13.xx\,\,mm 
        \end{equation}

    \item{Quantile 10\%:}
        \begin{equation}
        q_{10\%calibro} = x.xx\,\,mm^2 \qquad
        q_{10\%micrometro} = x.xx\,\,mm^2
        \end{equation}

    \item{Quantile 90\%:}
        \begin{equation}
        q_{90\%calibro} = x.x\,\,mm \qquad
        q_{90\%micrometro} = x.x\,\,mm
        \end{equation}
\end{itemize}
% Inserire oltre  a Media campionaria, Varianza campionaria, Fluttuazione standard
% campionaria anche mediana e quantili
% Ulteriori commenti

\begin{SCfigure}
	\centering
	\includegraphics[width=100mm]{grafici/cilindri_tutti.pdf}
	\caption{Misure della lunghezza dei cilindretti ottenute da tutti i gruppi del
        laboratorio (gruppi del lunedì). L'istogramma riporta le misure effettuate con
        il metro a nastro.}
\end{SCfigure}

\begin{wrapfloat}{figure}{l}{0pt}
	\centering
	\includegraphics[width=100mm]{grafici/cilindri_tutti_2.pdf}
    \caption{Lunghezza dei cilindretti ottenute da tutti i gruppi (del lunedì)
        del corso di laboratorio. L'istogramma relativo alle misure con il calibro
        ha binning di 0.05 mm, mentre nell'istogramma delle misure effettuate con
        il micrometro i bin sono da 0.02 mm}
\end{wrapfloat}

\subsection{Analisi dei dati a livello globale di laboratorio}
Contrariamente a quanto abbiamo potuto osservare e ipotizzare tramite i dati
raccolti dal nostro singolo gruppo con i tre strumenti, unendo i dati relativi
a tutti i gruppi di laboratorio, si può evidenziare l'esistenza di due popolazioni
distinte di cilindri. Infatti nell'istogramma (1.3) sono ben distinguibili due picchi,
che corrispondono alle due rispettive medie campionarie di lunghezza delle due
popolazioni di cilindretti.

