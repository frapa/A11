\newpage

\section{Pendolo}

La seconda parte di questa relazione tratta l'esperienza di misura del periodo
di un pendolo. Lo scopo dell'esperienza è imparare a trattare gli errori
casuali, dovuti alla procedura di misurazione del periodo con un cronometro,
che introduce inevitabilmente fluttuazioni casuali nelle misure effettuate,
e gli errori sistematici, dovuti alla diversa prontezza di riflessi degli
sperimentatori. In questo ultimo caso sorge il problema della 
compatibilità delle misure effettuate dai tre diversi membri del gruppo.

Ci siamo anche posti il problema di verificare sperimentalmente la
correttezza della legge del pendolo semplice, cosa che ogni studente di
fisica dovrebbe fare almeno una volta nella sua vita!

\subsubsection{Predizione teorica}

Per il calcolo teorico del periodo di un pendolo è stato utilizzato il modello del pendolo
semplice, che abbiamo considerato adeguato alla situazione. La massa $m_c$ del
cavo è stata considerata trascurabile poichè $m_c \ll m$ e quindi si è posto
$m_c = 0$. Inoltre, dato che in realtà il cilindro non è puntiforme, lo abbiamo approssimato ad un punto
materiale posizionato nel suo baricentro, che si trova ad una distanza
$\frac{h}{2}$ dal punto di sospensione del corpo.
Nel modello del pendolo semplice, la lunghezza del cavo è quindi
$l = l_c + \frac{h}{2} = (55.8 \pm 0.2 \, cm) + (2.5 \pm 0.025 \, cm) = 58.3 \pm 0.225 \, cm = 0.583 \pm 0.00225 \, m$.
Il valore di $g$, l'accelerazione di gravità, non è stato da noi misurato,
quindi abbiamo considerato $g = 9.8 \pm 0.02 \, \frac{m}{s^2}$ che è un
valore molto prudente. Con questi valori di $l$ e $g$ si ottene un periodo:

\begin{equation*}
    \mathcal{T} = 2\pi \sqrt{\frac{l}{g}} = 1.533 \pm 0.003 \, s 
\end{equation*}

Questo valore è quindi il valore atteso di periodo del pendolo.
Lo scopo dell'esperimento è quello di verificare sperimentalmente se questa predizione
è corretta.

\subsection{Descrizione della procedura di misura}

\subsubsection{Apparato sperimentale}

Il pendolo è stato realizzato con un filo da pesca, che si può considerare
inestensibile, fissato con un morsetto al supporto del tavolo da laboratorio.
All'altra estremità del cavo è stato appeso, attraverso un gancio, un corpo
cilindrico di metallo di massa $m$ = 205 g. Il pendolo si può schematizzare,
trascurando il gancio, come un cavo di lunghezza $l_c = 55.8 \pm 0.2$ 
cm unito ad un cilindro omogeneo di altezza $h = 5.1 \pm 0.05$ cm.
Le incertezze riportate nelle misure sono dovute a difficoltà incontrate nella
misurazione e alla risoluzione del metro a nastro che abbiamo usato.
Non abbiamo ritenuto di poter fare meglio di così.

\subsubsection{Processo di acquisizione dei dati sperimentali}

Per le misurazioni è stato utilizzato un comune cronometro con risoluzione di
misura $\Delta\mathcal{T}_{s}$ pari a 0.01 s, e con un errore di risoluzione massimo pari a 0.005 s. Ogni componente del gruppo ha
cronomerato 20 volte il tempo impiegato dal pendolo per compiere 10 periodi.
I valori grezzi misurati, riportati nella tabella \ref{tab:pendolo}, sono stati divisi per
10 per ottenere il valore medio del periodo del pendolo. Questo ci ha permesso
di diminuire gli errori sistematici dovuti alla prontezza di riflessi dei diversi
operatori, che verranno trattati più approfonditamente della sezione di analisi dei
dati, e di migliorare la risoluzione del cronometro fino a 1 millisecondo.

Successivamente un componente del gruppo ha eseguito 100 misurazioni di periodi
singoli dell'apparato, riportate nella tabella \ref{tab:pendolo100}. L'aquisizione di dati è stata
effettuata con la seguente procedura:

\begin{enumerate}
    \item{Il pendolo è stato fatto oscillare prestando attenzione al piano di oscillazione
        e all'ampiezza dell'oscillazione.}

    \item{Si è misurato un periodo, annotato il valore ottenuto e azzerato il cronometro.}

    \item{Le misurazioni sono state intervallate da 1-2 periodi non misurati.}

    \item{Ogni 10 misurazioni, che equivale a meno di 30 periodi, il pendolo è stato fermato
        e fatto ripartire, per evitare lo smorzamento dovuto all'attrito con l'aria.}
\end{enumerate}

Non è stata utilizzata la funzione ``giro'' del cronometro, al fine di evitare
di introdurre dipendenza tra le varie misure, azzerando il cronometro ogni 
volta le misure risultano statisticamente indipendenti.

\begin{table}[bt]
	\begin{tabular} {c c c c | c c c c | c c c c}
		\toprule
		\multicolumn{12}{c}{Periodo del pendolo [s]} \\
		\multicolumn{4}{c}{Francesco} & \multicolumn{4}{c}{Davide} & \multicolumn{4}{c}{Andrea} \\
		\midrule
		15.04 & 14.99 & 14.99 & 14.97 & 14.91 & 14.97 & 15.06 & 15.04 & 14.98 & 14.98 & 15.05 & 15.01 \\
		14.99 & 14.99 & 15.01 & 15.08 & 14.92 & 15.06 & 15.08 & 15.02 & 14.85 & 14.99 & 14.98 & 15.00 \\
		15.04 & 15.00 & 15.06 & 14.98 & 15.06 & 15.02 & 15.04 & 15.00 & 15.04 & 14.99 & 14.99 & 14.94 \\
		14.93 & 14.98 & 14.98 & 15.04 & 15.06 & 15.06 & 15.02 & 14.91 & 15.01 & 15.00 & 15.13 & 14.99 \\
		14.99 & 14.91 & 15.03 & 15.03 & 15.03 & 15.02 & 15.06 & 15.02 & 14.88 & 15.01 & 15.02 & 14.96 \\
		\bottomrule
	\end{tabular}

	\caption{Tabella delle misure del periodo del pendolo ottenute dai 3 membri del gruppo.
        Ogni sperimentatore ha raccolto 20 misure di 10 periodi. Sono riportati
        i dati grezzi di lettura del cronometro, riferiti a 10 periodi. }
    \label{tab:pendolo}
\end{table}

\begin{SCtable}[][tb]
	\centering
	\begin{tabular} {c c c c c | c c c c c}
		\toprule
		\multicolumn{10}{c}{Periodo del pendolo - Misure di un operatore [s]} \\
		\midrule
		1.46 & 1.54 & 1.55 & 1.57 & 1.41 & 1.45 & 1.52 & 1.45 & 1.37 & 1.53 \\
		1.57 & 1.50 & 1.52 & 1.50 & 1.48 & 1.49 & 1.44 & 1.49 & 1.43 & 1.53 \\
		1.50 & 1.50 & 1.56 & 1.61 & 1.45 & 1.38 & 1.52 & 1.41 & 1.60 & 1.49 \\
		1.48 & 1.53 & 1.52 & 1.55 & 1.54 & 1.46 & 1.51 & 1.51 & 1.49 & 1.52 \\
		1.52 & 1.50 & 1.48 & 1.46 & 1.41 & 1.48 & 1.45 & 1.48 & 1.52 & 1.51 \\
		\midrule
		1.55 & 1.52 & 1.55 & 1.49 & 1.51 & 1.50 & 1.52 & 1.49 & 1.54 & 1.52 \\
		1.45 & 1.49 & 1.47 & 1.47 & 1.48 & 1.48 & 1.53 & 1.51 & 1.52 & 1.47 \\
		1.54 & 1.42 & 1.43 & 1.45 & 1.49 & 1.42 & 1.47 & 1.36 & 1.50 & 1.55 \\
		1.58 & 1.52 & 1.45 & 1.48 & 1.44 & 1.52 & 1.51 & 1.50 & 1.54 & 1.52 \\
		1.52 & 1.52 & 1.47 & 1.52 & 1.44 & 1.56 & 1.50 & 1.49 & 1.52 & 1.56 \\
	\bottomrule
	\end{tabular}
	\caption{Misure di periodo effettuate da uno dei tre componenti del gruppo.
        Sono riportati i valori di lettura del cronometro riferiti a singole oscillazioni
        (periodi) dell'apparato.}
    \label{tab:pendolo100}
\end{SCtable}

\subsubsection{Errori sistematici}

Al fine di evitare errori sistematici riguardanti il modello teorico,
cioè errori causati dal fatto che il modello teorico non rispechia l'apparato
sperimentale usato, sono state prese le seguenti precauzioni:

\begin{itemize}
    \item{Si è tentato di contenere le oscillazioni del pendolo in un
        unico piano verticale. Abbiamo notato che a causa delle
        vibrazioni al momento del rilascio, il pendolo tende a seguire una
        traiettoria ellittica, invece che compiere la sua oscillazione su di un piano.
        Nei casi in cui la traiettoria si discostasse significativamente dal piano
        si è quindi proceduto a ripetere da capo la misura.}
    
    \item{Durante le operazioni di misura abbiamo mantenuto il massimo numero
        di oscillazioni entro il valore 20-30 affinché l'attrito con l'aria
        non influisse in modo significativo sul periodo del pendolo.}

    \item{Ci siamo accertati che l'ampiezza massima delle
        oscillazioni non superasse i 10$^\circ$ dalla verticale, in modo che
        l'approssimazione lineare $sin(\vartheta) \simeq \vartheta$ che viene
        utilizzata per calcolare la legge del pendolo sia accettabile. Infatti, per un
        angolo $\vartheta = 10^\circ = 0.175 \,\, rad$ si ha che:}

    \begin{equation}
        sin(\vartheta) = 0.174 \qquad \frac{\vartheta}{sin(\vartheta)} = 1.005
        \label{eq:theta_su_sintheta}
    \end{equation}

\end{itemize}

Possiamo provare a stimare l'errore sistematico commesso. Poichè nella (\ref{eq:theta_su_sintheta})
abbiamo ottenuto un errore del 5 per mille, l'errore sistematico dovuto a tutte le
cause elencate sopra è sicuramente maggiore di questo valore. Ci sentiamo pertanto abbastanza
sicuri stimando l'errore come quattro volte l'errore dovuto all'approssimazione $sin(\vartheta) \simeq \vartheta$ e otteniamo:

\begin{equation}
    \sigma_{sys} = 2 \cdot \frac{\vartheta}{sin(\vartheta)} = 1 \%
    \label{eq:errore_sistematico}
\end{equation}

Questo valore dovrebbe comprendere, verosimilmente, tutte le cause dell'incertezza sistematica descritte sopra.

Tra gli errori sistematici segnaliamo, inoltre, il fatto che le misure sono state
effettuate da sperimentatori diversi, che introducono degli errori sistematici
che variano da persona a persona. Questi errori saranno trattati nella sezione
di analisi dei dati, e non sono stati inclusi nella (\ref{eq:errore_sistematico}).

Le precauzioni prese quindi dovrebbero essere sufficienti a ridurre l'errore sistematico
entro limiti accettabili. Tuttavia è risaputo che gli errori sistematici
sono i più difficili da scovare ed eliminare.

\subsection{Analisi dei dati}

La Tabella \ref{tab:pendolo} e la Figura \ref{fig:pendolo} riguardano i 60 dati
relativi alla misura del periodo del pendolo da parte di tutti e tre i
componenti del gruppo.

\begin{figure}[p]
	\centering
	\includegraphics[width=120mm]{grafici/Pendolo.pdf}
	\caption{L'istogramma mostra le misure effettuate da tutti e tre gli sperimentatori.
        È indicato l'intervallo di incertezza tipo e la media campionaria. La larghezza
        dei bin dell'istogramma è 0.003 s. Come si può notare la forma dell'istogramma
        assomiglia ad una distribuzione gaussiana, come era prevedibile sapendo che gli
        errori dominanti delle misure sono casuali.}
    \label{fig:pendolo}
\end{figure}

\begin{figure}[p]
	\centering
	\includegraphics[width=120mm]{grafici/Pendolo100.pdf}
	\caption{Istogramma delle 100 misure di periodo del pendolo ottenute
        da un membro del gruppo. Sono evidenziate la media campionaria
        e l'intervallo di incertezza tipo. Come notato nella figura \ref{fig:pendolo}
        anche qui la distribuzione somiglia ad una distribuzione normale,
        a causa degli errori causali.}
    \label{fig:pendolo100}
\end{figure}

\begin{figure}[bt]
	\centering
	\includegraphics[width=150mm]{grafici/pendolo3.pdf}
	\caption{Il grafico illustra le misure di periodo del pendolo ottenute
        dai 3 diversi membri del gruppo, mettendo in risalto gli errori
        sistematici commessi. Da notare il fatto che Davide ha registrato più
        misure nel bin centrato in 1.503 e ha media campionaria più alta, mentre gli
        altri sperimentatori hanno picchi nel bin centrato in 1.500. In sottoimpressione
        è disegnato l'istogramma con i dati di tutti e 3 i componenti del gruppo, riportato
        anche in figura \ref{fig:pendolo}.}
    \label{fig:pendolo3}
\end{figure}

\subsubsection{Compatibiltà delle misure di diversi sperimentatori.}

La prima cosa che ci proponiamo di dimostrare è la compatibiltà
tra i dati misurati da ognuno di noi. Fissiamo a priori un fattore
di copertura $k$ = 2. Date due misure $x_a \pm \sigma_a$ e $x_b \pm \sigma_b$,
esse verranno considerate compatibili se soddisferanno la seguente condizione:

\begin{equation*}
    |m[x_a] - m[x_b]| =: |R| \leq k\sigma_R = k\sqrt{\sigma_a^2 + \sigma_b^2}
\end{equation*}

I dati raccolti ci permettono di esprimere tre diverse misure di periodo:

\begin{equation*}
	\begin{split}
		m_{Francesco}^* \pm \tilde{\sigma}_{Francesco}  = 1.500 \pm 0.004\,s \\
		m_{Davide}^* \pm \tilde{\sigma}_{Davide} = 1.502 \pm 0.005\,s \\
		m_{Andrea}^* \pm \tilde{\sigma}_{Andrea} = 1.499 \pm 0.006\,s
	\end{split}
\end{equation*}

Da queste possiamo ottenere la differenza tra le medie campionarie:

\begin{equation*}
	\begin{split}
		R_{1} = m_{Davide}^* - m_{Francesco}^* = 0.001\,s \\
		R_{2} = m_{Andrea}^* - m_{Francesco}^* = 0.002\,s \\
		R_{3} = m_{Andrea}^* - m_{Davide}^* = 0.001\,s
	\end{split}
\end{equation*}

L'errore standard campionario relativo a queste tre differenze è (con k = 2):

\begin{equation*}
	\begin{split}
		\sigma_{R_{1}} = k\,\sqrt{\tilde{\sigma}^2_{Davide} - \tilde{\sigma}^2_{Francesco}} = k\,0.003 = 0.006\\
		\sigma_{R_{2}} = k\,\sqrt{\tilde{\sigma}^2_{Andrea} - \tilde{\sigma}^2_{Francesco}} = k\,0.004 = 0.008\\
		\sigma_{R_{3}} = k\,\sqrt{\tilde{\sigma}^2_{Andrea} - \tilde{\sigma}^2_{Davide}} = k\,0.003 = 0.006
	\end{split}
\end{equation*}

Poichè risulta che:

\begin{equation*}
	R_{1}\leq{k\sigma_{R_{1}}}\,\,\,\,R_{2}\leq{k\sigma_{R_{2}}}\,\,\,\,R_{3}\leq{k\sigma_{R_{3}}}
\end{equation*}

possiamo concludere che le misure effettuate sono compatibili tra di loro.\\
 
Dal momento che le misure risultano essere compatibili tra di loro le misure
ripetute del periodo di un pendolo mostrano che le fluttuazioni delle misure
sono dovute principalmente ad errori casuali, poichè il periodo
$\mathcal{T}$ di oscillazione di un pendolo semplice è isocrono per ampiezze
di oscillazioni abbastanza piccole, nel nostro caso contenute entro i dieci gradi.

\subsubsection{Analisi dei dati dei tre sperimentatori}

Procedimo ora con l'analisi complessiva dei dati relativi alla misura del
periodo del pendolo eseguita da tutti e tre i componenti del gruppo e otteniamo che:

\begin{itemize}
    \item{La media campionaria é:}
        \begin{equation}
            m^*[\mathcal{T}] = \frac{1}{N} \sum_{i=1}^{N} \mathcal{T}_i = 1.500\,s
        \end{equation} 

    \item{La varianza campionaria del periodo di oscilazione del pendolo è:}
        \begin{equation}
            \tilde{D}[\mathcal{T}] = \frac{1}{N - 1} \sum_{i=1}^{N} (\mathcal{T}_i - m^*[\mathcal{T}])^2 = 0.00003\,s^2
        \end{equation}

    \item{La deviazione standard campionaria risulta essere:}
        \begin{equation}
            \tilde{\sigma}[\mathcal{T}] = \sqrt{\frac{1}{N - 1} \sum_{i=1}^{N} (\mathcal{T}_i - m^*[\mathcal{T}])^2} = 0.005 s
        \end{equation}

    \item{La mediana, il quantile 10\% e il quantile 90\% sono relativamente:}
        \begin{equation*}
            M = 1.501\,s \quad
            q_{10\%} = 1.493\,s \quad
            q_{90\%} = 1.506\,s
        \end{equation*}
\end{itemize}

Pertanto, applicando le necessarie approssimazioni e gli errori relativi ad ogni misura si ottiene

\begin{equation*}
	m^*[\mathcal{T}] = 1.500 \,s
\end{equation*}

\paragraph{Errore sistematico:}
Come si può notare dall'analisi fatta non abbiamo riscontrato particolari errori sistematici dovuti all'operatore, in quanto la media campionaria dei dati di ogni operatore è compatibile con la media di tutti i valori entro la deviazione standard campionaria.

Nella sezione 3.1.3 abbiamo analizzato gli altri errori sistematici e abbiamo stimato l'errore sistematico totale come $\sigma_{sys} = 1\,\%$ della media $m^*[\mathcal{T}]$. Quindi il valore numerico é:

\begin{equation*}
	\sigma_{sys} = 0.01\,\,m^*[\mathcal{T}] = 0.015\,s
\end{equation*}
Inoltre se le misure di uno sperimentatore fossero state affette da un errore sistematico i suoi valori medi sarebbero risultati significativamente differenti da quelli degli altri due componenti del gruppo.

\paragraph{Errore casuale o di tipo A:}
Il valore degli errori casuali è contenuto grazie al fatto che abbiamo deciso di misurare il periodo di 10 oscillazioni consecutive e quindi il loro valore è stato ridotto di un fattore dieci. Infatti abbiamo che:

\begin{equation*}
	\tilde{\sigma}[\mathcal{T}] = \sqrt{\frac{1}{N - 1} \sum_{i=1}^{N} (\mathcal{T}_i - m^*[\mathcal{T}])^2} = 0.005 s
\end{equation*}

\paragraph{Errore di risoluzione tipo:}
Al fine di stimare un errore comlessivo sulla misura del periodo del pendolo è doveroso prendere in considerazione anche l'errore di risoluzione standard dovuto allo strumento. Per questo motivo utilizzando le conoscenze acquisite in classe dire che:
\begin{equation*}
	\sigma_{ris} = \frac{1}{\sqrt{12}} \Delta\mathcal{T}_{m}\\
\end{equation*}
dove $\Delta\mathcal{T}_{m}$ rappresenta l'errore di risoluzione dovuto alla procedura di misura e quindi 	$\frac{\Delta\mathcal{T}_{s}}{10}$. Da questo otteniamo che

\begin{equation*}
	\sigma_{ris} = 0.0003\,s 
\end{equation*}

\paragraph{Unione dei vari errori:}
Quindi grazie hai risultati così ottenuti possiamo dire che l'errore complessivo sulla misura del periodo del pendolo risulta essre:

\begin{equation}
	\sigma_{tot} = \sqrt{\sigma_{sys}^2 + \tilde{\sigma}[\mathcal{T}]^2 + \sigma_{ris}^2} = 0.016\,s
\end{equation}
Grazie a questo risultato possiamo concludere riportando che il periodo del pendolo risulta essere:

\begin{equation}
	\mathcal{T} = \mathcal{T}_0 \pm \delta\mathcal{T} = 1.500\,s \pm 0.016\,s
\end{equation}\\
dove $\mathcal{T}_0$ rappresenta $m^*[\mathcal{T}]$ e $\delta\mathcal{T}$ rappresenta $\sigma_{tot}$.

\subsubsection{Analisi dei dati dello sperimentatore singolo}

La Tabella \ref{tab:pendolo100} e la Figura \ref{fig:pendolo100} invece sono relativi alla misura del periodo del pendolo da parte di un singolo membro del gruppo.
Procediamo ora con l'analisi dei dati relativi alle cento misurazioni del periodo:

\begin{itemize}
    \item{La media campionaria é:}
        \begin{equation}
            m^*[\mathcal{T}] = \frac{1}{N} \sum_{i=1}^{N} \mathcal{T}_i = 1.50\,s
        \end{equation} 

    \item{La varianza campionaria del periodo di oscilazione del pendolo è:}
        \begin{equation}
            \tilde{D}[\mathcal{T}] = \frac{1}{N - 1} \sum_{i=1}^{N} (\mathcal{T}_i - m^*[\mathcal{T}])^2 = 0.0022\,s^2
        \end{equation}

    \item{La deviazione standard campionaria risulta essere:}
        \begin{equation}
            \tilde{\sigma}[\mathcal{T}] = \sqrt{\frac{1}{N - 1} \sum_{i=1}^{N} (\mathcal{T}_i - m^*[\mathcal{T}])^2} = 0.05\,s
        \end{equation}

    \item{La mediana, il quantile 10\% e il quantile 90\% sono relativamente:}
        \begin{equation*}
            M = 1.50\,s \quad
            q_{10\%} = 1.44\,s \quad
            q_{90\%} = 1.55\,s
        \end{equation*}
\end{itemize}

Innanzitutto, si può notare che le cento misure di oscillazione di un singolo periodo hanno una precisione fino al centesimo di secondo, mentre quelle relative a tutti e tre i componenti avevano una precisione fino al millesimo. Infatti la deviazione standard di queste misure è di un ordine di grandezza maggiore rispetto al dato relativo alle misure precedenti.
Pertanto possiamo dire che:

\begin{itemize}
	\item{Errore sistematico: l'errore sistematico presente sulle cento misure è esattamente lo stesso che avevamo per le sessanta misure analizzate precedentemente. Quindi 
		\begin{equation}
			\sigma_{sys} = 0.01\,\,m^*[\mathcal{T}] = 0.015\,s
		\end{equation}	 }
	\item{Errore casuale o di tipo A: l'errore casuale grazie all'analisi appena fatta risulta essere
		\begin{equation}
			\tilde{\sigma}[\mathcal{T}] = 0.05\,s
		\end{equation}}
	\item{Errore di risoluzione tipo: l'errore di risoluzione tipo risulta essere:
		\begin{equation}
			 {\sigma}[\mathcal{T}]_{ris} = \frac{1}{\sqrt{12}} \,\, \Delta\mathcal{T}_{s} = 0.003\,s	
		\end{equation}}
\end{itemize}

Quindi anche in questo caso possiamo concludere dicendo che l'errore totale sul periodo di oscillazione del pendolo risulta essere:

\begin{equation}
	\sigma_{tot} = \sqrt{\sigma_{sys}^2 + \tilde{\sigma}[\mathcal{T}]^2 + \sigma_{ris}^2} = 0.05 \,s
\end{equation}
e pertanto otteniamo che:

\begin{equation}
	\mathcal{T} = \mathcal{T}_0 \pm \delta\mathcal{T} = 1.500\,s \pm 0.05\,s
\end{equation}
dove $\mathcal{T}_0$ rappresenta $m^*[\mathcal{T}]$ e $\delta\mathcal{T}$ rappresenta $\sigma_{tot}$.

\subsubsection{Distribuzione dei valori medi campionari}

Se procediamo ora con un'analisi più approfondita dei dati ottenuti dalle cento misure e li raggruppiamo in dieci gruppi di dieci misure ciascuno, possiamo calcolare la media campionaria di ogni singolo gruppo:

\begin{equation}
m_k^* \quad con \,\, K \in{\{1,2,...,10\}}
\end{equation}
dove k indica il numero del gruppo considerato.
Ottenute queste misure possiamo considerarle come singoli dati e farne una media campionaria ($ m[m_k^*] $):

\begin{equation}
m[m_k^*] = \sum_{k=1}^{10} (m_k^*) = 1.50\,s
\end{equation}
che risulta essere uguale alla media campionaria delle cento misure prese singolarmente. Calcolando la deviazione standard delle medie ($ \sigma[m_k^*] $):

\begin{equation}
\sigma[m_k^*] = \sqrt{\frac{10}{10-1} \sum_{k=1}^{10} (m_k^* - m[m_k^*])^2} = 0.020\,s
\end{equation}
otteniamo che il rapporto tra $\frac{\sigma[m_k^*]}{\tilde{\sigma}[\mathcal{T}]}$ risulta essere circa 0.42 che è un risultato attendibile dal momento che la predizione teorica ci dice che:

\begin{equation}
\frac{\sigma[m_k^*]}{\tilde{\sigma}[\mathcal{T}]} = \frac{1}{\sqrt{10}} \simeq \, 0.32
\end{equation}
