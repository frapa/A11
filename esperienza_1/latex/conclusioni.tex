\section{Conclusioni}

In questa esperienza abbiamo potuto osservare due differenti approcci analitici per due
differenti esperimenti. Infatti nel primo esperimento abbiamo voluto studiare la
distribuzione di lunghezza di un certo numero di cilindretti metallici: oggetti
diversi a cui corrispondono misure effettivamente diverse. Nel secondo esperimento,
invece, abbiamo ripetuto misure diverse e indipendenti dello stesso fenomeno.

I risultati del primo esperimento evidenziano, nel caso del gruppo singolo, un'evidente
distribuzione incentrata su un singolo valore, predicendo l'esistenza di una sola famiglia
di cilindretti. Nel caso dei dati di tutti i gruppi, invece, viene evidenziata anche
l'esistenza di una seconda famiglia di cilindretti. È inoltre sottolineato il fatto che
se lo strumento è inadatto al tipo di esperimento (come il metro a nastro),
è difficile cavare un ragno dal buco e l'esperimento diventa inutile.

Nel secondo esperimento, i risultati ottenuti misurando 10 oscillazioni del pendolo
evidenziano che il metodo è particolarmente buono in quanto permette di ridurre gli errori
casuali di un ordine di grandezza inferiore alla risoluzione dello strumento.
I risultati ottenuti dalle misure di oscillazioni singole, invece, hanno lo stesso ordine
di grandezza della risoluzione dello strumento e sono più affette dagli errori casuali.

Per quanto riguarda la compatibilità tra misure e predizione teorica possimo dire che,
nel caso delle 100 misure effettuate da uno solo dei tre sperimentatori, c'è un sostanziale
accordo entro i margini d'errore delle misure con il modello teorico. Nel caso delle misure
effettuate da tutti e tre i membri del gruppo, invece, la predizione teorica non è in accordo
con la misura sperimentale. Abbiamo interpretato questo fatto come un effetto della procedura
di misurazione. Essendo le misure effettuate dai tre sperimentatori dieci volte meno soggette
ad errori casuali e di risoluzione delle altre 100 misure, l'incertezza che emerge è quella
sistematica dovuta all'inadeguatezza del modello teorico e dagli errori pratici che
emergono durante le misurazioni. Da questo punto di vista ci si poteva aspettare una
situazione del genere.

Tuttavia vorremmo segnalare che c'è un altra possibile spiegazione alla discrepanza.
La nostra misura della lunghezza del filo è stata fatta in modo non rigoroso, poiché
l'abbiamo un po' trascurata. Un errore di trascrizione o di misura potrebbe essere la causa
di questo problema. Promettiamo di stare più attenti la prossima volta!
