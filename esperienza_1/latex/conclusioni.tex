\section{Conclusioni}
In questa esperienza abbiamo potuto osservare due differenti approcci analitici per due
differenti esperimenti. Infatti nel primo esperimento abbiamo voluto studiare la
distribuzione di lunghezza di un certo numero di cilindretti metallici: oggetti
diversi a cui corrispondono misure effettivamente diverse. Nel secondo esperimento, invece, abbiamo ripetuto misure diverse e indipendenti dello stesso fenomeno.

I risultati del primo esperimento evidenziano, nel caso del gruppo singolo, un'evidente
distribuzione incentrata su un singolo valore, predicendo l'esistenza di una sola famiglia
di cilindretti. Nel caso dei dati di tutti i gruppi, invece, viene evidenziata anche
l'esistenza di una seconda famiglia di cilindretti.

Nel secondo esperimento, i risultati ottenuti misurando 10 oscillazioni del pendolo
evidenziano che il metodo è particolarmente buono in quanto permette di ridurre gli errori
casuali e sistmatici di un ordine di grandezza inferiore alla risoluzione dello strumento.
I risultati ottenuti dalle misure di oscillazioni singole, invece, hanno lo stesso ordine
di grandezza della risoluzione dello strumento e sono più affette dagli errori casuali.


\newpage

\begin{equation}
\tilde{D} = \frac{1}{N - 1} \sum_{i=1}^{N} (x_i - m^*[x])^2
\end{equation}

\begin{equation}
\tilde{\sigma} = \sqrt{\frac{1}{N - 1} \sum_{i=1}^{N} (x_i - m^*[x])^2}
\end{equation}

\begin{equation}
m^*[x] = \frac{1}{N} \sum_{i=1}^{N} (x_i) \simeq \sum_{j=1}^{\mathcal{N}} (x_j p_j^*) 
\end{equation}

\begin{equation}
p_j^* = \frac{n_j^*}{N} 
\end{equation}

\begin{equation}
f_j^* = \frac{n_j^*}{N\Delta x}
\end{equation}

\begin{equation}
D^* = \langle(x - m^*)^2\rangle = \frac{1}{N} \sum_{i=1}^{N} (x_i - m^*[x])^2
\simeq \sum_{j=1}^{\mathcal{N}} (x_j - m^*)^2 p_j^*)
\end{equation}

\begin{equation}
m^*[\mathcal{T}] = \frac{1}{N} \sum_{i=1}^{N} \mathcal{T}_i
\end{equation}

\begin{equation}
\sigma^*[\mathcal{T}] = \sqrt{\frac{1}{N} \sum_{i=1}^{N} (\mathcal{T}_i - m^*)^2}
\end{equation}

\begin{equation}
\mathcal{T} = \mathcal{T}_0 \pm \delta\mathcal{T}
\end{equation}

\begin{equation}
\mathcal{T}_0 = m^* = \frac{1}{N} \sum_{i=1}^{N} \mathcal{T}_i
\end{equation}

\begin{equation}
\delta\mathcal{T}_{cas} = \tilde{\sigma}[m^*] = \frac{1}{\sqrt{N}} \sqrt{\frac{N}{N - 1}}\sigma^*[\mathcal{T}] = \sqrt{\frac{1}{N(N - 1)} \sum_{i=1}^{N} (\mathcal{T}_i - m^*)^2}
\end{equation}

