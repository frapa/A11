\begin{titlepage}
\begin{center}
	\hrule \vspace{0.5cm}
     	\textsc{\LARGE Misure ripetute di lunghezza e tempo}
	\vspace{0.5cm} \hrule \vspace{2cm}
      	{\large Francesco Pasa, Davide Bazzanella, Andrea Miani\\
		Gruppo A11}\\
	\vspace{0.5cm}
      	{\large 28 febbraio 2013 - 11 marzo 2013}
	\vfill
	{\begin{abstract}
Misura della lunghezza di un gruppo di 25 cilindri di metallo e della durata del periodo di oscillazione di un pendolo semplice.
Analisi dei valori ottenuti dagli esperimenti del singolo gruppo e dei valori raccolti dagli esperimenti di tutti i gruppi di laboratorio.
	 \end{abstract}}
\end{center}
\end{titlepage}

\newpage

\vspace*{\fill}
\begin{center}
	\tableofcontents
\end{center}
\vspace*{\fill}

\newpage

\section{Introduzione}
Questa relazione sarà divisa in due sezioni, una dedicata al primo esperimento
in cui verrà misurata la lunghezza di una popolazione di 25 cilindri metallici
con tre diversi strumenti (metro a nastro, calibro ventesimale, micrometro).
L'altra dedicata alla misura del periodo di un pendolo semplice (costruito con
un cavo inestensibile e un peso) effettuata da tutti i componenti del gruppo
al fine di evidenziare la presenza di eventuali errori sistematici, dovuti
alla differente prontezza di riflessi dei componenti dell'equipe. Entrambe
le sezioni saranno integrate con l'analisi dei dati ottenuti dagli
esperimenti, in particolare la prima conterrà anche lo studio dei dati
ottenuti da tutti gli altri gruppi del corso di laboratorio.

