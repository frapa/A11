Vogliamo ora calcolare la media pesata delle medie del periodo ($\mathcal{T}_i$) relative a ciasuna massa, dove l'indice $i$ fa riferimento a quale massa viene presa in considerazione, pertanto procediamo come segue:
\begin{itemize}
	\item{calcoliamo la media pesata dei periodi:
			\begin{equation*}
				\mathcal{T} \,=\, \frac{\sum_i \mathcal{T}_i w_i}{\sum w_i} \quad \text{dove} \quad w_i \,=\, \frac{1}{(\delta \mathcal{T}_i)^2}
			\end{equation*}
			%
			}
	\item{calcoliamo l'errore sul periodo nel seguente modo:
			\begin{equation*}
				\delta \mathcal{T} \,=\, \frac{1}{\sqrt{\sum w_i}}
			\end{equation*}
			%
			}
\end{itemize}
pertanto otteniamo una misura del periodo del pendolo di lunghezza fissata $\ell \pm \delta \ell \,=\, (1.0525 \pm 0.0006) \, m$ che ha il seguente valore (castrare una cifra da l e dire che non ha senso scendere sotto errore risoluzione strumento):

\begin{equation*}
	\mathcal{T} \pm \delta \mathcal{T} \,=\, (2.056 \pm 0.002) \, s
\end{equation*}
%
