nooooooooo prima medie periodi poi media pesata medie periodi.\\\\\\
Come prima operazione calcoliamo le medie pesate delle misure del periodo ($\mathcal{T}$). Ricordiamo che le misure del singolo periodo ($t_i$) devono essere divise per un fattore 5, poichè sono state cronometrate cinque oscillazioni. Per questo motivo l'errore che affligge ogni singola misura di periodo sarà il seguente:

\begin{equation*}
	\delta t_i \,=\, \frac{1}{5} \, \frac{\Delta t}{\sqrt{12}}
\end{equation*}
%
dove $\frac{\Delta t}{\sqrt{12}}$ rappresenta l'intervalo di incertezza tipo sulla misura del periodo singolo.\\
Pertanto per calolare la media pesata procediamo come segue:
\begin{itemize}
	\item{calcoliamo la media pesata dei periodi:
			\begin{equation*}
				\mathcal{T} \,=\, \frac{\sum_i t_i w_i}{\sum w_i} \quad \text{dove} \quad w_i \,=\, \frac{1}{(\delta t_i)^2}
			\end{equation*}
			%
			}
	\item{calcoliamo l'errore sul periodo nel seguente modo:
			\begin{equation*}
				\delta \mathcal{T} \,=\, \frac{1}{\sqrt{\sum w_i}}
			\end{equation*}
			%
			}
\end{itemize}
pertanto otteniamo una misura del periodo del pendolo di lunghezza fissata $\ell \pm \delta \ell \,=\, (1.0525 \pm 0.0006) \, m$ che ha il seguente valore (castrare una cifra da l e dire che non ha senso scendere sotto errore risoluzione strumento):

\begin{equation*}
	\mathcal{T} \pm \delta \mathcal{T} \,=\, (2.056 \pm 0.002) \, s
\end{equation*}
%







