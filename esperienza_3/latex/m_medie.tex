Vogliamo ora calcolare la media pesata delle medie del periodo ($\mathcal{T}_i$) relative a ciasuna massa, dove l'indice $i$ fa riferimento a quale massa viene presa in considerazione, ovvero $i \in \{1,2,3,4\}$.
Infatti, se i periodi misurati non dipendono dalla massa, essi dovrebbero essere compatibili con una costante.
Poiché le incertezze sulle medie dei periodi sono differenti, la miglior stima di questa costante è la media pesata delle medie.
Verificheremo successivamente che le medie dei periodi siano relamente conpatibili con un andamento costante.
Pertanto procediamo come segue:

\begin{itemize}
	\item{calcoliamo la media pesata dei periodi:
			\begin{equation*}
				\mathcal{T} \,=\, \frac{\sum_i \mathcal{T}_i w_i}{\sum w_i} \quad \text{dove} \quad w_i \,=\, \frac{1}{(\delta \mathcal{T}_i)^2}
			\end{equation*}
			%
			}
	\item{calcoliamo l'errore sulla media pesata nel seguente modo:
			\begin{equation*}
				\delta \mathcal{T} \,=\, \frac{1}{\sqrt{\sum w_i}}
			\end{equation*}
			%
			}
\end{itemize}
pertanto otteniamo una misura del periodo del pendolo di lunghezza fissata $\ell \pm \delta \ell \,=\, (\,1.053 \pm 0.0006\,)$ m che ha il seguente valore:

CHECK THIS MAN! ci sono cifre significative mancanti o di troppo 1.053 (3 cifre) 0.0006 (4 cifre) !!! PS: ho letto la frase dopo, ma mi puzza comunque un sacco sta roba - Dave

\begin{equation*}
	\mathcal{T} \pm \delta \mathcal{T} \,=\, (\,2.056 \pm 0.002\,) \, \text{s}
\end{equation*}
%
Ricordiamo che la misura della lunghezza non presenta tante cifre significative quante ne ha la sua l'incertezza poichè abbiamo deciso che non avrebbe senso fornire un valore più preciso di $\ell$ rispetto alla risoluzione dello strumento.
