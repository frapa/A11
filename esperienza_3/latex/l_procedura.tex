Abbiamo scelto dieci diverse lunghezze del filo, riportate nella prima colonna della tabella \ref{tab:lunghezze_periodi}.
Le misure riportate in tabella si riferiscono alla distanza tra l'attaccatura del filo al cilindro e la parte superiore
del morsetto che reggeva il cavo, e verranno indicate con $\bar{l}_i$, dove $i \in \{1, \dots, 10\}$.
Per ottenere la distanza del cilindro dal punto di sospensione è quindi necessario sottrarre a tali misure
la lunghezza $l\ped{mors} = 28.0 \pm 0.3$ mm del morsetto (per la discussione sulle incertezze si veda il paragrafo
\ref{l_medie}). Abbiamo preso le misure in questo modo per problemi tecnici nella misurazione diretta.

WARNING! LoL! 'problemi tecnici', Prodi direbbe che è vago... $\bigotimes$

Per ogni lunghezza $\bar{l}_i$ del cavo sono stati registrati $N = 10$ valori di periodi. Ogni valore è stato ottenuto
misurando 5 periodi e dividendo per 5 la misura. Si è prestato attenzione a mantenere le oscillazioni del pendolo
all'interno di un piano; si è inoltre fatto ripartire il pendolo ad ogni misura, al fine di ridurre l'effetto dell'attrito
dell'aria. I periodi registrati, opportunamente divisi per 5, sono riportati in tabella \ref{tab:lunghezze_periodi}.

Le lunghezze scelte sono separate da intervalli regolari di \SI{10}{\centi\metre} ciascuno, a parte la prima
misura per cui abbiamo usato i dati ricavati al punto \emph{\ref{dipendenza_massa}}$.$ \emph{Dipendenza del periodo del pendolo dalla massa}, per questioni di tempo.
Poiché nella sezione precedente sono stati presi 15 dati per massa e in questo esperimento ne abbiamo preso solo 10
per ogni lunghezza, abbiamo scartato gli ultimi 5 valori in Tabella \ref{tab:masse_periodi}.

\begin{table}
    \centering
    \begin{tabular}{c c c c c c c c c c c}
        \multicolumn{11}{c}{\textbf{Lunghezze e periodi}} \\
        \toprule
        Lunghezza [m] & \multicolumn{10}{c}{Periodi [s]} \\
        \midrule
        1.057 & 2.086 & 2.068 & 2.052 & 2.040 & 2.054 & 2.062 & 2.098 & 2.048 & 2.058 & 2.048 \\
        0.953 & 1.932 & 1.942 & 1.948 & 1.928 & 1.954 & 1.946 & 1.954 & 1.958 & 1.944 & 1.968 \\
        0.853 & 1.844 & 1.860 & 1.864 & 1.864 & 1.862 & 1.860 & 1.854 & 1.854 & 1.858 & 1.840 \\
        0.753 & 1.732 & 1.726 & 1.720 & 1.742 & 1.754 & 1.742 & 1.740 & 1.744 & 1.748 & 1.746 \\
        0.653 & 1.604 & 1.626 & 1.614 & 1.610 & 1.598 & 1.620 & 1.600 & 1.614 & 1.614 & 1.596 \\
        0.553 & 1.448 & 1.494 & 1.498 & 1.472 & 1.454 & 1.472 & 1.480 & 1.488 & 1.488 & 1.472 \\
        0.453 & 1.348 & 1.336 & 1.326 & 1.348 & 1.330 & 1.336 & 1.332 & 1.340 & 1.346 & 1.308 \\
        0.353 & 1.182 & 1.196 & 1.190 & 1.188 & 1.198 & 1.176 & 1.182 & 1.160 & 1.204 & 1.194 \\
        0.253 & 1.002 & 1.018 & 0.996 & 1.012 & 1.018 & 1.024 & 0.998 & 0.998 & 0.994 & 1.022 \\
        0.153 & 0.762 & 0.774 & 0.790 & 0.794 & 0.790 & 0.768 & 0.770 & 0.768 & 0.794 & 0.772 \\
        \bottomrule
    \end{tabular}
    \caption{Lunghezze del filo scelte per testare la dipendenza del periodo
        del pendolo dalla lunghezza. Le dieci misure sono equispaziate ad intervalli di \SI{10}{\centi\metre}.
        Le lunghezze riportate si riferiscono alla distanza tra l'attaccatura del filo al cilindro e la parte superiore
        del morsetto che reggeva il cavo. Per ottenere la distanza del cilindro dal punto di sospensione è quindi
        necessario sottrarre la lunghezza del morsetto.}
    \label{tab:lunghezze_periodi}
\end{table}
