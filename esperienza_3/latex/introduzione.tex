\section{Introduzione}


In questo esperimento studieremo in modo quantitativo quali relazioni sussistano tra il periodo di oscillazione di un pendolo ($\mathcal{T}$), la massa applicatavi ($m$) e la lunghezza del filo stesso ($l$).\\
Queste relazioni, nel caso di piccole oscillazioni, furono studiate e formalizzate per la prima volta da Galileo Galilei (Pisa, 15 febbraio 1564 – Arcetri, 8 gennaio 1642) e si possono riassumere come segue:

\begin{equation}
    \label{eq:periodo_pendolo}
	\mathcal{T} \,=\, 2\pi\sqrt{\frac{l}{g}}
\end{equation}
%
dove $g$ rappresente l'accelerazione di gravità (locale).\\
Quindi in questo esperimento ci proponiamo di verificare che il periodo di oscillazione di un pendolo,
sfruttando il modello di pendolo semplice, sia legato alla lunghezza del filo da una proporzionalità diretta,
e che non abbia dipendenza dalla massa.
Inoltre coi dati che andremo a raccogliere abbiamo intenzione di ricavare un valore sperimentale dell'accelerazione di gravità.
