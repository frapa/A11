Finora abbiamo assunta veritiera l'ipotesi che il periodo di oscillazione del pendolo non dipende linearmente dalla massa applicata. Ciononostante ora vogliamo controllare se effettivamente sia così e quindi abbiamo deciso difare una regressione lineare su $f$ in modo da dare una stima ai parametri A e B e verificare che A risulti compatibile con il periodo $\mathcal{T}$ trovato grazie alla media pesata e che il valore di B sia compatibie con lo 0 teorico.

Procediamo operativamente in questo modo per fare la regressione lieare:

\begin{itemize}
	\item{La funzione da minimizzare che misura la discrepanza è:
			\begin{equation}
                \sum_{i=1}^{N} \frac{(\mathcal{T} - A - B m)}{(\delta \mathcal{T}_{tot})^2}	
                \label{eq:min_quad}
			\end{equation}
			%
            dove $\delta \mathcal{T}_{tot}$ è l'incertezza totale sulle misure del periodo, ottenuta sommando l'incertezza $\delta \mathcal{T}_i$ e l'errore trasferito dal peso, che risulta essere nullo in quanto come stima preliminare del parametro B abbiamo usato lo 0 poichè anche in precedenza abbiamo assunto che il periodo non dipende dalla massa.}
\end{itemize}
\begin{itemize}
	\item{Quindi per quanto studiato in classe abbiamo che:

			\begin{equation*}
				A \,=\, \frac{(\sum_i w_i x_i^2)(\sum_i w_i y_i) - (\sum_i w_i x_i)(\sum_i w_i x_i y_i)}{\Delta} \,=\, 0.034 \,\, s^2
			\end{equation*}
			%
			\begin{equation*}
				B \,=\, \frac{(\sum_i w_i)(\sum_i w_i x_i y_i) - (\sum_i w_i y_i)(\sum_i w_i x_i)}{\Delta} \,=\, 4.06 \,\, s^2 / kg
			\end{equation*}
			%
			dove:
			\begin{equation*}
				\Delta \,=\, (\sum_i w_i)(\sum_i w_i x_i^2) - (\sum_i w_i x_i)^2 \,\,\,\,\,\,\, e \,\,\,\,\,\,\,
				w_i \,=\, \frac{1}{(\delta y_i)^2}
			\end{equation*}}
	\item{Di conseguenza abbiamo che le incertezze relative su A e B sono:

			\begin{equation*}
				(\delta A)^2 \,=\, \frac{\sum_i w_i x_i^2}{\Delta}  \,\,\,\,\, e \,\,\,\,\,
				(\delta B)^2 \,=\, \frac{\sum_i w_i}{\Delta} 
			\end{equation*}}
	\end{itemize} 
	Quindi possiamo riassumere i risultati di questa procedura in questo modo:

	\begin{equation*}
		A \,\pm\, \delta A \,=\, (0.034 \,\, \pm \,\, 0.001) \,\,s^2 \,\,\,\,\, e \,\,\,\,\,
		B \,\pm\, \delta B \,=\, (4.06 \,\, \pm \,\, 0.01) \,\,s^2 \,/\, kg
	\end{equation*}

non c'entra un'azzo ma ... mi serve come traccia!!!!!!!!!!!!!!!!!!!!!!!!!!!!!!!!!!!!!!!!!!!!!!!!!!!!!!!!!!!!!!!!!!!!!!!!!!!!!!