Per ottenere un valore unico di $g \pm \delta g$ procediamo seguendo due differenti metodologie:

\subsubsection{Primo metodo: Media pesata}

Il primo metodo che adotteremo per trovare il valore dell'accelerazione di gravità è il calcolo della media pesata, utilizzando le seguenti formule:

\begin{equation*}
g_w \,\, = \,\, \frac{\sum g_iw_i}{\sum w_i} \quad\quad \text{e} \quad\quad \delta g_w \,\, = \,\, \frac{1}{\sqrt{\sum w_i}} \quad\quad \text{dove} \quad\quad w_i \,\, = \,\, \frac{1}{(\delta g_i)^2}
\end{equation*}
%
$g_w$ e $\delta g_w$ rappresentano relativamente il risultato della media pesata e il suo errore relativo. Inoltre $w_i$ è il peso relativo ad ogni misura $g_i$.
Eseguendo questa procedura sui dati si ricava il seguente valore per l'accelerazione di gravità:

\begin{equation}
g_w \pm \delta g_w \,\, = \,\, 9.80 \pm 0.03 \,\, \si{\metre\per\square\second}
\end{equation}

\paragraph{Test del chi quadro\\}

Essendo la procedura della media pesata equivalente ad una regressione, possiamo effettuare il test del chi quadro per verificare
la correttezza del risultato ottenuto. Ripetendo la procedura già vista nelle sezioni precedenti calcoliamo

\begin{equation}
    \chi^2_{\text{oss}} = \sum_{i=1}^N \frac{(g_i - g_w)^2}{\delta g_i^2} = 8.0
\end{equation}
%
dove $N = 10$ è il numero di valori $g_i$ calcolati e $\delta g_i$ è l'incertezza su tali valori. Non occorre trasferire
l'errore poiché si tratta di una retta costante (e quindi parallela all'asse delle ascisse), pertanto il suo coefficiente ancgolare
è nullo.

\subsubsection{Secondo metodo: distribuzione dei valori}

Il secondo metodo che utilizziamo è quello di considerare la distribuzione dei valori $g_i$, cioè calcolarne il valore medio $m^*[g_i]$ e lo scarto quadratico medio della distribuzione dei valori medi $\sigma[m^*[g_i]]$.

\begin{equation*}
g_s \, = \, m^*[g_i] \, = \, \frac{\sum g_i}{N} \quad\quad \text{e} \quad\quad \delta g_s \, = \, \sigma[g_i] \,\, = \,\, \frac{1}{\sqrt{N}}\sqrt{\frac{\sum (g_i - m^*[g_i])^2}{N-1}}
\end{equation*}
%
da cui si ricava il valore:

\begin{equation}
g_s \pm \delta g_s \,\, = \,\, 9.80 \pm 0.03 \,\, \si{\metre\per\square\second}
\end{equation}

\paragraph{Osservazioni sulla metodologia e sulla compatibilità dei valori ricavati\\}

Comparando le due metodologie si può innanzitutto notare che il valore ottenuto con la media pesata risulta avere un errore minore del valore ottenuto con lo studio della distribuzione dei valori. Ciò è dovuto al fatto che la media pesata tiene in considerazione maggiormente i valori con un errore minore, ottenendo così una misura più precisa. Nonostante ciò, l'errore determinato con lo studio della distribuzione dei valori è poco più grande di quello determinato dalla media pesata. Questa eventualità è probabilmente dovuta al fatto che non vi è una marcata differenza tra i valori con $\delta g$ maggiore (ricavati con $\ell$ minore) e valori con $\delta g$ minore (ricavati con $\ell$ maggiore).

Per quanto riguarda la compatibilità dei due valori dell'accelerazione di gravità trovati noi sappiamo che $g_w$ e $g_s$ coincidono, compresi i loro errori di incertezza, pertanto possiamo affermare che le due grandezze sono compatibili.

% si potrebbe affermare senza dubbio che le due misure sono compatibili, tutta via a scanso di equivici verifichiamo la loro compatibilità nel seguente modo:
%\begin{itemize}
%	\item{fissiamo a priori un fattore di copertura k = 3, e si può osservare che:
%	
%		\begin{equation*}
%			R \,\, = \,\, |g_w - g_s| \,\, = \,\, \SI{0}{\metre\per\square\second}
%		\end{equation*}
%		%
%		\begin{equation*}
%			k\,\sigma_R \,\, = \,\, \sqrt{(\delta g_w)^2 + (\delta g_s)^2} \,\, = \,\, \SI{.04}{\metre\per\square\second}
%		\end{equation*}
%		%
%		}
%	\item{quindi abbiamo verificato che $R \leq k\,\sigma_R$ e pertanto sappiamo che le due misure dell'accelerazione di gravità risltano essere compatibili}
%\end{itemize}
