Per ottenere un valore unico di $g \pm \delta g$ procediamo seguendo due differenti metodologie:

\paragraph{Primo metodo: Media pesata\\}

Calcolo la media pesata con le seguenti formule:
\begin{equation}
	g \,\, = \,\, \frac{\sum g_iw_i}{\sum w_i} \quad\quad e \quad\quad \delta g \,\, = \,\, \frac{1}{\sqrt{\sum w_i}} \quad\quad dove \quad\quad w_i \,\, = \,\, \frac{1}{(\delta g_i)^2}
\end{equation}
%
ricavando il seguente valore per l'accelerazione di gravità:
\begin{equation}
	g \pm \delta g \,\, = \,\, 9,800 \pm 0,015 \,\, ms^{-2}
\end{equation}

\paragraph{Secondo metodo: distribuzione dei valori\\}

Si calcoli il valore medio $\langle g\rangle$ e il suo scarto quadratico medio $\sigma[\langle g \rangle]$:
\begin{equation}
	\langle g \rangle \,\, = \,\, \frac{\sum g_i}{N} \quad\quad 	e	\quad\quad	\sigma[\langle g \rangle] \,\, = \,\, \sqrt{\frac{\sum (g_i - \langle g \rangle)^2}{N}}
\end{equation}

da cui si ricava il valore:
\begin{equation}
	g \pm \delta g \,\, = \,\, 9,8 \pm 0,1 \,\, ms^{-2}
\end{equation}


\begin{center}
	---OSSERVAZIONI---
	
	osservazioni sui due metodi di calcolo di g
\end{center}