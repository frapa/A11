Per ottenere un valore unico di $g \pm \delta g$ procediamo seguendo due differenti metodologie:

\paragraph{Primo metodo: Media pesata\\}

Calcolo la media pesata con le seguenti formule:
\begin{equation}
	g \,\, = \,\, \frac{\sum g_iw_i}{\sum w_i} \quad\quad e \quad\quad \delta g \,\, = \,\, \frac{1}{\sqrt{\sum w_i}} \quad\quad dove \quad\quad w_i \,\, = \,\, \frac{1}{(\delta g_i)^2}
\end{equation}

\paragraph{Secondo metodo: distribuzione dei valori\\}

Si calcoli il valore medio $\langle g\rangle$:

%
Si stimi lo scarto quadratico medio della distribuzione dei valori di $\sigma[\langle g\rangle]$

Quindi si ricava il valore:
\begin{equation}
	g \pm \delta g \,\, = \,\, \ast\ast,\ast \pm \ast,\ast \,\, ms^{-2}
\end{equation}


\begin{center}
	---OSSERVAZIONI---
	
	osservazioni sui due metodi di calcolo di g
\end{center}