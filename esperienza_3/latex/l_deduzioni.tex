Dai dati sperimentali, abbiamo ricavato la legge

\begin{equation}
    \mathcal{T} = a\ell^b
    \tag{\ref{eq:ipotesi}}
\end{equation}
%
dove $a = 2.006 \pm 0.004 \; \text{s}\,\text{m}^{-\frac{1}{2}}$ e $b = 0.499 \pm 0.004$.

Ricordiamo che nel paragrafo \ref{l_pred_teo} abbiamo ricavato i seguenti valori teorici per $a$ e $b$

\begin{equation}
    a = \frac{2\pi}{\sqrt{g}} \; \frac{\text{s}}{\sqrt{\text{m}}} = 2.006 \; \frac{\text{s}}{\sqrt{\text{m}}} \qquad \qquad b = \frac{1}{2} = 0.5
\end{equation}

Possiamo quindi discutere la compatibilità tra valori ottenuti teoricamente e calcolati sperimentalmente.
In entrambi i casi il risultato dell'esperimento è positivo, infatti i valori teorici sono compatibili
con quelli sperimentali entro l'incertezza.

Abbiamo quindi verificato la correttezza del modello del pendolo semplice e della sua predizione del periodo
del pendolo. Abbiamo inoltre verificato che la dipendenza del periodo del pendolo dalla lunghezza del filo
va come $\sqrt{\ell}$. I dati sperimentali sono in perfetto accordo con la teoria, anche oltre le nostre aspettative,
ci riteniamo soddisfatti dell'esito dell'esperimento, almeno per quanto riguarda questa sezione.
