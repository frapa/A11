Nella prima parte dell'esperimento abbiamo verificato, tramite la misurazione del periodo di oscillazione di quattro diverse masse, che non vi fosse una dipendenza del periodo del pendolo $\mathcal{T}$ dalla massa appesa.

Nella seconda parte è stata analizzata invece la dipendenza del periodo di oscillazione dalla lunghezza del filo: è stata verificata tramite la misurazione del periodo di oscillazione per dieci diverse lunghezze del filo una dipendenza del tipo descritto dall'equazione (\ref{eq:periodo_pendolo}). Tramite la verifica dei valori sperimentali e dei valori teorici, nonostante la probabile presenza di un errore sistematico nei dati raccolti, abbiamo potuto verificare la bontà del nostro esperimento.

Nell'ultima parte, infine, tramite i dati raccolti nella seconda parte è stato possibile calcolare un valore per l'accelerazione di gravità g. Tale valore è stato poi verificato essere abbastanza accurato se confrontato con i valori tabulati e il modello teorico.