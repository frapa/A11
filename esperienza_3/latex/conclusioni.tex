Per concludere possiamo dire che nella prima parte dell'esperimento abbiamo verificato, tramite la misurazione del periodo di oscillazione di quattro diverse masse, che non vi fosse una dipendenza del periodo del pendolo $\mathcal{T}$ dalla massa appesa. Questo è un risultato atteso, e quasi scontato, poiché, tenendo conto delle semplificazioni al modello apportate, in nessuna teoria esistente è presente una dipendenza del periodo dalla massa. Ciononostante se si osserva più criticamente l'esperimento eseguito si potrebbe ipotizzare che, per una lunghezza del filo accettabile, se la massa applicata fosse stata molto maggiore di quelle utilizzate, il periodo di oscillazione non sarebbe risultato così vicino alla predizione teorica. Diciamo questo perchè in questa condizione si sarebbe potuto verificare un allungamento nel filo, non stimato dal nostro modello, che sarebbe andato a modificare la lunghezza iniziale di quest'ultimo, e quindi avrebbe variato il valore atteso del periodo di oscillazione. Ma anche in questo caso l'elemento che porta una variazone al periodo non è la massa, ma è una variazione della lunghezza del filo, che nella realtà non è inestensibile, dovuta ad un carico eccessivo applicato ad esso.

Nella seconda parte della relazione quindi si è deciso di analizzare la dipendenza del periodo di oscillazione del pendolo dalla lunghezza del filo. E' stata infatti verificata, tramite la misurazione del periodo di oscillazione per dieci diverse lunghezze del filo, una dipendenza descritta dall'equazione (\ref{eq:periodo_pendolo}). Pertanto, grazie alla verifica dei valori sperimentali e dei valori teorici, nonostante la probabile presenza di un errore sistematico nei dati raccolti, abbiamo potuto verificare la bontà del nostro esperimento.

Nell'ultima parte, infine, tramite i dati raccolti e la modifica delle incertezze sulle misure apportata nella seconda parte della relazione, è stato possibile calcolare un valore per l'accelerazione di gravità $g$. Tale valore è stato poi verificato essere abbastanza accurato se confrontato con i valori tabulati e il modello teorico. E' importante sottolineare che il valore dell'accelerazione di gravità non è uniforme lungo tutta la superficie terrestre, ma varia in base alla latitudine e all'altitidine del luogo degli esperimenti. Per questo motivo se si dovessero svolgere esperimenti che richiedono una grande precisione è doveroso assumere un valore particolarmente preciso anche di questi parametri per ricavare un valore più vicino al reale valore di $g$.
