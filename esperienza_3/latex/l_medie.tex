\label{l_medie}

Innanzitutto vogliamo specificare le incertezze a cui sono soggetti i nostri dati.

Le lunghezze $\bar{l}_i$ del filo e la lunghezza del morsetto $l\ped{mors}$ sono state misurate con un metro a nastro di risoluzione
$\Delta l$ = 0.001 m. L'errore standard di risoluzione di queste misure vale quindi:

\begin{equation}
    \sigma(l) = \frac{\Delta l}{\sqrt{12}} = \SI{0.0003}{\metre}
\end{equation}

L'altezza $h$ del cilindro è stata misurata con un calibro ventesimale di risoluzione $\Delta l\ped{calibro} = 0.00005$ m
per cui l'errore tipo di risoluzione è:

\begin{equation}
    \sigma\ped{calibro}(l) = \frac{\Delta l\ped{calibro}}{\sqrt{12}} = \SI{0.00001}{\metre}
\end{equation}

Poichè abbiamo utilizzato il modello del pendolo semplice, siamo interessati a conoscere la distanza tra il punto
di sospensione ed il baricentro del cilindro. Il cilindro, la cui densità è stata considerata omogenea,
viene così approssimato ad un punto materiale posto nel suo centro. Vedremo in seguito come migliorare questa approssimazione.
Per ottenere la distanza tra punto di sospensione del filo e baricentro del cilindro si è usata la formula

\begin{equation}
	\l_i \,=\, \bar{l}_i \,-\, l\ped{mors} \,+\, \frac{h}{2}
    \label{eq:l_i}
\end{equation}

che è stata applicata ad ogni lunghezza del filo. I risultati dei calcoli sono riportati nella prima colonna
della tabella \ref{tab:l_dati}. L'incertezza sulla lunghezza trovata con la (\ref{eq:l_i}) si può ricavare mediante la propagazione
degli errori:

\begin{equation}
	\delta l_i = \sqrt{4\,\sigma(l)^2 + \left\frac{\sigma\ped{calibro}(l)}{2}\right ^2}
\end{equation}

\begin{SCtable}
    \centering
    \begin{tabular}{c c}
        \multicolumn{2}{c}{\textbf{Dati}} \\
        \toprule
        $l_i$ [m] & Periodo [s] \\
        \midrule
        1.0526 &  \\
        0.9486 &  \\
        0.8486 &  \\
        0.7486 &  \\
        0.6486 &  \\
        0.5486 &  \\
        0.4486 &  \\
        0.3486 &  \\
        0.2486 &  \\
        0.1486 &  \\
        \bottomrule
    \end{tabular}
    \caption{}
    \label{tab:l_dati}
\end{SCtable}
