\label{l_medie}

Innanzitutto vogliamo specificare le incertezze a cui sono soggetti i nostri dati.

Le lunghezze $\bar{l}_i$ del filo e la lunghezza del morsetto $l\ped{mors}$ sono state misurate con un metro a nastro di risoluzione
$\Delta l$ = 0.001 m. L'errore standard di risoluzione di queste misure vale quindi:

\begin{equation}
    \sigma(l) = \frac{\Delta l}{\sqrt{12}} = \SI{0.0003}{\metre}
\end{equation}

L'altezza $h$ del cilindro è stata misurata con un calibro ventesimale di risoluzione $\Delta l\ped{calibro} = \SI{0.00005}{\meter}$
per cui l'errore tipo di risoluzione è:

\begin{equation}
    \sigma\ped{calibro}(l) = \frac{\Delta l\ped{calibro}}{\sqrt{12}} = \SI{0.00001}{\metre}
\end{equation}

Poichè abbiamo utilizzato il modello del pendolo semplice, siamo interessati a conoscere la distanza tra il punto
di sospensione ed il baricentro del cilindro. Il cilindro, la cui densità è stata considerata omogenea,
viene così approssimato ad un punto materiale posto nel suo centro. Vedremo in seguito come migliorare questa approssimazione.
Per ottenere la distanza tra punto di sospensione del filo e baricentro del cilindro si è usata la formula

\begin{equation}
	\l_i \,=\, \bar{l}_i \,-\, l\ped{mors} \,+\, \frac{h}{2}
    \label{eq:l_i}
\end{equation}
%
che è stata applicata ad ogni lunghezza del filo. I risultati dei calcoli sono riportati nella prima colonna
della tabella \ref{tab:l_dati}. L'incertezza sulla lunghezza trovata con l'equazione (\ref{eq:l_i}) si può ricavare mediante la propagazione
degli errori:

\begin{equation}
	\delta l = \sqrt{4\,\sigma(l)^2 + \frac{\sigma\ped{calibro}(l)^2}{4}} = \SI{6e-4}{m}
\end{equation}
%
che è stata indicata con $\delta l$ poiché è uguale per tutti i valori $l_i$.

\begin{SCtable}[1.4][b]
    \centering
    \begin{tabular}{c c c c}
        \multicolumn{4}{c}{\textbf{Dati}} \\
        \toprule
        $l_i$ [m] & $\mathcal{T}_i$ [s] & $\sigma(\mathcal{T}_i)$ [s] & $\delta T_i$ [s] \\
        \midrule
        1.053 & 2.061 & 0.006 & 0.006 \\
        0.949 & 1.947 & 0.004 & 0.004 \\
        0.849 & 1.856 & 0.003 & 0.003 \\
        0.749 & 1.739 & 0.003 & 0.003 \\
        0.649 & 1.610 & 0.003 & 0.003 \\
        0.549 & 1.477 & 0.005 & 0.005 \\
        0.449 & 1.335 & 0.004 & 0.004 \\
        0.349 & 1.187 & 0.004 & 0.004 \\
        0.249 & 1.008 & 0.004 & 0.004 \\
        0.149 & 0.778 & 0.004 & 0.004 \\
        \bottomrule
    \end{tabular}
    \caption{Sono elencate le lunghezze $l_i$ tra punto di sospensione e baricentro,
        la media $\mathcal{T}_i$ dei periodi misurati, la deviazione tipo $\sigma(\mathcal{T}_i)$
        sulla media (ricavata con metodo statistico) e l'incertezza totale $\delta T_i$
        sulle medie dei periodi. L'incertezza totale è stata ottenuta sommando alla deviazione tipo sulla media
        l'errore tipo di risoluzione, che, come si vede, non è significativo.
        Nella prima colonna, nonostante l'errore tipo valga $\delta l = \SI{6e-4}{\metre}$, si sono riportate
        soltanto le prime 3 cifre dopo la virgola in quanto non porterebbe informazione andare oltre la risoluzione
        dello strumento di misura. Per maggiori informazioni sulle
        incertezze si veda il paragrafo \ref{l_medie}.}
    \label{tab:l_dati}
\end{SCtable}

Passando alle incertezze sui periodi, la risoluzione di misura dei periodi, che dipende dal cronometro, è
$\Delta \mathcal{T} = 0.01$ s. Avendo misurato 5 periodi per ogni dato si ha che l'errore tipo di risoluzione vale:

\begin{equation}
	\sigma\ped{ris}(\mathcal{T}) = \frac{\Delta \mathcal{T}}{5\sqrt{12}} = \SI{6e-4}{s}
\end{equation}

Avendo misurato un insieme di $N = 10$ periodi per ogni lunghezza $l_i$ del cavo, possiamo trattare statisticamente i dati
al fine di ottenere un unico valore di periodo per ogni lunghezza, ma anche per ottenere una stima degli errori casuali.
Abbiamo quindi calcolato la media $\mathcal{T}_i$ di ogni insieme di periodi, ottenendo i dati riportati in tabella \ref{tab:l_dati}.

Si sono poi calcolate le deviazioni standard $\sigma(\mathcal{T}_i)$ sulle medie $\mathcal{T}_i$, utilizzando la consueta formula:

\begin{equation}
	\sigma(\mathcal{T}_i) = \sqrt{\frac{1}{N(N - 1)}\sum_{j=1}^N (\mathcal{T}_{ij} - \mathcal{T}_i)^2}
\end{equation}
%
dove $N = 10$ indica il numero di dati per in ciascun insieme, mentre $\mathcal{T}_{ij}$ indica il j-esimo dato dell'insieme di misure
relativo alla lunghezza $l_i$.

Alle incertezze $\sigma(\mathcal{T}_i)$ ricavate è stato sommato il contributo derivante dall'errore di risoluzione, calcolato poco prima,
mediante la seguente formula

\begin{equation}
	\delta\mathcal{T}_i = \sqrt{\sigma(\mathcal{T}_i)^2 + \sigma\ped{ris}(\mathcal{T})^2}
\end{equation}

In questo modo abbiamo ottenuto l'incertezza totale $\delta\mathcal{T}_i$ per ogni media $\mathcal{T}_i$, che tiene conto
dell'errore di risoluzione e degli errori casuali; possiamo quindi esprimere 
un unica misura di periodo per ogni lunghezza del filo, nel seguente modo

\begin{equation}
	\mathcal{T}_i \pm \delta\mathcal{T}_i
\end{equation}

I valori $\sigma(\mathcal{T}_i)$ e $\delta\mathcal{T}_i$ per tutte le lunghezze sono riportati in Tabella \ref{tab:l_dati}.
