Verifichiamo la bontà della regressione eseguita nel paragrafo precedente tramite il test del chi quadro.
Il valor atteso $\chi^2_{\text{teo}}$ del test è uguale al numero di gradi di libertà del sistema. Nel nostro caso
esso vale $\chi^2_{\text{teo}} = N - 2 = 8$ poiché dal fit sono stati calcolati 2 parametri. Poiché
il numero di gradi di libertà è basso, abbiamo deciso di adottare un intervallo di confidenza del tipo $[0, h]$,
dove $h$ è il valore critico del chi quadro. Scelta la probabilità di falso allarme del 5 \%, dalla distribuzione
del chi quadro si può calcolare $h = 15.5$. Quindi l'intervallo di confidenza è [0, 15.5]. 

Utilizzando i valori $A$ e $b$ trovati precedentemente, calcoliamo l'espressione

\begin{equation}
    \chi^2_{\text{oss}} = \sum_{i=1}^N \frac{(Y_i - A - bX_i)^2}{(\delta Y_i^{\text{tot}})^2} = 29.4
\end{equation}
%
che è chiaramente al di fuori dell'intervallo di confidenza.

Probabilmente questa incompatibilità è dovuta al fatto che l'incertezza sui periodi è stata sottostimata.

[giustificazione del perché]

Aggiustiamo quindi le incertezze per far tornare il chi quadro. Poiché le incertezze sono diverse per ciascun punto,
moltiplichiamo ogni incertezza $\delta Y_i^{\text{tot}}$ per un fattore $q$ e determiniamo per quale valore di $q$
il chi quadro diventa uguale al suo valore atteso

\begin{equation}
    \chi^2 = \sum_{i=1}^N \frac{(Y_i - A - bX_i)^2}{(q\, \delta Y_i^{\text{tot}})^2} = N - 2 = 8
\end{equation}
%
Poiché $q$ è uguale per tutti i valori di $\delta Y_i^{\text{tot}}$, si può portare fuori dal segno di sommatoria.
Successivamente, mediante passaggi algebrici elementari, si arriva al seguente risultato

\begin{equation}
    q^2 = \frac{1}{8} \sum_{i=1}^N \frac{(Y_i - A - bX_i)^2}{(\delta Y_i^{\text{tot}})^2} = 3.68 \qquad \text{ovvero} \qquad q = 1.92
\end{equation}

Abbiamo quindi calcolato che, affinchè il chi quadro diventi uguale al suo valor atteso, è necessario moltiplicare
le incertezze per un fattore $q \simeq 1.9$. Le incertezze vanno quindi aumentate del 90 \%.

\subsubsection{Conclusione della regressione}

Poiché sono state modificate le incertezze, prima di calcolare i valori definitivi di $a$ e $b$ e concludere l'esperimento,
è necessario ripetere la procedura di fit per calcolare le nuove incertezze $\delta A$ e $\delta b$

[intermezzo pubblicitario]

Risaliamo ora al valore di a:

\begin{equation}
    a = 10^A = 2.006 \qquad \qquad \delta a = \ln(10) \cdot 10^A \cdot \delta A = 0.004
\end{equation}
%
dove $\ln$ è il logaritmo naturale.
