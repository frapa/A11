\begin{center}
--- INSERIRE TABELLA DATI ---

colonne:

$\ell_i$, $\delta\ell$, $\mathcal{T}_i$, $\delta\mathcal{T}$, $g_i$, $\delta g_i$
\end{center}

dai dati $\ell_i$ e $\mathcal{T}_i$ abbiamo calcolato l'incertezza $\delta g_i$ utilizzando la formula generale per la propagazione degli errori:
%formula generale per la propagazione degli errori
\begin{equation}
	(\delta g)^2 \,\, \simeq \,\, \left( \frac{\partial g}{\partial \ell} \right)^2 (\delta \ell)^2 \,\, + \,\, \left( \frac{\partial g}{\partial \mathcal{T}} \right)^2 (\delta \mathcal{T})^2 \,\, \simeq \,\, \left( \frac{2 \pi}{\mathcal{T}} \right)^4 (\delta \ell)^2 \,\, + \,\, \left( \frac{8 \pi^2 \ell}{\mathcal{T}^3} \right)^2 (\delta \mathcal{T})^2
\end{equation}
\begin{equation*}
	\delta g \,\, \simeq \,\, \sqrt{\left( \frac{2 \pi}{\mathcal{T}} \right)^4 (\delta \ell)^2 \,\, + \,\, \left( \frac{8 \pi^2 \ell}{\mathcal{T}^3} \right)^2 (\delta \mathcal{T})^2}
\end{equation*}

Osservando i valori, si può notare che le incertezze $\delta\ell$ sono più influenti delle incertezze $\delta\mathcal{T}$ nel determinare $\delta g_i$

\begin{center}
--- INSERIRE GRAFICO---

$g_i$ in funzione di $\ell_i$ 
\end{center}

Sono più affidabili i valori di g ricavati da $\ell$ grandi o $\ell$ piccoli?

Basandosi sulle approssimazioni di pendolo semplice (punto materiale, filo inestensibile e privo di massa) e di piccole oscillazioni si può osservare che i valori più attendibili di g saranno quelli realizzati con $\ell$ grandi in quanto più vicini alle approssimazioni effettuate.