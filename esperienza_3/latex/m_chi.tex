Dopo aver calcolato la media pesata delle misure $\mathcal{T}_i$ sfruttiamo il test del chi quadro per verificare se sono compatibili o meno. Per questo motivo defiano la seguente funzione $f$

\begin{equation}
	f \,:=\, A + B m 
\end{equation}
%
dove B rappresenta il coefficiente angolare della retta, che assumiamo nullo, $m$ la massa applicata al pendolo e A una costante che nel nostro caso particolare è il valore della media pesata dei periodi ($\mathcal{T}$).\\
Pertanto ora procediamo nell'eseguire il test del chi quadro per accertarci se il risultato da noi trovato sia accettabile o meno. Quindi:
\begin{itemize}
	\item{calcoliamo il chi quadro osservato ($\chi_{oss}^2$):
			\begin{equation*}
				\chi_{oss}^2 \,=\, \frac{\sum_{i=0}^{N} \mathcal{T}_i - A - B m)^2}{(\delta \mathcal{T}_{tot})^2} \,=\, 1.6607
			\end{equation*}
			%
			è importante sottolineare che il $\delta \mathcal{T}_{tot}$ non è altro che $\delta \mathcal{T}_i$ in quanto $B = 0$ per ipotesi
			}
\end{itemize}