Poichè vogliamo verificare che il periodo non dipende dalla massa applicata prendiamo la funzione $f$ così definita:

\begin{equation*}
	f \,=\, A + B m 
\end{equation*}
%
dove B rappresenta il coefficiente angolare della retta, che assumiamo nullo, $m$ la massa applicata al pendolo e A una costante che nel nostro caso particolare è il valore della media pesata dei periodi ($\mathcal{T}$).\\
Pertanto ora procediamo nel fare il test del chi quadro per accertarci se il risultato da noi trovato è accettabile o meno. Quindi:
\begin{itemize}
	\item{calcoliamo il chi quadro osservato ($\chi_{oss}^2$):
			\begin{equation*}
				\chi_{oss}^2 \,=\, \frac{\sum_{i=0}^{N} (t_i - A - B m)^2}{(\delta t_{tot})^2} \,=\, 1.6607
			\end{equation*}
			%
			è importante sottolineare che il $\delta t_{tot}$ non è altro che $\delta t_i$ in quanto $B = 0$ per ipotesi
			}
\end{itemize}