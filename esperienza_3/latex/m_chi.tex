Dopo aver calcolato la media pesata delle misure $\mathcal{T}_i$ sfruttiamo il test del chi quadro per verificare se sono compatibili o meno.

%Per questo motivo defiano la seguente funzione $f$

%\begin{equation}
%	f \,:=\, A + B m 
%\end{equation}
%
%dove B rappresenta il coefficiente angolare della retta, che assumiamo nullo, $m$ la massa applicata al pendolo e A una costante che nel nostro caso particolare è il valore della media pesata dei periodi ($\mathcal{T}$).\\
%Pertanto ora procediamo nell'eseguire il test del chi quadro per accertarci se il risultato da noi trovato sia accettabile o meno.
Calcoliamo quindi il chi quadro osservato ($\chi_{oss}^2$):

\begin{equation*}
	\chi_{oss}^2 \,=\, \frac{\sum_{i=0}^{N} (\mathcal{T}_i - \mathcal{T})^2}{(\delta \mathcal{T}_i)^2} \,=\, 1.6607
\end{equation*}
%
%è importante sottolineare che il $\delta \mathcal{T}$ non è altro che $\delta \mathcal{T}_i$ in quanto $B = 0$ per ipotesi.
dove $\mathcal{T}_i$ rappresentano le quattro misure medie del periodo relative alle quattro massa considerate e $\mathcal{T}$ rappresenta il valore della media pesata trovato precedentemente.\\

Sapendo che i gradi di libertà del nostro sistema sono 3 e che vogliamo una probabilità del $5\%$ che i dati da noi raccolti non siano in accordo con le ipotesi fatte finora, cioè avendo posto la probabilità di falso allarme al $5\%$, otteniamo che l'intervallo di accettazione del $\chi_{oss}^2$ va da 0 a 7.82. Pertanto dal momento che in nostro $\chi_{oss}^2$ ha valore di 1.66 e quindi ricade all'interno dell'intervallo di accettazione possiamo dire che le misure $\mathcal{T}_i$ del periodo risultano essere compatibili. Ovvero noi sappiamo che mediamente una volta su venti può accadere che il dato sperimentale raccolto risulti essere in contraddizione con le ipotesi fatte, in particolare con quella che nega una dipendenza del periodo dalla massa, visto che abbiamo posto $B = 0$.
Abbiamo ricavato i valori degli estremi di accettazione del chi quadro grazie ad un programma presente sul sito internet "Stat Trek"\footnote{http://http://stattrek.com/online-calculator/chi-square.aspx}.