Al punto $\bigotimes$ la relazione tra periodo del pendolo e lunghezza era stata espressa nella formula
\begin{equation*}
\mathcal{T} \,\, = \,\, a\ell^b
\end{equation*}
che confrontata con l'equazione (\ref{eq:periodo_pendolo}) permette di ricavare $a \, = \, \frac{2 \pi}{\sqrt{g}}$, da cui si ottiene:
\begin{equation}
g \,\, = \,\, \left( \frac{2 \pi}{a}\right)^2 \quad\quad e \quad\quad \delta g \,\, = \,\, \frac{\vert dg \vert}{\vert da\vert} \delta a = \frac{8 \pi^2}{a^3} \delta a
\end{equation}

Quindi si ricava il valore:
\begin{equation}
g_f \pm \delta g_f \,\, = \,\, 9,81 \pm 0,04 \,\, ms^{-2}
\end{equation}

\paragraph{Osservazioni sulla compatibilità dei valori ricavati dal grafico e dalla tabella}

Fare osservazioni sulla compatibilità tra $g_w$, $g_s$ e $g_f$