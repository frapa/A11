Al punto \ref{l_pred_teo} la relazione tra periodo del pendolo e lunghezza era stata espressa nella formula

\begin{equation*}
\mathcal{T} \,\, = \,\, a\ell^b
\end{equation*}
%
che confrontata con l'equazione (\ref{eq:periodo_pendolo}) permette di ricavare $a \, = \, \frac{2 \pi}{\sqrt{g}}$, da cui si ottiene:

\begin{equation*}
g \,\, = \,\, \left( \frac{2 \pi}{a}\right)^2 \quad\quad e \quad\quad \delta g \,\, = \,\, \left| \frac{dg}{da} \right|  \delta a = \frac{8 \pi^2}{a^3} \delta a
\end{equation*}
%
Quindi si ricava il valore:

\begin{equation}
g_f \pm \delta g_f \,\, = \,\, 9.81 \pm 0.04 \,\, ms^{-2}
\end{equation}

\paragraph{Osservazioni sulla compatibilità dei valori ricavati dal grafico e dalla tabella\\} 
Anche in questo caso possiamo verificare che $g_f$ risulti compatibile con le altre due misure di accelerazione di gravità trovate nei due paragrafi precedenti. Ovvero ci basta verificare che $g_f$ sia compatibile con una sola delle due misusre di $g$ precedenti. Quindi procediamo col test della compatibilità:
\begin{itemize}
	\item{fissato a priori un fattore di copertura k = 3, si può osservare che:
	
		\begin{equation*}
			R \,\, = \,\, |g_f - g_w| \,\, = \,\, \SI{0.01}{\metre\per\square\second}
		\end{equation*}
		%
		\begin{equation*}
			k\,\sigma_R \,\, = \,\, \sqrt{(\delta g_e)^2 + (\delta g_w)^2} \,\, = \,\, \SI{0.15}{\metre\per\square\second}
		\end{equation*}
		%
		}
	\item{quindi abbiamo verificato che $R \leq k\,\sigma_R$ e pertanto sappiamo che le due misure dell'accelerazione di gravità risultano essere compatibili e quindi $g_f$, $g_w$ e $g_s$ risultano compatibili tra di loro.}
\end{itemize}