Al punto $\bigotimes$ la relazione tra periodo del pendolo e lunghezza era stata espressa nella formula
\begin{equation*}
	\mathcal{T} \,\, = \,\, ab^{\ell}
\end{equation*}
che confrontata con l'equazione $\bigotimes$ \ref{equazione del pendolo} permette di ricavare $a \,\, = \,\, 2 \pi / \sqrt{g}$, da cui si ottiene:
\begin{equation}
	g \,\, = \,\, \left( \frac{2 \pi}{a}\right)^2 \quad\quad e \quad\quad \delta g \,\, = \,\, \frac{\vert dg \vert}{\vert da\vert} \delta a = \frac{8 \pi^2}{a^3} \delta a
\end{equation}

Quindi si ricava il valore:
\begin{equation}
	g \pm \delta g \,\, = \,\, 9,80 \pm 0,04 \,\, ms^{-2}
\end{equation}