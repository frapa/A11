%discutere se i valori sono compatibili o meno
%se sono incompatibili usare la formula di propagazione delle incertezze per collegare l'errore sistematico di g agli errori sistematici di l e T. Ponendo prima l e poi T uguali a 0, discutere la plausibilità dell'altro errore sistematico
%(dl=0 --> dT=?; dT=0 --> dl=?)

La tabella seguente riporta le misure dell'accelerazione di gravità
per varie latitudini al livello del mare. Tali dati hanno un incertezza molto bassa,
che influisce sulla 6 cifra dopo la virgola, quindi in questa sede li considereremo valori esatti,
poiché i nostri dati hanno un incertezza molto maggiore.

\begin{center}
    \begin{tabular}{c c}
        \toprule
        Latitudine [$^\circ$] & $g$ [m/s$^2$] \\
        \midrule
        40 & 9.80171 \\
        50 & 9.81071 \\
        \bottomrule
    \end{tabular}
\end{center}

Poiché il laboratorio si trova circa ad una latitudine di 46$^\circ$, occorre interpolare linearmente i due valori
riportati. Abbiamo eseguito l'interpolazione con la formula:

\begin{equation}
    g_{46} = g_{40} + \frac{(g_{50} - g_{40})}{50 - 40} \cdot 6 \; \si{\meter\per\square\second} =
    \frac{4\, g_{40} + 6\, g_{50}}{10} \; \si{\meter\per\square\second} =
    9.8071 \; \si{\meter\per\square\second}
    \label{eq:g46}
\end{equation}
%
dove $g_{40}$ e $g_{50}$ indicano i valori di $g$ riportati in tabella, rispettivamente per le latitudini 40$^\circ$ e 50$^\circ$.

I dati riportati si riferiscono al valore di $g$ misurato al livello del mare. Poiché l'esperimento è stato effettuato circa all'altitudine di
$\Delta R = 400$ m sul livello del mare, occorre correggere il valore $g_{46}$. Dalla legge di gravitazione universale si può ricavare g

\begin{equation}
    g = \frac{GM}{R^2}
\end{equation}
%
differenziando e dividendo per g si ottiene

\begin{equation}
    \frac{dg}{g} = - \frac{2\,dR}{R^2}
\end{equation}
%
poiché $\Delta R \ll R$ si può stimare

\begin{equation}
    dg = - \frac{2\,g}{R^2} dR = \SI{-1e-3}{\meter\per\square\second}
    \qquad \qquad
    g_e = g_{46} + dg = \SI{9.806}{\meter\per\square\second}
    \label{eq:g_f}
\end{equation}
%
dove $g_e$ indica il valore dell'accelerazione di gravità definitivo per il laboratorio, proveniente dai dati tabulati,
e si è assunto $R = \SI{6.3709e6}{\metre}$, ovvero il valor medio del raggio della terra.

Questo è il valore che andremo a confrontare con i valori ricavati dall'esperimento e che assumiamo come valore vero
dell'accelerazione di gravità. Notare che i valori tabulati riportati hanno 5 cifre dopo la virgola, mentre il valore (\ref{eq:g46})
ne ha 4 poiché si tratta di un approssimazione abbastanza grezza. Il valore (\ref{eq:g_f}) riduce ulteriormente le cifre decimali a 3,
poiché il valore $\delta g$ che è stato sommato ha una precisione attorno a questo ordine di grandezza. In ogni caso
le cifre decimali extra non hanno rilevanza ai fini della discussione, poiché i valori di g da noi calcolati hanno solo 2
cifre decimali. 

Dall'esperimento abbiamo ricavato 3 valori di $g$:

\begin{itemize}
    \item{$g_w = 9.80 \pm 0.03 \; \si{\metre\per\square\second}$ calcolato con la media pesata.}
    \item{$g_s = 9.80 \pm 0.03 \; \si{\metre\per\square\second}$ calcolato con metodi statistici.}
    \item{$g_f = 9.81 \pm 0.04 \; \si{\metre\per\square\second}$ calcolato a partire dai parametri ricavati dalla regressione
        del paragrafo \ref{l_regressione}.}
\end{itemize}

La compatibilità dei valori da noi calcolati con il valore di riferimento (\ref{eq:g_f}) è ovvia in quanto esso cade all'interno
degli intervalli di errore tipo di ciascuna delle tre stime di $g$.
