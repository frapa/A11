%discutere se i valori sono compatibili o meno
%se sono incompatibili usare la formula di propagazione delle incertezze per collegare l'errore sistematico di g agli errori sistematici di l e T. Ponendo prima l e poi T uguali a 0, discutere la plausibilità dell'altro errore sistematico
%(dl=0 --> dT=?; dT=0 --> dl=?)

La tabella seguente riporta le misure dell'accelerazione di gravità
per varie latitudini al livello del mare. Tali dati hanno un incertezza molto bassa,
che influisce sulla 6 cifra dopo la virgola, quindi in questa sede li considereremo valori esatti,
poiché i nostri dati hanno un incertezza molto maggiore.

\begin{center}
    \begin{tabular}{l c}
        \toprule
        Latitudine [$^\circ$] & $g_e$ [m/s$^2$] \\
        \midrule
        40 & 9.80171 \\
        50 & 9.81071 \\
        \bottomrule
    \end{tabular}
\end{center}

Poiché il laboratorio si trova circa ad una latitudine di 46$^\circ$, occorre interpolare linearmente i due valori
riportati. Abbiamo eseguito l'interpolazione con la formula:

\begin
