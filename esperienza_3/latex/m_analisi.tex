\subsubsection{Considerazioni iniziali}
In questa prima parte della relazione vogliamo verificare che il periodo di oscillazione del pendolo, a parità di lunghezza del filo, non dipende dalla massa applicatavi.
Per fare questo abbiamo cronometrato il periodo di ocillazione del pendolo per tutte e quattro le masse a nostra disposizone.\\
Poichè abbiamo utilizzato l'approssimazione di pendolo semplice, nel quale la massa si considera totalmente applicata in un unico punto, è importante sottolineare che, quando è stata presa la misura del periodo relativo alla massa più piccola, abbiamo dovuto regolare la lunghezza del filo in modo che il baricentro della massa risultasse essere alla stessa distanza dal morsetto, a cui è applicato il filo, rispetto alle masse precedenti.\\
Ricordiamo inoltre che per ottenere la lunghezza complessiva del filo si è seguita la seguente procedura: la lunghezza complessiva del filo è data da:

\begin{equation}
	\l_{tot} \,=\, l_{filo} + \frac{h}{2} - l_{morsetto}
\end{equation}
%
dove $l _{filo}$ è la lunghezza del filo misurata dall'estremo superiore del morsetto fino al punto di applicazione sulla massa, $l_{morsetto}$ è l'altezza del morsetto e $\frac{h}{2}$ è l'altezza che rappresenta il baricentro della massa.\\
Pertanto l'errore sulla misura della lunghezza del filo sarà dato dalla propagazione dell'errore per la somma di misure indipendenti, ovvero:

\begin{equation*}
	\delta l_{tot} \,=\, \sqrt{(\delta l_{filo})^2 + (\delta \frac{h}{2})^2 + (\delta l_{morsetto})^2}
\end{equation*}
%
ricordando che le inceretzze relative alle varie misure sono le incertezze tipo, ovvero:

\begin{equation*}
	\text{inceretzza tipo su l} \,=\, \frac{\Delta l}{\sqrt{12}}
\end{equation*}
%
Mentre le masse sono affette da un incertezza tipo pertanto ogni massa avrà la seguente incertezza:

\begin{equation*}
	\delta m \,=\, \frac{\Delta m}{\sqrt{12}} \,=\, 0.03 \,\, g
\end{equation*}

\subsubsection{Dati}
Le masse che abbiamo applicato al filo da pesca sono quelle esposte nella tabella \ref{masse} mentre nella tabella \ref{tab:masse_periodi} sono elencati tutti i periodi che abbiamo rilevato per ciascuna massa.\\

\begin{table}
    \centering
    \begin{tabular}{c c c c c c c}
        \multicolumn{7}{c}{\textbf{Masse e periodi}} \\
        \toprule
        Massa [\si{\gram}] & \multicolumn{5}{c}{Periodi [s]} & Media dei periodi [s]\\
        \midrule
	\multirow{3}{*}{88.9}	& 10.09	& 10.38	& 19.10	& 10.29	& 10.24 & \multirow{3}{*}{2.053}\\
				& 10.36	& 10.34	& 10.45	& 10.24	& 10.20 &\\
				& 10.31	& 10.38	& 10.11	& 10.23	& 10.27 &\\
				\midrule
	\multirow{3}{*}{162.7}	& 10.26	& 10.29	& 10.40	& 10.31	& 10.35 & \multirow{3}{*}{2.053}\\
				& 10.21	& 10.38	& 10.24	& 10.20	& 10.18	\\
				& 10.20	& 10.16	& 10.23	& 10.35	& 10.24 \\
				\midrule
	\multirow{3}{*}{202.0}	& 10.20	& 10.29	& 10.16	& 10.31	& 10.34 & \multirow{3}{*}{2.055}\\
				& 10.34	& 10.28	& 10.16	& 10.36	& 10.46 \\
				& 10.24	& 10.31	& 10.28	& 10.25	& 10.17 \\
				\midrule
	\multirow{3}{*}{486.7}	& 10.43	& 10.34	& 10.26	& 10.20	& 10.27 & \multirow{3}{*}{2.060}\\
				& 10.31	& 10.49	& 10.24	& 10.29	& 10.24 \\
				& 10.31	& 10.41	& 10.22	& 10.27	& 10.24 \\
        \bottomrule
    \end{tabular}
    \caption{CONTROLLARE PERIODI SE METTERGLI GIA' DIVISI PER 5 O MENO E POI MEDIE... + COMMENTO TABELLA!!!}
    \label{tab:masse_periodi}
\end{table}

Dalle tabelle si può notare che il valore delle masse no presenta tante cifre siglificative come l'incertezza tipo che lo affligge, questo è dovuto al fatto che non avrebbe senso aggiungerne in quanto andreamo oltre la risoluzione della bilancia elettronica, fornendo così al lettore valori non coretti della misura.\\

Ricordiamo che le misure del periodo ($t_i$) devono essere divise per un fattore 5 al fine di ottenere il valore del singolo periodo, poichè sono state cronometrate cinque oscillazioni. Per questo motivo l'errore che affligge ogni singola misura di periodo sarà il seguente:

\begin{equation*}
	\delta t_i \,=\, \frac{1}{5} \, \frac{\Delta t}{\sqrt{12}}
\end{equation*}
%
dove $\frac{\Delta t}{\sqrt{12}}$ rappresenta l'intervalo di incertezza tipo sulla misura del periodo singolo.\\
Procediamo ora calcolando la media dei periodi ($\mathcal{T}$) per le quattro masse da noi considerate. Ricordiamo che:

\begin{equation*}
	\mathcal{T} \,=\, m[t]^* \,=\, \frac{\sum_{i}^{N} t_i}{N}
\end{equation*}
%
e che l'errore sulla media dei periodi risulta essere:

\begin{equation*}
	\delta \mathcal{T} \,=\, \frac{1}{\sqrt{N}} \, \tilde\sigma[\mathcal{T}] \quad \text{dove} \quad \tilde\sigma[\mathcal{T}] \,=\, \frac{1}{N-1} \sum_{i}^{N} (t_i - m[t]^*)^2
\end{equation*}
%
pertanto otteniamo i seguenti valori di periodo per le quattro masse applicate:
\begin{itemize}
	\item{ massa da ($88.9 \pm 0.03$)g: $\mathcal{T} \pm \delta\mathcal{T} \,=\, (2.053 \pm 0.006) s $}
	\item{ massa da ($162.7 \pm 0.03$)g: $\mathcal{T} \pm \delta\mathcal{T} \,=\, (2.053 \pm 0.004) s $}
	\item{ massa da ($202.0 \pm 0.03$)g: $\mathcal{T} \pm \delta\mathcal{T} \,=\, (2.055 \pm 0.004) s $}
	\item{ massa da ($496.7 \pm 0.03$)g: $\mathcal{T} \pm \delta\mathcal{T} \,=\, (2.060 \pm 0.004) s $}
\end{itemize}

%Periodi 2.0532   2.0533   2.0603   2.0553
%Incertezza    0.0056621   0.0039637   0.0043496   0.0043487

%Dal momento che lo scopo di questa prima analisi è quella di verificare se sussiste una dipendenza tra periodo di oscillazione del pendolo e massa applicatavi, procediamo con l'analizzare il \ref{grafico masse vs periodi}.\\
%Come si può notare, non siamo in grado di individuare le rette di massima e minima pendenza in quanto le barre di errore risultano essere troppo piccole, anche rispetto alla rappresentazione puntiforme del dato nel grafico. Ciononostante osserviamo che la retta sembra essere una retta costante e pertanto possiamo ipotizzare che il valore del suo coefficiente angolare dovrebbe essere zero.\\
%Questa ipotesi non semra essere così errata in quanto è stato provato ripetutamente che la legge del pendolo applicata ad un modello di pendolo semplice per piccole oscillazioni mostra che il periodo di ocsillazione non dipende dalla massa applicata al filo. 

%Nonostante questo possiamo provare a calcolare il coefficiente angolare della retta per avere una minima %idea del valore che esso assume. Pertantio il coefficiente angolare della retta sarà ricavato come segue:

%\begin{equation*}
%	\text{k} \,=\, \frac{t_4 - t_1}{m_4 - m_1} 
%\end{equation*}
%
%e l'errore che affligge questa misura si puòricavare nel seguente modo:

%\begin{equation}
%	\frac{(\delta k)^2}{k^2} \,=\, \frac{\delta}{•}
%\end{equation}