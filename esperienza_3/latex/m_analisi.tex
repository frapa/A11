\subsubsection{Considerazioni iniziali}
In questa prima parte della relazione vogliamo verificare che il periodo di oscillazione del pendolo, a parità di lunghezza del filo, non dipende dalla massa applicatavi.
Per fare questo abbiamo cronometrato il periodo di ocillazione del pendolo per tutte e quattro le masse a nostra disposizine.\\
Poichè abbiamo utilizzato l'approssimazione di pendolo semplice, nel quale la massa si considera totalmente applicata in un unico punto, è importante sottolineare che, quando è stata presa la misura del periodo relativo alla massa più piccola, abbiamo dovuto regolare la lunghezza del filo in modo che il baricentro della massa risultasse essere alla stessa distanza dal morsetto, a cui è applicato il filo, rispetto alle masse precedenti.\\
Ricordiamo inoltre che per ottenere la lunghezza complessiva del filo si è seguita la seguente procedura: la lunghezza complessiva del filo è data da:

\begin{equation}
	\l_{tot} \,=\, l_{filo} + \frac{h}{2} - l_{morsetto}
\end{equation}
%
dove $l _{filo}$ è la lunghezza del filo misurata dall'estremo superiore del morsetto fino al punto di applicazione sulla massa, $l_{morsetto}$ è l'altezza del morsetto e $\frac{h}{2}$ è l'altezza che rappresenta il baricentro della massa.\\
Pertanto l'errore sulla misura della lunghezza del filo sarà dato dalla propagazione dell'errore per la somma di misure indipendenti, ovvero:

\begin{equation}
	\delta l_{tot} \,=\, \sqrt{(\delta l_{filo})^2 + (\delta \frac{h}{2})^2 + (\delta l_{morsetto})^2}
\end{equation}
%
ricordando che le inceretzze relative alle varie misure sono le incertezze tipo, ovvero:

\begin{equation}
	\text{inceretzza tipo su l} \,=\, \frac{\Delta l}{\sqrt{12}}
\end{equation}
%
Mentre le masse sono affette da un incertezza tipo pertanto ogni massa avrà la seguente incertezza:

\begin{equation*}
	\delta m \,=\, \frac{\Delta m}{\sqrt{12}} \,=\, 0.03 \,\, g
\end{equation*}









errori usati per le singole misure\\
trattazione accurata errori\\
errori totali ricavati da quali formule e come sono stati usati\\
discorso sulle misure dell'allungamento e sul baricentro dek cilindri con approssimazione pendolo semplice e non fisico, inerzia\\
regressione lineare e test del chi quadro\\
il filo non si è allungato (forse)\\
per i chi quadro e loro valore fare riferimento al sito internet datoci dal profo\\
mini conclusione sul fatto che la massa non influisce minimamente sul periodo del pendolo, verificato praticamente\\
Finire con analisi dimensionale per mostrare che va bene il fatto di non dipendenza da massa.\\