\documentclass[11pt, twoside, a4paper]{article}
\usepackage[italian]{babel}
\usepackage[utf8]{inputenc}
\usepackage{amsmath}
\usepackage{fullpage}
\usepackage{graphicx}
\usepackage{booktabs}
\usepackage{wrapfig}
\usepackage{multirow}
\usepackage{sidecap}
\usepackage{siunitx}
\usepackage[font=small]{caption}
\usepackage[bookmarks, bookmarksopen, hidelinks]{hyperref}

\begin{document}

\begin{titlepage}
\begin{center}

	\hrule \vspace{0.5cm}
     	\textsc{\LARGE TITOLO}
	\vspace{0.5cm} \hrule \vspace{2cm}

      	{\large Francesco Pasa, Davide Bazzanella, Andrea Miani\\
		Gruppo A11}\\
	\vspace{0.5cm}
      	{\large 29 Aprile 2013 - 6 Maggio 2013}
	\vfill

	\includegraphics[width=4cm]{unitn_logo.png}\\
	\vspace{1cm}
        \textsc{\Large Università degli studi di Trento}
	\vfill

	{\begin{abstract}
ABSTRACT line 1

ABSTRACT line 2
	 \end{abstract}}
\end{center}
\end{titlepage}

\newpage
\vspace*{\fill}
\begin{center}
	\tableofcontents
\end{center}
\vspace*{\fill}
\newpage
\begin{titlepage}
\begin{center}
	\hrule \vspace{0.5cm}
     	\textsc{\LARGE Misure ripetute di lunghezza e tempo}
	\vspace{0.5cm} \hrule \vspace{2cm}
      	{\large Francesco Pasa, Davide Bazzanella, Andrea Miani\\
		Gruppo A11}\\
	\vspace{0.5cm}
      	{\large 28 febbraio 2013 - 11 marzo 2013}
	\vfill
	{\begin{abstract}
Misura della lunghezza di un gruppo di 25 cilindri di metallo e della durata del periodo di oscillazione di un pendolo semplice.
Analisi dei valori ottenuti dagli esperimenti del singolo gruppo e dei valori raccolti dagli esperimenti di tutti i gruppi di laboratorio.
	 \end{abstract}}
\end{center}
\end{titlepage}

\newpage

\vspace*{\fill}
\begin{center}
	\tableofcontents
\end{center}
\vspace*{\fill}

\newpage

\section{Introduzione}
Questa relazione sarà divisa in due sezioni, una dedicata al primo esperimento
in cui verrà misurata la lunghezza di una popolazione di 25 cilindri metallici
con tre diversi strumenti (metro a nastro, calibro ventesimale, micrometro).
L'altra dedicata alla misura del periodo di un pendolo semplice (costruito con
un cavo inestensibile e un peso) effettuata da tutti i componenti del gruppo
al fine di evidenziare la presenza di eventuali errori sistematici, dovuti
alla differente prontezza di riflessi dei componenti dell'equipe. Entrambe
le sezioni saranno integrate con l'analisi dei dati ottenuti dagli
esperimenti, in particolare la prima conterrà anche lo studio dei dati
ottenuti da tutti gli altri gruppi del corso di laboratorio.



\section{Dipendenza del periodo del pendolo dalla massa}
\label{dipendenza_massa}
%\input{massa.tex}
%bla bla bla - introduzione alla massa
	\subsection{Apparato Sperimentale}
	L'apparato utilizzato per svolgere questo esperimento è composto dai seguenti elementi:
\begin{itemize}
	\item{quattro cilindri di massa diversa di cui tre aventi le stesse dimensioni, mentre il quarto risulta essere più piccolo in altezza ma di sezione equivalente;}
	\item{un'asta millimetrata con risoluzione di 1 millimetro e lunga un metro, un calibro ventesimale con risoluzione di 0.05 millimetri, un cronometro elettronico con risoluzione di lettura pari a 0.01 secondi e una bilancia elettronica con una risoluzione di 0.1 grammi;}
	\item{un filo da pesca e un supporto fisso al banco da lavoro da utilizzare per la costruzione del pendolo.}
\end{itemize}

	\subsection{Procedura di misura}
	L'acquisizione dei dati sperimentali è avvenuta nel seguente modo:
\begin{itemize}
	\item{abbiamo costruito il pendolo applicando al filo da pesca, di massa trascurabile, una delle tre masse simili tra di loro, e successivamente abbiamo appeso l'apparato al supporto;}
	\item{grazie al cronometro abbiamo rilevato il periodo di oscillazioe del pendolo cronometrando un ciclo di 5 oscillazioni per quindici volte, cinque per ogni componente del gruppo, in modo da ridurre l'errore sistematico sulla misura dovuto ad una differente prontezza di riflessi dei vari operatori;}
	\item{abbiamo prestato attenzione che l'ampiezza delle oscillazioni fosse contenuta entro i 10 di ampiezza grazie al disegno dell'angolo massimo su un foglio di carta;}
	\item{}
\end{itemize}

	\subsection{Analisi dati}
	\subsubsection{Considerazioni iniziali}
In questa prima parte della relazione vogliamo verificare che il periodo di oscillazione del pendolo, a parità di lunghezza del filo, non dipende dalla massa applicatavi.
Per fare questo abbiamo cronometrato il periodo di ocillazione del pendolo per tutte e quattro le masse a nostra disposizone.\\
Poichè abbiamo utilizzato l'approssimazione di pendolo semplice, nel quale la massa si considera totalmente applicata in un unico punto, è importante sottolineare che, quando è stata presa la misura del periodo relativo alla massa più piccola, abbiamo dovuto regolare la lunghezza del filo in modo che il baricentro della massa risultasse essere alla stessa distanza dal morsetto, a cui è applicato il filo, rispetto alle masse precedenti.\\
Ricordiamo inoltre che per ottenere la lunghezza complessiva del filo si è seguita la seguente procedura: la lunghezza complessiva del filo è data da:

\begin{equation}
	\l_{tot} \,=\, l_{filo} + \frac{h}{2} - l_{morsetto}
\end{equation}
%
dove $l _{filo}$ è la lunghezza del filo misurata dall'estremo superiore del morsetto fino al punto di applicazione sulla massa, $l_{morsetto}$ è l'altezza del morsetto e $\frac{h}{2}$ è l'altezza che rappresenta il baricentro della massa.\\
Pertanto l'errore sulla misura della lunghezza del filo sarà dato dalla propagazione dell'errore per la somma di misure indipendenti, ovvero:

\begin{equation*}
	\delta l_{tot} \,=\, \sqrt{(\delta l_{filo})^2 + (\delta \frac{h}{2})^2 + (\delta l_{morsetto})^2}
\end{equation*}
%
ricordando che le inceretzze relative alle varie misure sono le incertezze tipo, ovvero:

\begin{equation*}
	\text{inceretzza tipo su l} \,=\, \frac{\Delta l}{\sqrt{12}}
\end{equation*}
%
Mentre le masse sono affette da un incertezza tipo pertanto ogni massa avrà la seguente incertezza:

\begin{equation*}
	\delta m \,=\, \frac{\Delta m}{\sqrt{12}} \,=\, 0.03 \,\, g
\end{equation*}

\subsubsection{Dati}
Le masse che abbiamo applicato al filo da pesca sono quelle esposte nella \ref{masse} mentre nella \ref{periodi} sono elencati tutti i periodi che abbiamo rilevato per ciascuna massa.\\

Procediamo ora calcolando la media dei periodi per le quattro masse da noi considerate. Ricordiamo che:

\begin{equation*}
	\mathcal{T} \,=\, m[t]^* \,=\, \frac{\sum_{i}^{N} t_i}{N}
\end{equation*}
%
e che l'errore sulla media dei periodi risulta essere:

\begin{equation*}
	\delta \mathcal{T} \,=\, \frac{1}{\sqrt{N}} \, \tilde\sigma[\mathcal{T}] \quad \text{dove} \quad \tilde\sigma[\mathcal{T}] \,=\, \frac{1}{N-1} \sum_{i}^{N} (t_i - m[t]^*)^2
\end{equation*}
%
pertanto otteniamo:

%Dal momento che lo scopo di questa prima analisi è quella di verificare se sussiste una dipendenza tra periodo di oscillazione del pendolo e massa applicatavi, procediamo con l'analizzare il \ref{grafico masse vs periodi}.\\
%Come si può notare, non siamo in grado di individuare le rette di massima e minima pendenza in quanto le barre di errore risultano essere troppo piccole, anche rispetto alla rappresentazione puntiforme del dato nel grafico. Ciononostante osserviamo che la retta sembra essere una retta costante e pertanto possiamo ipotizzare che il valore del suo coefficiente angolare dovrebbe essere zero.\\
%Questa ipotesi non semra essere così errata in quanto è stato provato ripetutamente che la legge del pendolo applicata ad un modello di pendolo semplice per piccole oscillazioni mostra che il periodo di ocsillazione non dipende dalla massa applicata al filo. 


%Nonostante questo possiamo provare a calcolare il coefficiente angolare della retta per avere una minima %idea del valore che esso assume. Pertantio il coefficiente angolare della retta sarà ricavato come segue:

%\begin{equation*}
%	\text{k} \,=\, \frac{t_4 - t_1}{m_4 - m_1} 
%\end{equation*}
%
%e l'errore che affligge questa misura si puòricavare nel seguente modo:

%\begin{equation}
%	\frac{(\delta k)^2}{k^2} \,=\, \frac{\delta}{•}
%\end{equation}
		\subsubsection{Media pesata dei periodi}
		Vogliamo ora calcolare la media pesata delle medie del periodo ($\mathcal{T}_i$) relative a ciasuna massa, dove l'indice $i$ fa riferimento a quale massa viene presa in considerazione, ovvero $i \in \{1,2,3,4\}$.
Infatti, se i periodi misurati non dipendono dalla massa, essi dovrebbero essere compatibili con una costante.
Poiché le incertezze sulle medie dei periodi sono differenti, la miglior stima di questa costante è la media pesata delle medie.
Verificheremo successivamente che le medie dei periodi siano relamente conpatibili con un andamento costante.
Pertanto procediamo come segue:

\begin{itemize}
	\item{calcoliamo la media pesata dei periodi:
			\begin{equation*}
				\mathcal{T} \,=\, \frac{\sum_i \mathcal{T}_i w_i}{\sum w_i} \quad \text{dove} \quad w_i \,=\, \frac{1}{(\delta \mathcal{T}_i)^2}
			\end{equation*}
			%
			}
	\item{calcoliamo l'errore sulla media pesata nel seguente modo:
			\begin{equation*}
				\delta \mathcal{T} \,=\, \frac{1}{\sqrt{\sum w_i}}
			\end{equation*}
			%
			}
\end{itemize}
pertanto otteniamo una misura del periodo del pendolo di lunghezza fissata $\ell \pm \delta \ell \,=\, (\,1.053 \pm 0.0006\,)$ m che ha il seguente valore:

CHECK THIS MAN! ci sono cifre significative mancanti o di troppo 1.053 (3 cifre) 0.0006 (4 cifre) !!! PS: ho letto la frase dopo, ma mi puzza comunque un sacco sta roba - Dave

\begin{equation*}
	\mathcal{T} \pm \delta \mathcal{T} \,=\, (\,2.056 \pm 0.002\,) \, \text{s}
\end{equation*}
%
Ricordiamo che la misura della lunghezza non presenta tante cifre significative quante ne ha la sua l'incertezza poichè abbiamo deciso che non avrebbe senso fornire un valore più preciso di $\ell$ rispetto alla risoluzione dello strumento.

		\subsubsection{Test del chi quadro per la media pesata dei periodi}
		Dopo aver calcolato la media pesata delle misure $\mathcal{T}_i$ sfruttiamo il test del chi quadro per verificare se sono compatibili o meno.

%Per questo motivo defiano la seguente funzione $f$

%\begin{equation}
%	f \,:=\, A + B m 
%\end{equation}
%
%dove B rappresenta il coefficiente angolare della retta, che assumiamo nullo, $m$ la massa applicata al pendolo e A una costante che nel nostro caso particolare è il valore della media pesata dei periodi ($\mathcal{T}$).\\
%Pertanto ora procediamo nell'eseguire il test del chi quadro per accertarci se il risultato da noi trovato sia accettabile o meno.
Calcoliamo quindi il chi quadro osservato ($\chi_{oss}^2$):

\begin{equation*}
	\chi_{oss}^2 \,=\, \frac{\sum_{i=0}^{N} (\mathcal{T}_i - \mathcal{T})^2}{(\delta \mathcal{T}_i)^2} \,=\, 1.6607
\end{equation*}
%
%è importante sottolineare che il $\delta \mathcal{T}$ non è altro che $\delta \mathcal{T}_i$ in quanto $B = 0$ per ipotesi.
dove $\mathcal{T}_i$ rappresentano le quattro misure medie del periodo relative alle quattro massa considerate e $\mathcal{T}$ rappresenta il valore della media pesata trovato precedentemente.\\

Sapendo che i gradi di libertà del nostro sistema sono 3 e che vogliamo una probabilità del $5\%$ che i dati da noi raccolti non siano in accordo con le ipotesi fatte finora, cioè avendo posto la probabilità di falso allarme al $5\%$, otteniamo che l'intervallo di accettazione del $\chi_{oss}^2$ va da 0 a 7.82. Pertanto dal momento che in nostro $\chi_{oss}^2$ ha valore di 1.66 e quindi ricade all'interno dell'intervallo di accettazione possiamo dire che le misure $\mathcal{T}_i$ del periodo risultano essere compatibili. Ovvero noi sappiamo che mediamente una volta su venti può accadere che il dato sperimentale raccolto risulti essere in contraddizione con le ipotesi fatte, in particolare con quella che nega una dipendenza del periodo dalla massa, visto che abbiamo posto $B = 0$.
Abbiamo ricavato i valori degli estremi di accettazione del chi quadro grazie ad un programma presente sul sito internet "Stat Trek"\footnote{http://http://stattrek.com/online-calculator/chi-square.aspx}.
		\subsubsection{Regressione lineare}
		Finora abbiamo assunta veritiera l'ipotesi che il periodo di oscillazione del pendolo non dipende linearmente dalla massa applicata. Ciononostante ora vogliamo controllare se effettivamente sia così e quindi abbiamo deciso difare una regressione lineare su $f$ in modo da dare una stima ai parametri A e B e verificare che A risulti compatibile con il periodo $\mathcal{T}$ trovato grazie alla media pesata e che il valore di B sia compatibie con lo 0 teorico.

Procediamo operativamente in questo modo per fare la regressione lieare:

\begin{itemize}
	\item{La funzione da minimizzare che misura la discrepanza è:
			\begin{equation}
                \sum_{i=1}^{N} \frac{(\mathcal{T} - A - B m)}{(\delta \mathcal{T}_{tot})^2}	
                \label{eq:min_quad}
			\end{equation}
			%
            dove $\delta \mathcal{T}_{tot}$ è l'incertezza totale sulle misure del periodo, ottenuta sommando l'incertezza $\delta \mathcal{T}_i$ e l'errore trasferito dal peso, che risulta essere nullo in quanto come stima preliminare del parametro B abbiamo usato lo 0 poichè anche in precedenza abbiamo assunto che il periodo non dipende dalla massa.}
\end{itemize}
\begin{itemize}
	\item{Quindi per quanto studiato in classe abbiamo che:

			\begin{equation*}
				A \,=\, \frac{(\sum_i w_i x_i^2)(\sum_i w_i y_i) - (\sum_i w_i x_i)(\sum_i w_i x_i y_i)}{\Delta} \,=\, 0.034 \,\, s^2
			\end{equation*}
			%
			\begin{equation*}
				B \,=\, \frac{(\sum_i w_i)(\sum_i w_i x_i y_i) - (\sum_i w_i y_i)(\sum_i w_i x_i)}{\Delta} \,=\, 4.06 \,\, s^2 / kg
			\end{equation*}
			%
			dove:
			\begin{equation*}
				\Delta \,=\, (\sum_i w_i)(\sum_i w_i x_i^2) - (\sum_i w_i x_i)^2 \,\,\,\,\,\,\, e \,\,\,\,\,\,\,
				w_i \,=\, \frac{1}{(\delta y_i)^2}
			\end{equation*}}
	\item{Di conseguenza abbiamo che le incertezze relative su A e B sono:

			\begin{equation*}
				(\delta A)^2 \,=\, \frac{\sum_i w_i x_i^2}{\Delta}  \,\,\,\,\, e \,\,\,\,\,
				(\delta B)^2 \,=\, \frac{\sum_i w_i}{\Delta} 
			\end{equation*}}
	\end{itemize} 
	Quindi possiamo riassumere i risultati di questa procedura in questo modo:

	\begin{equation*}
		A \,\pm\, \delta A \,=\, (0.034 \,\, \pm \,\, 0.001) \,\,s^2 \,\,\,\,\, e \,\,\,\,\,
		B \,\pm\, \delta B \,=\, (4.06 \,\, \pm \,\, 0.01) \,\,s^2 \,/\, kg
	\end{equation*}

non c'entra un'azzo ma ... mi serve come traccia!!!!!!!!!!!!!!!!!!!!!!!!!!!!!!!!!!!!!!!!!!!!!!!!!!!!!!!!!!!!!!!!!!!!!!!!!!!!!!
	\subsection{Conclusioni parziali}
	Quindi grazie a quanto illustrato finora abbiamo verificato che, data una lunghezza fissata, il periodo di oscillazione del pendolo non dipende dalla massa applicata ad esso.

La non dipendenza del periodo dalla massa si può anche ricavare grazie ad un analisi dimensionale; ovvero: se supponiamo che il periodo del pendolo $\mathcal{T}$ possa dipendere dalla massa appesa $m$, dalla lunghezza del filo $\ell$ e dall'accelerazione di gravità $g$ mediante una relazione del tipo $\mathcal{T} \,\propto\, m^\alpha \, \ell^\beta \, g^\gamma$.\\
Studiando le dimensioni di $\mathcal{T}, \,\, m, \,\, \ell \,\,e\,\, g$ si ottiene che l'equazione dimensionale corretta è la seguente:

\begin{equation*}
	[T]^1 \,=\, [M]^\alpha[L]^{\beta+\gamma}[T]^{-2\gamma}
\end{equation*}
%
che risulta essere soddisfatta per

\begin{equation*}
	\alpha \,=\, 0; \quad \gamma \,=\, -1/2; \quad \beta \,=\, -\gamma \,=\, 1/2 
\end{equation*}
%
da cui abbiamo che:

\begin{equation*}
	\mathcal{T} \,=\, \mathcal{C} \, \sqrt{\frac{\ell}{g}}
\end{equation*}
%
dove il parametro $\mathcal{C}$ è una costante il cui valore non si può determinare dal semplice calcolo dimensionale. Si nota che, poichè risulta essere necessario che $\alpha$ sia uguale a 0, allora anche dall'analisi dimensionale abbiamo ottenuto una conferma che effettivamente il periodo di oscillazione di un pendolo non dipende dalla massa applicata.

\newpage
\section{Dipendenza dalla lunghezza}
	%Come già fatto nel caso, del tutto analogo, del
periodo di oscillazione di una molla, siamo andati a verificare la dipendenza
del periodo del pendolo al variare della lungezza del filo.

	%bla bla bla - introduzione alla lunghezza
	\subsection{Apparato Sperimentale}
	L'apparato sperimentale usato in questa seconda parte dell'esperimento è
identico a quello usto nella prima parte. In questo caso abbiamo usato una sola massa,
ovvero il cilindro di ottone più massiccio di altezza $h$ = 47.10 $\pm$ 0.01 mm.

	\subsection{Procedura di misura}
	Abbiamo scelto 10 diverse lunghezze del filo, riportate in tabella \ref{tab:lunghezze_filo}.

\begin{table}
    \centering
    \begin{tabular}{c c c c c c c c c c}
        \multicolumn{10}{c}{\textbf{Lunghezze del filo [cm]}} \\
        \toprule
        105.7 & 95.3 & 85.3 & 75.3 & 65.3 & 55.3 & 45.3 & 35.3 & 25.3 & 15.3 \\
        \bottomrule
    \end{tabular}
    \caption{Lunghezze del filo scelte per testare la dipendenza del periodo
        del pendolo dalla lunghezza. Le dieci misure sono equispaziate ad intervalli di \SI{10}{\centi\metre}.}
    \label{tab:lunghezze_filo}
\end{table}

\begin{table}
    \centering
    \begin{tabular}{c c c c c c c c c c c}
        \multicolumn{11}{c}{\textbf{Lunghezze e periodi}} \\
        \toprule
        Lunghezza [m] & \multicolumn{10}{c}{Periodi [s]} \\
        \midrule
        1.057 & 2.086 & 2.068 & 2.052 & 2.040 & 2.054 & 2.062 & 2.098 & 2.048 & 2.058 & 2.048 \\
        0.953 & 1.932 & 1.942 & 1.948 & 1.928 & 1.954 & 1.946 & 1.954 & 1.958 & 1.944 & 1.968 \\
        0.853 & 1.844 & 1.860 & 1.864 & 1.864 & 1.862 & 1.860 & 1.854 & 1.854 & 1.858 & 1.840 \\
        0.753 & 1.732 & 1.726 & 1.720 & 1.742 & 1.754 & 1.742 & 1.740 & 1.744 & 1.748 & 1.746 \\
        0.653 & 1.604 & 1.626 & 1.614 & 1.610 & 1.598 & 1.620 & 1.600 & 1.614 & 1.614 & 1.596 \\
        0.553 & 1.448 & 1.494 & 1.498 & 1.472 & 1.454 & 1.472 & 1.480 & 1.488 & 1.488 & 1.472 \\
        0.453 & 1.348 & 1.336 & 1.326 & 1.348 & 1.330 & 1.336 & 1.332 & 1.340 & 1.346 & 1.308 \\
        0.353 & 1.182 & 1.196 & 1.190 & 1.188 & 1.198 & 1.176 & 1.182 & 1.160 & 1.204 & 1.194 \\
        0.253 & 1.002 & 1.018 & 0.996 & 1.012 & 1.018 & 1.024 & 0.998 & 0.998 & 0.994 & 1.022 \\
        0.153 & 0.762 & 0.774 & 0.790 & 0.794 & 0.790 & 0.768 & 0.770 & 0.768 & 0.794 & 0.772 \\
        \bottomrule
    \end{tabular}
\end{table}

	\subsection{Analisi dati}
	%\input{•}
		\subsubsection{Dati ed incertezze}
		\label{l_medie}

Innanzitutto vogliamo specificare le incertezze a cui sono soggetti i nostri dati.

Le lunghezze $\bar{l}_i$ del filo e la lunghezza del morsetto $l\ped{mors}$ sono state misurate con un metro a nastro di risoluzione
$\Delta l$ = 0.001 m. L'errore standard di risoluzione di queste misure vale quindi:

\begin{equation}
    \sigma(l) = \frac{\Delta l}{\sqrt{12}} = \SI{0.0003}{\metre}
\end{equation}

L'altezza $h$ del cilindro è stata misurata con un calibro ventesimale di risoluzione $\Delta l\ped{calibro} = \SI{0.00005}{\meter}$
per cui l'errore tipo di risoluzione è:

\begin{equation}
    \sigma\ped{calibro}(l) = \frac{\Delta l\ped{calibro}}{\sqrt{12}} = \SI{0.00001}{\metre}
\end{equation}

Poichè abbiamo utilizzato il modello del pendolo semplice, siamo interessati a conoscere la distanza tra il punto
di sospensione ed il baricentro del cilindro. Il cilindro, la cui densità è stata considerata omogenea,
viene così approssimato ad un punto materiale posto nel suo centro. Vedremo in seguito come migliorare questa approssimazione.
Per ottenere la distanza tra punto di sospensione del filo e baricentro del cilindro si è usata la formula

\begin{equation}
	\l_i \,=\, \bar{l}_i \,-\, l\ped{mors} \,+\, \frac{h}{2}
    \label{eq:l_i}
\end{equation}
%
che è stata applicata ad ogni lunghezza del filo. I risultati dei calcoli sono riportati nella prima colonna
della tabella \ref{tab:l_dati}. L'incertezza sulla lunghezza trovata con l'equazione (\ref{eq:l_i}) si può ricavare mediante la propagazione
degli errori:

\begin{equation}
	\delta l = \sqrt{4\,\sigma(l)^2 + \frac{\sigma\ped{calibro}(l)^2}{4}} = \SI{6e-4}{m}
\end{equation}
%
che è stata indicata con $\delta l$ poiché è uguale per tutti i valori $l_i$.

\begin{SCtable}[1.4][b]
    \centering
    \begin{tabular}{c c c c}
        \multicolumn{4}{c}{\textbf{Dati}} \\
        \toprule
        $l_i$ [m] & $\mathcal{T}_i$ [s] & $\sigma(\mathcal{T}_i)$ [s] & $\delta T_i$ [s] \\
        \midrule
        1.053 & 2.061 & 0.006 & 0.006 \\
        0.949 & 1.947 & 0.004 & 0.004 \\
        0.849 & 1.856 & 0.003 & 0.003 \\
        0.749 & 1.739 & 0.003 & 0.003 \\
        0.649 & 1.610 & 0.003 & 0.003 \\
        0.549 & 1.477 & 0.005 & 0.005 \\
        0.449 & 1.335 & 0.004 & 0.004 \\
        0.349 & 1.187 & 0.004 & 0.004 \\
        0.249 & 1.008 & 0.004 & 0.004 \\
        0.149 & 0.778 & 0.004 & 0.004 \\
        \bottomrule
    \end{tabular}
    \caption{Sono elencate le lunghezze $l_i$ tra punto di sospensione e baricentro,
        la media $\mathcal{T}_i$ dei periodi misurati, la deviazione tipo $\sigma(\mathcal{T}_i)$
        sulla media (ricavata con metodo statistico) e l'incertezza totale $\delta T_i$
        sulle medie dei periodi. L'incertezza totale è stata ottenuta sommando alla deviazione tipo sulla media
        l'errore tipo di risoluzione, che, come si vede, non è significativo.
        Nella prima colonna, nonostante l'errore tipo valga $\delta l = \SI{6e-4}{\metre}$, si sono riportate
        soltanto le prime 3 cifre dopo la virgola in quanto non porterebbe informazione andare oltre la risoluzione
        dello strumento di misura. Per maggiori informazioni sulle
        incertezze si veda il paragrafo \ref{l_medie}.}
    \label{tab:l_dati}
\end{SCtable}

Passando alle incertezze sui periodi, la risoluzione di misura dei periodi, che dipende dal cronometro, è
$\Delta \mathcal{T} = 0.01$ s. Avendo misurato 5 periodi per ogni dato si ha che l'errore tipo di risoluzione vale:

\begin{equation}
	\sigma\ped{ris}(\mathcal{T}) = \frac{\Delta \mathcal{T}}{5\sqrt{12}} = \SI{6e-4}{s}
\end{equation}

Avendo misurato un insieme di $N = 10$ periodi per ogni lunghezza $l_i$ del cavo, possiamo trattare statisticamente i dati
al fine di ottenere un unico valore di periodo per ogni lunghezza, ma anche per ottenere una stima degli errori casuali.
Abbiamo quindi calcolato la media $\mathcal{T}_i$ di ogni insieme di periodi, ottenendo i dati riportati in tabella \ref{tab:l_dati}.

Si sono poi calcolate le deviazioni standard $\sigma(\mathcal{T}_i)$ sulle medie $\mathcal{T}_i$, utilizzando la consueta formula:

\begin{equation}
	\sigma(\mathcal{T}_i) = \sqrt{\frac{1}{N(N - 1)}\sum_{j=1}^N (\mathcal{T}_{ij} - \mathcal{T}_i)^2}
\end{equation}
%
dove $N = 10$ indica il numero di dati per in ciascun insieme, mentre $\mathcal{T}_{ij}$ indica il j-esimo dato dell'insieme di misure
relativo alla lunghezza $l_i$.

Alle incertezze $\sigma(\mathcal{T}_i)$ ricavate è stato sommato il contributo derivante dall'errore di risoluzione, calcolato poco prima,
mediante la seguente formula

\begin{equation}
	\delta\mathcal{T}_i = \sqrt{\sigma(\mathcal{T}_i)^2 + \sigma\ped{ris}(\mathcal{T})^2}
\end{equation}

In questo modo abbiamo ottenuto l'incertezza totale $\delta\mathcal{T}_i$ per ogni media $\mathcal{T}_i$, che tiene conto
dell'errore di risoluzione e degli errori casuali; possiamo quindi esprimere 
un unica misura di periodo per ogni lunghezza del filo, nel seguente modo

\begin{equation}
	\mathcal{T}_i \pm \delta\mathcal{T}_i
\end{equation}

I valori $\sigma(\mathcal{T}_i)$ e $\delta\mathcal{T}_i$ per tutte le lunghezze sono riportati in Tabella \ref{tab:l_dati}.

		\subsubsection{Predizione teorica e linearizzazione}
		Graficando i dati ottenuti nel paragrafo precedente, ci si accorge immediatamente (si veda
la figura \ref{fig:lunghezza_periodo}) che il periodo del pendolo dipende dalla lunghezza
del suo filo, come correttamente previsto dalla teoria. Tuttavia la dipendenza non è lineare.

Dall'espressione per il periodo del pendolo semplice (\ref{eq:periodo_pendolo}), si sa che
il periodo vale

\begin{SCfigure}
    \centering
    \includegraphics[width=120mm]{immagini/lunghezza_periodo.pdf}
    \caption{}
    \label{fig:lunghezza_periodo}
\end{SCfigure}

\begin{equation}
    \mathcal{T} = 2\pi\sqrt{\frac{\ell}{g}}
    \tag{\ref{eq:periodo_pendolo}}
\end{equation}
%
tuttavia qui assumiamo solo che i punti del grafico seguano una legge del tipo:

\begin{equation}
    \mathcal{T} = a\ell^b
    \label{eq:ipotesi}
\end{equation}
%
dove a e b sono due costanti. Dal grafico si può intuire una dipendenza di questo tipo.
Andremo quindi a verificare la legge (\ref{eq:periodo_pendolo}).
Al fine di linearizzare la (\ref{eq:ipotesi}) per poter utilizzare la regressione, definiamo due nuove variabili:

\begin{equation}
    X := \log_{10}{(\ell)} \qquad \qquad Y := \log_{10}{(\mathcal{T})}
\end{equation}

Le incertezze su $X$ e $Y$, calcolate con le solite regole di propagazione, sono

\begin{equation}
    \delta Y = \frac{\log_{10}(e)}{\mathcal{T}}\delta\mathcal{T}
    \qquad \qquad
    \delta X = \frac{\log_{10}(e)}{\ell}\delta \ell
\end{equation}

Quindi la (\ref{eq:ipotesi}) diventa

\begin{equation}
    Y = \log_{10} a + b \log_{10} \ell = A + b \log_{10} \ell
\end{equation}
%
dove $A := \log_{10} a$ e dunque $a = 10^A$.

%\begin{table}
%    \centering
%    \begin{tabular}{c c c c}
%    \end{tabular}
%\end{table}

Procediamo ora con il calcolo dei parametri A e b tramite regressione lineare. 
La regressione si basa sulla minimizzazione della funzione

\begin{equation}
    \sum_{i=1}^N \frac{(Y_i - A - bX_i)^2}{\delta Y\ped{tot}}
    \label{eq:min_quad}
\end{equation}
%
che misura la discrepanza tra legge lineare e dati sperimentali. Minimizzando la (\ref{eq:min_quad}) si ottengono
i valori di $A$ e $b$, ovvero la retta, che interpretano in modo ``migliore'' i dati.
Come prima cosa stimiamo il parametro b
per trasferire l'incertezza dalle X alle Y. La stima è stata ottenuta ignorando l'incertezza sulle X e facendo un fit preliminare.


\begin{equation}
    A = 0.69631 \qquad \qquad b = 0.49915
\end{equation}

Le incertezze su $A$ e $b$ valgono:

\begin{equation}
    \delta A = 0.0010388 \qquad \qquad \delta b = 0.0018843
\end{equation}

Risaliamo quindi al valore di a:

\begin{equation}
    a = 2.0063 \qquad \qquad \delta a = 0.0039972
\end{equation}

		\subsubsection{Test del chi quadro}
		Verifichiamo la bontà della regressione eseguita nel paragrafo precedente tramite il test del chi quadro.
Il valor atteso $\chi^2_{\text{teo}}$ del test è uguale al numero di gradi di libertà del sistema. Nel nostro caso
esso vale $\chi^2_{\text{teo}} = N - 2 = 8$ poiché dal fit sono stati calcolati 2 parametri. Poiché
il numero di gradi di libertà è basso, abbiamo deciso di adottare un intervallo di confidenza del tipo $[0, h]$,
dove $h$ è il valore critico del chi quadro. Scelta la probabilità di falso allarme del 5 \%, dalla distribuzione
del chi quadro si può calcolare $h = 15.5$. Quindi l'intervallo di confidenza è [0, 15.5]. 

Utilizzando i valori $A$ e $b$ trovati precedentemente, calcoliamo l'espressione

\begin{equation}
    \chi^2_{\text{oss}} = \sum_{i=1}^N \frac{(Y_i - A - bX_i)^2}{(\delta Y_i^{\text{tot}})^2} = 29.4
\end{equation}
%
che è chiaramente al di fuori dell'intervallo di confidenza.

\begin{SCfigure}[][t]
    \centering
    \includegraphics[width=115mm]{immagini/l_discrepanza.pdf}
    \caption{Il grafico mostra la discrepanza tra dati sperimentali e retta di fit, ottenuta con la formula $Y_i - A - bX_i$.
        Le barre di errore blu mostrano il valore delle incertezze prima della correzione, mentre quelle verdi mostrano gli errori
        corretti. Si nota che le barre d'errore prima della correzione erano piccole, e ciò ha causato un alto
        valore del $\chi^2$, e che la correzione è abbastanza cospicua. Non sembra visibile alcun andamento residuo nei dati.
        In entrambi gli assi sono riportati numeri puri.}
    \label{fig:l_discrepanza}
\end{SCfigure}

A questo punto abbiamo controllato che non ci fossero punti sperimentali che contribuissero significativamente
al chi quadro. Il grafico in figura \ref{fig:l_discrepanza} mostra la discrepanza tra dati e retta di fit,
ovvero la differenza $Y_i - A - bX_i$. In questo modo si possono visualizzare le incertezze e si può verificare
in maniera approssimata la stima del chi quadro. Non ci sono punti che influenzano da soli in modo significativo
il valore del chi quadro (si osservino le barre d'errore blu), al contrario risulta chiaro che l'alto valore di $\chi^2$
è dovuto a più punti.

Segue che probabilmente questa incompatibilità è dovuta al fatto che l'incertezza sui periodi è stata sottostimata.
Questo può essere dovuto a diversi fattori, primo fra tutti l'errore sistematico dovuto all'operatore. Dobbiamo
far notare che le misure, in questo esperimento, sono state rilevate da un singolo operatore che può aver introdotto
un errore sistematico. Supponiamo che
l'errore dovuto allo sperimentatore sia la causa del valore eccessivo del chi quadro. Procederemo ora a correggere le incertezze,
e successivamente verificheremo se la correzione eseguita è accettabile e convincente.

Aggiustiamo quindi le incertezze per far tornare il chi quadro.  Poiché le incertezze sono diverse per ciascun punto,
moltiplichiamo ogni incertezza $\delta Y_i^{\text{tot}}$ per un fattore $q$ e determiniamo per quale valore di $q$
il chi quadro diventa uguale al suo valore atteso

\begin{equation}
    \chi^2 = \sum_{i=1}^N \frac{(Y_i - A - bX_i)^2}{(q\, \delta Y_i^{\text{tot}})^2} = N - 2 = 8
\end{equation}
%
Poiché $q$ è uguale per tutti i valori di $\delta Y_i^{\text{tot}}$, si può portare fuori dal segno di sommatoria.
Successivamente, mediante passaggi algebrici elementari, si arriva al seguente risultato

\begin{equation}
    q^2 = \frac{1}{8} \sum_{i=1}^N \frac{(Y_i - A - bX_i)^2}{(\delta Y_i^{\text{tot}})^2} = 3.68 \qquad \text{ovvero} \qquad q = 1.92
\end{equation}

Abbiamo quindi calcolato che, affinchè il chi quadro diventi uguale al suo valor atteso, è necessario moltiplicare
le incertezze per un fattore $q \simeq 1.9$. Le incertezze vanno quindi aumentate del 90 \% circa.

L'aggiutamento da eseguire è quindi abbastanza elevato. La correzione si riferisce al valore $\delta Y_i^{\text{tot}}$.
[Per verificare se è plausibile una correzione di questa entità, proviamo a risalire alla correzione da fare sul valore di lettura
dello strumento.]
Poiché sappiamo che $\mathcal{T} = 10^Y$, con la propagazione dell'incertezza otteniamo:

\begin{equation}
    \delta \mathcal{T} = \frac{10^Y}{\log_{10}(e)} \delta Y
\end{equation}

\subsubsection{Conclusione della regressione}

Poiché sono state modificate le incertezze, prima di calcolare i valori definitivi di $a$ e $b$ e concludere l'esperimento,
è necessario ripetere la procedura di fit per calcolare le nuove incertezze $\delta A$ e $\delta b$

Occorre quindi minimizzare la funzione

\begin{equation}
    \sum_{i=1}^N \frac{(Y_i - A - bX_i)^2}{(b\,\delta Y_i^{\text{tot}})^2}
    \label{eq:l_min_quad_2}
\end{equation}

Dalla minimizzazione si ricavano i valori

\begin{equation}
    A = 0.3024 \qquad \qquad b = 0.499
\end{equation}
%
che sono, correttamente, identici a quelli ricavati nel paragrafo \ref{l_regressione}.
Infatti la funzione (\ref{eq:l_min_quad_2}) differisce da (\ref{eq:min_quad}) solo per una costante
moltiplicativa e conserva inalterati il punto di minimo.

Le incertezze su $A$ e $b$ calcolate partendo dai valori aggiustati degli errori $\delta Y_i^{\text{tot}}$
valgono

\begin{equation}
    \delta A = 0.0009 \qquad \qquad \delta b = 0.004
\end{equation}

Le incertezze sono aumentate, e questi sono i valori definitivi.

Risaliamo ora al valore di $a$:

\begin{equation}
    a = 10^A = 2.006 \; \text{s}\,\text{m}^{-b} \qquad \qquad \delta a = \ln(10) \cdot 10^A \cdot \delta A = 0.004 \; \text{s}\,\text{m}^{-b}
\end{equation}
%
dove $\ln$ è il logaritmo naturale.

	\subsection{Conclusioni dell'analisi dati}
	Dai dati sperimentali, abbiamo ricavato la legge

\begin{equation}
    \mathcal{T} = a\ell^b
    \tag{\ref{eq:ipotesi}}
\end{equation}
%
dove $a = 2.006 \pm 0.004 \; \text{s}\,\text{m}^{-\frac{1}{2}}$ e $b = 0.499 \pm 0.004$.

Ricordiamo che nel paragrafo \ref{l_pred_teo} abbiamo ricavato i seguenti valori teorici per $a$ e $b$

\begin{equation}
    a = \frac{2\pi}{\sqrt{g}} \; \frac{\text{s}}{\sqrt{\text{m}}} = 2.006 \; \frac{\text{s}}{\sqrt{\text{m}}} \qquad \qquad b = \frac{1}{2} = 0.5
\end{equation}

Possiamo quindi discutere la compatibilità tra valori ottenuti teoricamente e calcolati sperimentalmente.
In entrambi i casi il risultato dell'esperimento è positivo, infatti i valori teorici sono compatibili
con quelli sperimentali entro l'incertezza.

Abbiamo quindi verificato la correttezza del modello del pendolo semplice e della sua predizione del periodo
del pendolo. Abbiamo inoltre verificato che la dipendenza del periodo del pendolo dalla lunghezza del filo
va come $\sqrt{\ell}$. I dati sperimentali sono in perfetto accordo con la teoria, anche oltre le nostre aspettative,
ci riteniamo soddisfatti dell'esito dell'esperimento, almeno per quanto riguarda questa sezione.
\newpage
\section{Determinzione dell'accelerazione di gravità}
%\input{misurazione.tex}
Nell'approssimazione di piccole oscillazioni, si può sfruttare la relazione che lega l'accelerazione di gravità $g$ e la lunghezza $\ell$ del pendolo semplice al periodo $\mathcal{T}$ dello stesso, per ricavare l'accelerazione di gravità:
\begin{equation}
	g \,\, = \,\, (2 \pi / \mathcal{T})^2 \ell
	\label{eq:g}
\end{equation}
	\subsection{Analisi tabella}
	\begin{SCtable}
    \centering
    \begin{tabular}{c c c}
        \multicolumn{3}{c}{\textbf{Lunghezze, periodi e accelerazione ricavata}} \\
        \toprule
        Lunghezza [\si{\metre}] & Periodi [\si{\second}] & Accelerazione [\si{\metre\per\square\second}] \\ %di gravità
        $\ell_i$ & $\mathcal{T}_i \pm \delta\mathcal{T}$ & $g_i \pm \delta g_i$ \\
        \midrule
			1.0525 & 2.061 $\,\pm\,$ 0.006 & 9.78 $\,\pm\,$ 0.05 \\
			0.9485 & 1.947 $\,\pm\,$ 0.004 & 9.87 $\,\pm\,$ 0.04 \\
			0.8485 & 1.856 $\,\pm\,$ 0.003 & 9.72 $\,\pm\,$ 0.03 \\
			0.7485 & 1.739 $\,\pm\,$ 0.003 & 9.77 $\,\pm\,$ 0.04 \\
			0.6485 & 1.610 $\,\pm\,$ 0.003 & 9.88 $\,\pm\,$ 0.04 \\
			0.5485 & 1.477 $\,\pm\,$ 0.005 & 9.93 $\,\pm\,$ 0.07 \\
			0.4485 & 1.335 $\,\pm\,$ 0.004 & 9.94 $\,\pm\,$ 0.06 \\
			0.3485 & 1.187 $\,\pm\,$ 0.004 & 9.77 $\,\pm\,$ 0.07 \\
			0.2485 & 1.008 $\,\pm\,$ 0.004 & 9.65 $\,\pm\,$ 0.08 \\
			0.1485 & 0.778 $\,\pm\,$ 0.004 & 9.68 $\,\pm\,$ 0.1 \\
        \bottomrule
    \end{tabular}
    \caption{In questa tebella sono riportate nella prima colonna le misure della lunghezza del filo che sono tutte affette da un'incertezza di 0.0006 m ricavata nel paragrafo precedente al punto \ref{l_medie}. Nella seconda colonna sono riportati i valori del periodo di oscillazione del pendolo relativo a ciscuna lunghezza. Questi valori derivano dalla msura di un periodo di dieci oscillazioni, ed è per questo che l'incertezza che li affligge risulta essere minore della risoluzione dello strumento si un fattore 10. Infine nella terza colonna sono riportati i valori di $g_i$ derivanti dai dati grazie all'equazione (\ref{eq:g}) e (\ref{eq:delta_g})}
    \label{tab:calcolo_g}
\end{SCtable}

Facendo riferimento ai valori sperimentali dell'allungamento, $\ell_i$, e del periodo, $\mathcal{T}_i$, riportati nella tabella \ref{tab:calcolo_g} vogliamo calcolare l'accelerazione di gravità, $g_i$, per ognuno di essi sfruttando la relazione (\ref{eq:g}). Inoltre sappiamo che il valore così rovato di $g$ non è assoluo ma è affetto da un incertezza ($\delta g_i$) che possiamo stimare sfruttando la formula generale per la propagazione degli errori, ovvero:

\begin{equation*}
(\delta g)^2 \, \simeq \, \left( \frac{\partial g}{\partial \ell} \right)^2 (\delta \ell)^2 \, + \, \left( \frac{\partial g}{\partial \mathcal{T}} \right)^2 (\delta \mathcal{T})^2
\end{equation*}
%
Pertanto sapendo che:

\begin{equation*}
(\delta g)^2 \, \simeq \, \left( \frac{2 \pi}{\mathcal{T}} \right)^4 (\delta \ell)^2 \, + \, \left( \frac{8 \pi^2 \ell}{\mathcal{T}^3} \right)^2 (\delta \mathcal{T})^2
\end{equation*}
%
e quindi,possiamo riassumere che l'incertezza sul valote sperimentale dell'acelerazione di gravità relativo ad ogni singola misura è il seguente:

\begin{equation} \label{eq:delta_g}
\delta g \,\, \simeq \,\, \sqrt{\left( \frac{2 \pi}{\mathcal{T}} \right)^4 (\delta \ell)^2 \,\, + \,\, \left( \frac{8 \pi^2 \ell}{\mathcal{T}^3} \right)^2 (\delta \mathcal{T})^2}
\end{equation}
%
Sempre analizzando i dati da noi raccolti ci possiamo accorgere che al fine del calcolo dell'errore sull'accelerazione di gravità le inceretzze relative alla misura della lunghezza del pendolo sono meno influenti rispetto alle incertezze relative al periodo. Affermiamo questo perchè $\delta \ell$ risulta essere più piccolo di $\delta \mathcal{T}$ di un fattore dieci e sotto radice questa differenza si accentuerebbe e la loro differenza aumenterebbe di un fattore cento.
%Ciononostante, esaminando la formula (\ref{eq:delta_g}), si può notare come al decrescere del valore di $\ell_i$ diminusica l'influenza di $\delta\mathcal{T}$ e aumenti quella di $\delta\ell$.

\begin{center}
--- INSERIRE GRAFICO---

$g_i$ in funzione di $\ell_i$
\end{center}

% togliere le domande e inserire le risposte !!
Nel grafico è evidente una dipendenza di g da l? Sono più affidabili i valori di g ricavati da $\ell$ grandi o $\ell$ piccoli?

Basandosi sul modello di pendolo semplice (punto materiale, filo inestensibile e privo di massa) e di piccole oscillazioni si può osservare che i valori più attendibili di g sono quelli realizzati con $\ell$ grandi in quanto più vicini alle approssimazioni effettuate.
	\subsection{Determinare l'accelerazione di gravità dalla tabella}
	Per ottenere un valore unico di $g \pm \delta g$ procediamo seguendo due differenti metodologie:

\subsubsection{Primo metodo: Media pesata}

Il primo metodo che adotteremo per trovare il valore dell'accelerazione di gravità è il calcolo della media pesata, utilizzando le seguenti formule:

\begin{equation*}
g_w \,\, = \,\, \frac{\sum g_iw_i}{\sum w_i} \quad\quad \text{e} \quad\quad \delta g_w \,\, = \,\, \frac{1}{\sqrt{\sum w_i}} \quad\quad \text{dove} \quad\quad w_i \,\, = \,\, \frac{1}{(\delta g_i)^2}
\end{equation*}
%
$g_w$ e $\delta g_w$ rappresentano relativamente il risultato della media pesata e il suo errore relativo. Inoltre $w_i$ è il peso relativo ad ogni misura $g_i$.
Eseguendo questa procedura sui dati si ricava il seguente valore per l'accelerazione di gravità:

\begin{equation}
g_w \pm \delta g_w \,\, = \,\, 9.80 \pm 0.03 \,\, \si{\metre\per\square\second}
\end{equation}

\paragraph{Test del chi quadro\\}

Essendo la procedura della media pesata equivalente ad una regressione, possiamo effettuare il test del chi quadro per verificare
la correttezza del risultato ottenuto. Ripetendo la procedura già vista nelle sezioni precedenti calcoliamo

\begin{equation}
    \chi^2_{\text{oss}} = \sum_{i=1}^N \frac{(g_i - g_w)^2}{\delta g_i^2} = 8.0
\end{equation}
%
dove $N = 10$ è il numero di valori $g_i$ calcolati e $\delta g_i$ è l'incertezza su tali valori. Non occorre trasferire
l'errore poiché si tratta di una retta costante (e quindi parallela all'asse delle ascisse), pertanto il suo coefficiente ancgolare
è nullo.

\subsubsection{Secondo metodo: distribuzione dei valori}

Il secondo metodo che utilizziamo è quello di considerare la distribuzione dei valori $g_i$, cioè calcolarne il valore medio $m^*[g_i]$ e lo scarto quadratico medio della distribuzione dei valori medi $\sigma[m^*[g_i]]$.

\begin{equation*}
g_s \, = \, m^*[g_i] \, = \, \frac{\sum g_i}{N} \quad\quad \text{e} \quad\quad \delta g_s \, = \, \sigma[g_i] \,\, = \,\, \frac{1}{\sqrt{N}}\sqrt{\frac{\sum (g_i - m^*[g_i])^2}{N-1}}
\end{equation*}
%
da cui si ricava il valore:

\begin{equation}
g_s \pm \delta g_s \,\, = \,\, 9.80 \pm 0.03 \,\, \si{\metre\per\square\second}
\end{equation}

\paragraph{Osservazioni sulla metodologia e sulla compatibilità dei valori ricavati\\}

Comparando le due metodologie si può innanzitutto notare che il valore ottenuto con la media pesata risulta avere un errore minore del valore ottenuto con lo studio della distribuzione dei valori. Ciò è dovuto al fatto che la media pesata tiene in considerazione maggiormente i valori con un errore minore, ottenendo così una misura più precisa. Nonostante ciò, l'errore determinato con lo studio della distribuzione dei valori è poco più grande di quello determinato dalla media pesata. Questa eventualità è probabilmente dovuta al fatto che non vi è una marcata differenza tra i valori con $\delta g$ maggiore (ricavati con $\ell$ minore) e valori con $\delta g$ minore (ricavati con $\ell$ maggiore).

Per quanto riguarda la compatibilità dei due valori dell'accelerazione di gravità trovati noi sappiamo che $g_w$ e $g_s$ coincidono, compresi i loro errori di incertezza, pertanto possiamo affermare che le due grandezze sono compatibili.

% si potrebbe affermare senza dubbio che le due misure sono compatibili, tutta via a scanso di equivici verifichiamo la loro compatibilità nel seguente modo:
%\begin{itemize}
%	\item{fissiamo a priori un fattore di copertura k = 3, e si può osservare che:
%	
%		\begin{equation*}
%			R \,\, = \,\, |g_w - g_s| \,\, = \,\, \SI{0}{\metre\per\square\second}
%		\end{equation*}
%		%
%		\begin{equation*}
%			k\,\sigma_R \,\, = \,\, \sqrt{(\delta g_w)^2 + (\delta g_s)^2} \,\, = \,\, \SI{.04}{\metre\per\square\second}
%		\end{equation*}
%		%
%		}
%	\item{quindi abbiamo verificato che $R \leq k\,\sigma_R$ e pertanto sappiamo che le due misure dell'accelerazione di gravità risltano essere compatibili}
%\end{itemize}

	\subsection{Determinare l'accelerazione di gravità dal grafico}
	Al punto $\bigotimes$ la relazione tra periodo del pendolo e lunghezza era stata espressa nella formula
\begin{equation*}
	\mathcal{T} \,\, = \,\, ab^{\ell}
\end{equation*}
che confrontata con l'equazione $\bigotimes$ \ref{equazione del pendolo} permette di ricavare $a \,\, = \,\, 2 \pi / \sqrt{g}$, da cui si ottiene:
\begin{equation}
	g \,\, = \,\, \left( \frac{2 \pi}{a}\right)^2 \quad\quad e \quad\quad \delta g \,\, = \,\, \frac{\vert dg \vert}{\vert da\vert} \delta a = \frac{8 \pi^2}{a^3} \delta a
\end{equation}

Quindi si ricava il valore:
\begin{equation}
	g \pm \delta g \,\, = \,\, 9,80 \pm 0,04 \,\, ms^{-2}
\end{equation}
	\subsection{Confronto con valori tabulati}
	---
%discutere se i valori sono compatibili o meno
%se sono incompatibili usare la formula di propagazione delle incertezze per collegare l'errore sistematico di g agli errori sistematici di l e T. Ponendo prima l e poi T uguali a 0, discutere la plausibilità dell'altro errore sistematico
%(dl=0 --> dT=?; dT=0 --> dl=?)

\newpage
\section{Conclusioni}
\section{Conclusioni}
[i cilindretti fanno cacare]



\begin{equation}
\tilde{D} = \frac{1}{N - 1} \sum_{i=1}^{N} (x_i - m^*[x])^2
\end{equation}

\begin{equation}
\tilde{\sigma} = \sqrt{\frac{1}{N - 1} \sum_{i=1}^{N} (x_i - m^*[x])^2}
\end{equation}

\begin{equation}
m^*[x] = \frac{1}{N} \sum_{i=1}^{N} (x_i) \simeq \sum_{j=1}^{\mathcal{N}} (x_j p_j^*) 
\end{equation}

\begin{equation}
p_j^* = \frac{n_j^*}{N} 
\end{equation}

\begin{equation}
f_j^* = \frac{n_j^*}{N\Delta x}
\end{equation}

\begin{equation}
D^* = \langle(x - m^*)^2\rangle = \frac{1}{N} \sum_{i=1}^{N} (x_i - m^*[x])^2
\simeq \sum_{j=1}^{\mathcal{N}} (x_j - m^*)^2 p_j^*)
\end{equation}

\begin{equation}
m^*[\mathcal{T}] = \frac{1}{N} \sum_{i=1}^{N} \mathcal{T}_i
\end{equation}

\begin{equation}
\sigma^*[\mathcal{T}] = \sqrt{\frac{1}{N} \sum_{i=1}^{N} (\mathcal{T}_i - m^*)^2}
\end{equation}

\begin{equation}
\mathcal{T} = \mathcal{T}_0 \pm \delta\mathcal{T}
\end{equation}

\begin{equation}
\mathcal{T}_0 = m^* = \frac{1}{N} \sum_{i=1}^{N} \mathcal{T}_i
\end{equation}

\begin{equation}
\delta\mathcal{T}_{cas} = \tilde{\sigma}[m^*] = \frac{1}{\sqrt{N}} \sqrt{\frac{N}{N - 1}}\sigma^*[\mathcal{T}] = \sqrt{\frac{1}{N(N - 1)} \sum_{i=1}^{N} (\mathcal{T}_i - m^*)^2}
\end{equation}



\end{document}
