\subsubsection{Studio delle incertezze ed elaborazione dei dati}

\begin{table}
\end{table}

Iniziamo questa discussione analizzando le incertezze che affliggono le misure di lunghezza che abbiamo rilevato. Infatti che per trovare il dislivello tra una colonna d' acqua e l'altra del manometro abbiamo dovuto sottrarre all'altezza $h_a$ della colonna di acqua nel ramo relativo alla bottiglia, fissata inizialmente alla tacca dei \SI{98}{\centi\metre} e mantenuta costante, l'altezza $h_c$ della colonna di acqua presente nel secondo ramo dello strumento.\\
Pertanto il dislivello tra le due colonnine di liquido sarà il seguente:

\begin{equation*}
    d \,=\, h_a \,-\, h_c \qquad \qquad \qquad \text{dove} \qquad h_a \,=\, 98.0 \pm 0.03 \; \si{\centi\metre} 
\end{equation*}
%
L'incertezza sul valore di $h_a$ è l'incertezza standard di risoluzione.
Grazie alla regola per la propagazione delle incertezze sulla somma di misure si ottiene che l'errore relativo al dislivello è dato da:

\begin{equation*}
    \delta d \,=\, \sqrt{(\sigma [h_a])^2+(\sigma [h_c])^2} \,=\, \SI{4e-4}{\metre}
\end{equation*}
%
e notiamo che ha lo stesso valore per tutte le misure di dislivello effettuate. Ricordiamo che le misure dell'altezza della colonnina di acqua sono affette da un incertezza tipo di risoluzione che è identica per tutte queste e vale:

\begin{equation*}
	\sigma [h] \,=\, \sigma [h_a] \,=\, \sigma [h_c] \,=\, \frac{\Delta h}{\sqrt{12}} \,=\, \SI{0.0003}{m} 
\end{equation*}
%
dove $\Delta h$ rappresenta la risoluzione dello strumento, che trattandosi di un asta millimetrata è di 1 millimetro. Per concludere questa prima analisi possiamo osservare che i valori riportati in Tabella \ref{tab:dati} presentano un numero di cifre decimali inferiore rispetto a quelle dell'errore che le affligge. Questo è dovuto al fatto che gli errori sono più piccoli della risoluzione dello strumento, pertanto altre cifre non porterebbero alcuna informazione.

Procediamo ora con l'analisi delle incertezze sulle misure di temperatura. Noi abbiamo usato un termometro digitale con risoluzione di misura $\Delta \theta = 0.01 ^\circ$C, pertanto l'errore che afflgge le misure di temperatura non è altro che quello dato dall'incertezze di risoluzione standard, ovvero:

\begin{equation*}
	\delta \theta \,=\, \frac{\Delta \theta}{\sqrt{12}} \,=\, \SI{0.003}{\celsius}
\end{equation*}
%
%Per quanto riguarda la misura del volume della quantità di gas contenuta all'interno della bottiglia abbiamo deciso di calcolarlo nel modo seguente: a fine esperimento abbiamo posto la bottiglia di vetro sulla bilancia elettronica e abbiamo azzerato quest'ultima, successivamente abbiamo riempito di acqua la bottiglia e abbiamo rilevato quanto fosse la sua massa, quindi sfruttando la relazione:

%\begin{equation*}
%	V \,=\, \frac{\rho}{m}  
%\end{equation*}
%
%dove ricordiamo che $m$ e la massa dell'acqua, $\rho$ è la densità dell'acqua, che abbiao assunto avere il seguente valore: $\rho \,=\, \SI{1000}{\,\,kg/m^3}$ e $V$ rappresenta il volume dell'acqua che nel nostro caso è anche quello della bottiglia.\\
%Quindi per calcolare l'incertezza sulla misura del volume dobbiamo usare la tecnia della propagazione delle incertezze per rapporti tra misure. In questo caso otteniamo che:

%\begin{equation*}
%	\sigma [V] \,=\, V \,\, \sqrt{\left(\frac{\sigma [m]}{m}\right)^2}  
%\end{equation*}
%
%ricordando che l'incertezza sulla massa ($m$) non è altro che l'incertezza tipo sulla massa che vale:

%\begin{equation*}
%	\sigma [m] \,=\, \frac{\Delta m}{\sqrt{12}} \,=\, \SI{0.03}{g}
%\end{equation*}
%
%dove $\Delta m$ e la risoluzione di misura della bilancia elettronica che vale 0.1 grammi.\\
%Mentre per la stima del volume di gas presente sotto il tappo della bottiglia e nella parte di tubo non colma di liquido abbiamo deciso di procedere come segue: Abbiamo riempito le due parti con dell'acqua tramite una siringa graduata, in modo da poter stimare il volume di liquido necessario per tale operazione. In questo modo abbiamo anche una stima preliminare dell'incertezza che affligge queste due misure. In particolare abbiamo ottenuto che:

%\begin{itemize}
%	\item{il volume di gas presente sotto il tappo della bottiglia è: }
%	\item{il volume di gas presente nel tubicino e: }
%\end{itemize}
%
%dove ricordiamo che la siringa ha una risoluzione di 1 millilitro, pertanto in metri cubi sappiamo che qusto corrisponde a circa 1 centimetro cubo.

Nell'equazione (\ref{eq:legge_stato_gas}) entra in gioco anche la pressione interna $P\ped{int}$ del gas. Quest'ultima è uguale alla pressione atmosferica $P_a$ più la pressione dovuta al dislivello dell'acqua nel tubo:

\begin{equation}
	P\ped{int} \,=\, P_a \,+\, \rho  g  d
	\label{eq:P_int}
\end{equation}
%
dove $\rho = \SI{1000}{\kilo\gram\per\cubic\metre}$ è il valore di densità dell'acqua da noi assunto, $g = \SI{9.807}{\meter\per\square\second}$ è l'accelerazione di gravità locale del laboratorio e $d$ è il dislivello tra le due colonne di liquido. Grazie all'equazione (\ref{eq:P_int}) siamo stati in grado di calcolare la pressione interna del gas alle varie temperature che è riportata in Tabella \ref{tab:dati}.\\
Per calcolare l'errore sulla stima della pressione interna del gas occorre procedere nel seguente modo:

\begin{equation}
	\delta P\ped{int} \,=\, \sqrt{(\sigma [P_a])^2 + (\rho\,g\,\delta d)^2}
	\label{eq:sigma_P_int}
\end{equation}
%
La pressione atmosferica esterna è stata misurata con il barometro a mercurio a nostra disposizione. Per ogni ora sono state prese cinque misure della pressione atmosferica $P\ped{ext}$, in modo indipendente le une dalle altre, ovvero ricalibrando il barometro ad ogni lettura. I valori di pressione ottenuti sono elencati nella Tabella \ref{tab:ptu}. I dati di ogni ora sono poi stati trattati statisticamente per ricavare la media $P_a = m^*[P\ped{ext}]$ e la deviazione standard sulla media $\sigma [P_a] = \sigma[m^*[P\ped{ext}]]$. Abbiamo associato ad ogni punto sperimentale la media delle pressioni atmosferica $P_a$ dell'ora in cui è stato rilevato. I valori $P_a$ e $\sigma[P_a]$ sono stati utilizzati nelle formule (\ref{eq:P_int}) e (\ref{eq:sigma_P_int}) per calcolare la pressione
interna e la sua deviazione. I dati ottenuti sono riportati in Tabella \ref{tab:dati}.


\begin{table}
    \begin{tabular}{c | c c c | c c c}
	    \multicolumn{7}{c}{\textbf{Valori ambientali di pressione, temperatura e umidità}} \\
        \toprule
        \multicolumn{1}{c}{} & \multicolumn{3}{c}{serie 1} & \multicolumn{3}{c}{Serie 2} \\
        Ore & Pressione & Temperatura & Umidità & Pressione & Temperatura & Umidità \\
         & [\si{\milli\bar}] & [\si{\celsius}] & [\%] & [\si{\milli\bar}] & [\si{\celsius}] & [\%] \\
        \midrule
        \multirow{5}{*}{15:00} & $\,$ & $\,$ & $\,$ & 974.40 & 26 & 59 \\
         & $\,$ & $\,$ & $\,$ & 973.60 & 26 & 59 \\
         & $\,$ & $\,$ & $\,$ & 972.15 & 26 & 59 \\
         & $\,$ & $\,$ & $\,$ & 973.40 & 26 & 58 \\
         & $\,$ & $\,$ & $\,$ & 970.10 & 26 & 58 \\
        \midrule
        \multirow{5}{*}{16:00} & 961.35 & 25 & 68 & 976.00 & 25 & 59 \\
         & 961.70 & 25 & 68 & 976.20 & 25 & 59 \\
         & 963.90 & 25 & 68 & 972.20 & 25 & 59 \\
         & 963.20 & 25 & 68 & 973.70 & 25 & 59 \\
         & 965.25 & 25 & 68 & 974.50 & 25 & 59 \\
        \midrule
        \multirow{5}{*}{17:00} & 962.30 & 25 & 68 & 976.60 & 26 & 58 \\
         & 963.00 & 26 & 58 & 975.95 & 25 & 68 \\
         & 964.35 & 25 & 68 & 967.90 & 26 & 58\\
         & 962.35 & 25 & 68 & 969.20 & 26 & 59\\
         & 960.05 & 25 & 68 & 969.40 & 26 & 59 \\
        \midrule
        \multirow{5}{*}{18:00} & 965.40 & 25 & 69 & $\,$ & $\,$ & $\,$ \\
         & 960.15 & 25 & 69 & $\,$ & $\,$ & $\,$ \\
         & 962.10 & 25 & 69 & $\,$ & $\,$ & $\,$ \\
         & 960.75 & 25 & 69 & $\,$ & $\,$ & $\,$ \\
         & 959.75 & 25 & 69 & $\,$ & $\,$ & $\,$ \\
        \bottomrule
    \end{tabular}


    \caption{In questa tabella sono presentati i valori di pressione, temperatura e umidità relativi alle due giornate d'esperimento.}
    \label{tab:ptu}
\end{table}
