\subsection{Procedura di aquisizione dati}
L'allestimento dell'apparato sperimentale è avvenuto nel seguente modo:

\begin{itemize}
	\item{come operazione preliminare è stato controllato che tutto l'apparato fosse asciutto ed in particolare che la bottiglia di vetro non fosse bagnata di acqua o rugiada, in coso contrario si sarebbe usato un compressore per elminare i residui di umido presenti sul materiale;}
	\item{abbiamo montato il manometro a gas posizionando la bottigli di vetro contenente aria, con i rubinetti aperti, all'interno del becker di plastica, il tutto posizionato sopra l'agitatore magnetico. Successivamente abbiamo collegato ad una delle due valvole della bottiglia uno dei due rami del manometro. Abbiamo portato il livello dell'acqua in questo ramo del manometro al livello più alto possibile facendo attenzione che non entrasse nella bottiglia. In questo modo abbiamo ridotto il volume del gas che non è a contatto con il bagno termico;}
	\item{abbiamo preparato il bagno termico iniziale, e dopo aver atteso che la temperatura si stabilizzasse, abbiamo chiuso il secondo rubinetto, ovvero la valvola di sfogo del gas, dopo che il livello delle due colonne di acqua è stato posto alla stessa altezza. In questo modo abbiamo fissato il volume del gas e la quantità di materia presente all'interno della bottiglia;}
	\item{ricordiamo che alla fine di questa procedura il livelo dell'acqua nei due rami è alla stessa altezza, poichè la pressione interna del gas è stata posta uguale a quella esterna.}
\end{itemize}
%

In base ad una scelta del gruppo, è stato preparato il manometro in modo tale da poter misurare una diminuizione della pressione del gas nella bottiglia rispetto alla pressione di partenza, cioè la pressione atmosferica $P_A$. Questa scelta ci ha permesso di termalizzare il gas ad una temperatura che è circa paragonabile alla temperatura ambiente. Da questa temperatura poi noi saremo scesi fino a raggiungere una temperatura il più possibile vicina allo 0 $^\circ$C, e da questa saremo ritornati indietro fino a raggiungere nuovamente la temperatura di partenza.

Pertato siamo partiti da una temperatura iniziale di circa 20 $^\circ$C. A questa temperatura è stato tarato il manometro in modo tale che il livello dell'acqua fosse uguale in entrambi i suoi rami. Completato questo passo abbiamo quindi chiuso il rubinetto della bottiglia vincolando per il resto dell'esperimento la quantità di gas presente all'interno della bottiglia. Da notare che questa scelta ci porta ad avere una bottiglia di vetro che risulta essere in depressione, pertanto tutto l'apparato dovrà essere montato sotto il livello della bottiglia in modo che il ramo libero del manometro abbia una certa libertà di movimento, ovvero che possa traslare verticalmente di circa \SI{80}{\centi\metre} rispetto alla posizione di equilibrio iniziale.

Quindi completata la preparazione dell'apparato e fissati così i dati iniziali abbiamo stabilizzato il sistema ad una temperatura inferiore di circa 1-1.5 $^\circ$C rispetto a quella di partenza. Abbiamo pertanto riposizionato il livello dell'acqua nel ramo del manometro collegato alla bottiglia alla posizione iniziale in modo da mantenere il volume del gas costante e quindi misurato il dislivello del liquido tra i due rami del manometro.
Abbiamo ripetuto questo procedimento fino ad arrivare ad una temperatura vicina a 0 $^\circ$C per poi ripartire e ritornare ad una temperatura vicina a quella iniziale.

L'intera procedura è stata eseguita due volte in due giorni differenti. Si sono quindi raccolte due serie di dati indipendenti, riportate
in Tabella (\ref{tab:dati}). I valori di della pressione atmosferica ($P_a$) delle due giornate di esperienza sono
riportati in tabella (\ref{tab:ptu}), sotto le rispettive colonne.

\begin{SCtable}
    \centering
    \begin{tabular}{c c | c c}
        \multicolumn{4}{c}{\textbf{Temperature e dislivelli}} \\
        \toprule
        \multicolumn{2}{c}{Serie 1} & \multicolumn{2}{c}{Serie 2} \\

        Temperatura & Dislivello  & Temperatura & Dislivello  \\  
        $\theta$ [$^\circ$C] & $d$ [m] & $\theta$ [$^\circ$C] & $d$ [m] \\ 
        \midrule
            20.81 &  0.000  & 23.20 &  0.000  \\
            18.14 & -0.088  & 21.45 & -0.083  \\
            16.05 & -0.155  & 19.73 & -0.166  \\
            13.77 & -0.253  & 17.57 & -0.259  \\
            12.44 & -0.299  & 15.58 & -0.348  \\
            10.29 & -0.381  & 13.27 & -0.449  \\
            8.51  & -0.451  & 11.57 & -0.517  \\
            6.70  & -0.526  & 9.38  & -0.603  \\
            4.60  & -0.625  & 7.50  & -0.677  \\
            2.26  & -0.695  & 5.76  & -0.738  \\
            0.05  & -0.771  & 3.10  & -0.836  \\
            1.89  & -0.709  & 2.50  & -0.864  \\
            3.62  & -0.636  & 4.64  & -0.779  \\
            5.55  & -0.564  & 6.48  & -0.713  \\
            7.68  & -0.475  & 8.80  & -0.616  \\
            9.38  & -0.410  & 10.56 & -0.545  \\
            11.35 & -0.335  & 12.44 & -0.470  \\
            13.00 & -0.272  & 14.30 & -0.391  \\
            15.01 & -0.194  & 16.24 & -0.307  \\
            17.03 & -0.123  & 18.65 & -0.197  \\
            19.06 & -0.053  & 20.50 & -0.112  \\
            21.03 &  0.011  & 22.42 & -0.027  \\
                  &         & 23.95 &  0.050  \\
        \bottomrule
    \end{tabular}
	\caption{I dati qui elencati sono i valori di lettura degli strumenti e sono i dati da noi rilevati durante l'esperimento.
    Sono riportati in ordine di rilevamento. L'incertezza sulla temperatura è di \SI{0.3}{mm}, mentre l'incertezza sulla temperatura è
    \SI{0.003}{\celsius}.}
    \label{tab:dati}
\end{SCtable}

\newpage
