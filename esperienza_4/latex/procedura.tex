\subsection{Procedura di aquisizione dati}
In base ad una scelta del gruppo, è stato preparato il manometro in modo tale da poter misurare una diminuizione della pressione del gas nella bottiglia rispetto alla pressione di partenza, cioè la pressione atmosferica $P_A$. Questa scelta ci ha permesso di termalizzare il gas circa a temperatura ambiente per poi scendere a circa 0 $^o$C ed infine ritornare alla temperatura iniziale.

Una volta immersa la bottiglia in un bagno d'acqua ad una temperatura iniziale di circa 20 $^o$C, abbiamo è tarato il manometro in modo tale che il livello dell'acqua fosse uguale in entrambi i rami del manometro. Completato questo passo abbiamo chiuso il rubinetto della bottiglia vincolando per il resto dell'esperimento la quantità di gas presente all'interno della bottiglia.

Completata la preparazione dell'apparato e fissati così i dati iniziali abbiamo stabilizzato il sistema ad una temperatura inferiore di circa 1-1.5 $^o$C e riposizionato il livello dell'acqua nel ramo del manometro collegato alla bottiglia e misurato il dislivello dell'acqua tra i due rami del manometro.
Abbiamo ripetuto tale procedimento fino ad arrivare ad una temperatura vicina a 0 $^o$C e poi ritornare ad una temperatura vicina a quella iniziale.

I dati ricavati sono i seguenti:

\begin{center}
	--- --- TABELLA DATI GIORNO 1 --- ---
	
	--- --- TABELLA DATI GIORNO 2 --- ---
\end{center}