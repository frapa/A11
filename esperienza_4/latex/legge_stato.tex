\section{Legge di stato}

$\Uparrow$ Questo titolo fa schifo. No, tu fai schifo! Gollum! Gollum! Gollum! Tesssoro!
\bigskip

Procediamo ora con l'analizare in maniera più approfondita la legge di stato dei gas perfetti, che ricordiamo essere la seguente:

\begin{equation*}
	P \, V \,=\, n \, R \, \theta \quad\quad \text{dove} \quad\quad R \,=\, 8.31472 \, \frac{\text{J}}{\text{mol} \, K}
\end{equation*}

E' doveroso ricordare che questa legge, come in molti altri casi della fisica, è una modellizzazione della realtà, ovvero per utilizzarla bisogna fare delle ipotesi sull'proprio apparato sperimentale, in particolare per questo esperimento noi dobbiamo assumere che:

\begin{itemize}
	\item{il gas da noi utilizzato si possa trattare come fosse un gas perfetto, ovvero un gas che possiede le seguenti proprietà: le molecole si possono considerare puntiformi, l'inerazione tra le molecole stesse e le pareti del recipiente è schematizzabile con urti perfettamente elastici, le molecole sono non interagenti e infine le molecole del gas sono identiche tra loro e indistinguibili;}
	\item{il gas compie una trasformazione isocora quasistatica: ovvero istante per istante il gas passa da una situazione di equilibrio ad un'altra, e quindi il processo si può considerare reversibile ed in particolare la variazione di entropia del gas è nulla;}
	\item{la pressione atmosfrica deve essere costante quando vengono effettuate le misure.}
\end{itemize}
%
Possiamo notare che una volta stabilite le condizioni iniziali del gas, ovvero fissati i parametri $V$, $n$, ed $R$ le due variabili dell'equazione (\ref{eq:legge_stato_gas}) sono la pressione e la temperatura che sono legate da un rapporto di proporzionalità diretta. 
