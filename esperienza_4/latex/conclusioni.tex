\section{Conclusioni}

I dati che abbiamo raccolto sono di scarsa qualità e sono affetti da errori sistematici e problemi vari, che non siamo riusciti a trovare.
La presenza di questi problemi non ci ha permesso di concludere in maniera ottimale ed esaustiva l'analisi dei dati, ne di
calcolare con una precisione accettabile lo zero assoluto. Possiamo dire quindi di aver verificato che gli esperimenti di
termodinamica sono, generalmente, più complessi e difficilmente realizzabili rispetto agli esperimenti di meccanica.
Inoltre la progettazione dell'esperimento, nel nostro caso assente o quasi, è fondamentale al fine di ottenere dati di buona
qualità dai quali poter ricavare qualche informazione utile dall'esperimento.

Abbiamo inoltre incontrato andamenti residui cospicui e ben visibili, che però non siamo riusciti a spiegare con precisione.
Si è tentato di risolvere il problema suddividendo i dati ed analizzandoli separatamente, nonché eliminado un dato
sicuramente preso male. Purtroppo tutto questo è stato vano e probabilmente l'approccio migliore sarebbe stato ripetere
l'esperimento, dopo aver verificato gli errori commessi e poterli quindi evitare nell'esecuzione successiva, in modo da ottenere dei dati più accurati con cui poter trarre delle conclusioni più approfondite.

