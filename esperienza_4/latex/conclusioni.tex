\section{Conclusioni}

I dati che abbiamo raccolto sono di scarsa qualità ed sono affetti da errori sistematici e problemi vari.
La presenza di questi problemi non ci ha permesso di concludere in maniera ottimale ed esaustiva l'analisi dati, ne di
calcolare con una precisione accettabile lo zero assoluto. POssiamo dire quindi di aver verificato che gli esperimenti di
termodinamica sono più complessi e difficilmente realizzabili rispetto agli esperimenti di meccanica.
Inoltre la progettazione dell'esperimento, nel nostro caso assente o quasi, è fondamentale per ottenere dati di buona
qualità e poter ricavare qualche informazione utile dall'esperimento.

Abbiamo incontrato andamenti residui cospicui e ben visibili, che però non siamo riusciti a spiegare con precisione.
Si è tentato di risolvere il problema suddividendo i dati ed analizzandoli separatamente, nonché eliminado un dato
sicuramente preso male. Tutto questo è stato vano e probabilmente l'approccio migliore sarebbe stato ripetere
l'esperimento con più accuratezza, dopo aver verificato gli errori commessi.

