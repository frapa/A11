\section{Calcolo dello zero assoluto}

Una volta ricavato l'andamento lineare di $h$ in funzione di $\theta$, proseguiamo nel determinare il valore dello zero assoluto. Sappiamo che 

\begin{equation}
	P \,\, = \,\, P_A + \rho g d \,\, = \,\, P_A + \rho g (A + B \theta) \,\, = \,\, f(\theta)
\end{equation}
%
che pone in relazione la pressione del gas con la temperatura da noi calcolata. Invertendo l'equazione si ottiene:

\begin{equation}
	\theta_0 \,\, = \,\, \frac{1}{B} \left( \frac{P - P_A}{pg} - A \right)
\end{equation}

\subsection{Utilizzo dei dati di pressione}
cosa devo scrivere?

\subsection{Utilizzo delle serie complete}

inserire il calcolo di $\theta_0$ delle serie complete

\subsection{Confronto dei dati delle serie con il valore noto}
\label{confronto}
Prima di verificare la compatibilià tra il valore noto e i valori di $\theta_0$ delle due serie di dati, poniamo un fattore di copertura $k=3$. Sapendo che il valore noto è -273.15, il procedimento per verificare la compatibilià si compone di:

\begin{itemize}
\item Calcolare la discrepanza $R$ tra $\theta_{0,i}$ e il valore noto:
\begin{equation*}
	R \,\, = \,\, \theta_{0,i} - (-273.15)
\end{equation*}
\item controllare se:
\begin{equation*}
	R \,\, \leq \,\, k \sigma[\theta_{0,i}]
\end{equation*}
\end{itemize}

Eseguendo tale procedura su i valori di ognuna delle due serie, si ottiene che entrambi i valori $\theta_{0,1}$ e $\theta_{0,2}$ non risultano essere compatibili con il valore noto.

Ciò è probabilmente dovuto al fatto che il modello teorico non tiene in considerazione importanti fattori come la diversa termalizzazione del gas o la presenza di vapore nel gas che possono provocare anche consistenti errori sistematici.
\bigskip

Per questo motivo abbiamo spezzato i dati ricavati in diverse sottoserie per ottenere dei fit migliori ed evidenziare gli errori sistematici.

\subsection{Utilizzo delle sottoserie}

Con il fine di migliorare la stima dello zero assoluto, abbiamo suddiviso i dati in sottoserie (vedi sez. \ref{sottoserie} per chiarimenti). 

\subsection{Confronto dei dati delle sottoserie con il valore noto}

Utilizando lo stesso procedimento descritto al paragrafo \ref{confronto}, verifichiamo la compatibilità dei nuovi valori ottenuti per lo zero assoluto.
Ricordiamo che il fattore di copertura utilizzato è $k=3$ e che il valore noto dello zero assoluto è -273.15.

Osserviamo che i nuovi valori sono compatibili (speriamo!)