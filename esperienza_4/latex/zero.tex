\section{Calcolo dello zero assoluto}

Una volta ricavato l'andamento lineare di $h$ in funzione di $\theta$, proseguiamo nel determinare il valore dello zero assoluto. Sappiamo che 

\begin{equation}
	P \,\, = \,\, P_A + \rho g d \,\, = \,\, P_A + \rho g (A + B \theta) \,\, = \,\, f(\theta)
\end{equation}

\subsection{Utilizzo dei dati di pressione}



\subsection{Confronto con il valore noto}

Prima di confrontare con il valore noto i valori di $\theta_0$ delle diverse serie di dati, poniamo un fattore di copertura $k=3$. Sapendo che il valore noto é -273.15 il procedimento per il valore di ogni serie si compone di:

\begin{itemize}
\item Calcolare la discrepanza $R$ tra $\theta_{0,i}$ e il valore noto:
\begin{equation*}
	R \,\, = \,\, \theta_{0,i} - (-273.15)
\end{equation*}
\item controllare se:
\begin{equation*}
	R \,\, \leq \,\, k \sigma[\theta_{0,i}]
\end{equation*}
\end{itemize}

Eseguendo tale procedura su i valori ricavati analizzando separatamente i valori di ognuna delle due serie si ottiene che entrambi i valori $\theta_{0,1}$ e $\theta_{0,2}$ non risultano essere compatibili con il valore noto.

Ciò è probabilmente dovuto al fatto che il modello teorico non tiene in considerazione importanti fattori come la diversa termalizzazione del gas o la presenza di vapore nel gas che possono provocare anche consistenti errori sistematici.

Per questo motivo abbiamo spezzato i dati ricavati in diverse sottoserie per ottenere dei fit migliori ed evidenziare gli errori sistematici.