\section{Calcolo dello zero assoluto}

Una volta ricavati gli andamenti lineari di $d$ in funzione di $\theta$, proseguiamo nel determinare il valore dello zero assoluto.
Data la pressione atmosferica $P_A$, e presi $g = \SI{9.807}{\meter\per\square\second}$ e
$\rho = \SI{1000}{\kilo\gram\per\cubic\metre}$ parametri di qui ignoriamo l'incertezza, sappiamo che:

\begin{equation}
	P \,\, = \,\, P_A + \rho g d \,\, = \,\, P_A + \rho g (A + B \theta) \,\, = \,\, f(\theta)
\end{equation}
%
che pone in relazione la pressione del gas con la temperatura da noi calcolata. Lo zero assoluto è la temperatura a cui la
pressione $P$ esercitata dal gas va a zero. Invertendo l'equazione sopra e ponendo $P = 0$ si ottiene:

\begin{equation}
	\theta_0 \,\, = \,\, - \frac{1}{B} \left( \frac{P_A}{pg} + A \right)
    \label{eq:t0}
\end{equation}
%
Utilizzando le regole di propagazione dell'incertezza si ottiene:

\begin{equation}
    \delta \theta_0 \,\, = \,\, \sqrt{\left(\frac{1}{B \rho g}\right)^2 (\delta P_A)^2 +
    \left(\frac{1}{B}\right)^2 (\delta A)^2 + \left(\frac{P_A + \rho g A}{B^2 \rho g} \right)^2 \delta B^2}
    \label{eq:dt0}
\end{equation}


\subsection{Utilizzo dei dati di pressione}
\label{press}

Nelle formule (\ref{eq:t0}) e (\ref{eq:dt0}) si sono usati i valori $P_A$ e $\delta P_A$ che finora non sono stati calcolati.
Essi indicano il valore della pressione atmosferica durante l'esperimento. La tabella \ref{tab:ptu} riporta i valori rilevati
durante l'esperimento. Lo scopo di questo paragrafo è ricavare un unico valore, con incertezza della pressione atmosferica nelle
due giornate. Abbiamo infatti deciso di ottenere un unica media per ogni giornata in quanto durante le due giornate il tempo metereologico e la
pressione atmosferica sono rimasti approssimativamente costanti.

I calcoli che seguono sono stati eseguiti due volte per i dati delle due giornate.

Come valore di $P_A$ si è presa la media aritmetica delle pressioni atmosferiche $P_{A,i}$ rilevate:

\begin{equation}
    P_A = m(P_{A,i}) = \frac{1}{N}\sum_{i=1}^N P_{A,i}
\end{equation}
%
dove $N = 15$ è il numero di misure fatte. Mentre come stimatore della varianza abbiamo usato la deviazione standard sulla media

\begin{equation}
    \delta P_A = \sigma(P_{A,i}) = \sqrt{\frac{1}{N(N-1)}\sum_{i=1}^N (P_{A,i} - P_A)^2}
\end{equation}
%
Otteniamo quindi due misure relative alle due serie:

\begin{equation}
    P_{A1} \pm \delta P_{A1} = (96230 \pm 50) \; \si{\pascal} \qquad P_{A2} \pm P_{A2} = (97300 \pm 70) \; \si{\pascal}
    \label{eq:pa}
\end{equation}
%
dove le cifre significative sono state arrotondate all'ordine di grandezza dall'incertezza e gli zeri finali non sono significativi.

%\subsection{Confronto dei dati delle serie con il valore noto}
%\label{confronto}
%Prima di verificare la compatibilià tra il valore noto e i valori di $\theta_0$ delle due serie di dati, poniamo un fattore di copertura $k=3$. Sapendo che il valore noto è -273.15, il procedimento per verificare la compatibilià si compone di:
%
%\begin{itemize}
%\item Calcolare la discrepanza $R$ tra $\theta_{0,i}$ e il valore noto:
%\begin{equation*}
%	R \,\, = \,\, \theta_{0,i} - (-273.15)
%\end{equation*}
%\item controllare se:
%\begin{equation*}
%	R \,\, \leq \,\, k \sigma[\theta_{0,i}]
%\end{equation*}
%\end{itemize}
%
%Eseguendo tale procedura su i valori di ognuna delle due serie, si ottiene che entrambi i valori $\theta_{0,1}$ e $\theta_{0,2}$ non risultano essere compatibili con il valore noto.
%
%Ciò è probabilmente dovuto al fatto che il modello teorico non tiene in considerazione importanti fattori come la diversa termalizzazione del gas o la presenza di vapore nel gas che possono provocare anche consistenti errori sistematici.
%\bigskip
%
%Per questo motivo abbiamo spezzato i dati ricavati in diverse sottoserie per ottenere dei fit migliori ed evidenziare gli errori sistematici.

\subsection{Calcolo dello zero dalle regressioni}

Nel paragrafo {\ref{sottoserie}} abbiamo ottenuto quattro diverse rette di fit. Per ciascuna retta abbiamo calcolato
il rispettivo zero assoluto grazie alle formule (\ref{eq:t0}) e (\ref{eq:dt0}) e ai valori di pressione (\ref{eq:pa}). Si ottengono i seguenti
valori:

\begin{center}
    \begin{tabular}{l c c c c}
        \multicolumn{5}{c}{\textbf{Zero assoluto}} \\
        \toprule
        & Gruppo 1 & Gruppo 2 & Gruppo 3 & Gruppo 4 \\
        \midrule
        $\theta_0$ [\si{\celsius}] & -272 & -234 & -195 & -231 \\
        $\delta \theta_0$ [\si{\celsius}] & 3 & 3 & 4 & 3 \\
        \bottomrule
    \end{tabular}
\end{center}

La tabella mostra i valori dello zero assoluto calcolati partendo dalle regressioni sulle quattro sottoserie di dati.
I dati sono riportati con precisione data dall'incertezza.
  
\subsection{Confronto dei dati delle sottoserie con il valore noto}

Commentiamo brevemente i risultati e la loro compatibilità
con il valore teorico $T_0 = \SI{-273.15}{\celsius}$. Per la compatibilità adottiamo un fattore di copertura $k = 3$.

\begin{itemize}
    \item{\textbf{Gruppo 1.} Il valore ottenuto è molto vicino a quello teorico ed è compatibile con quello teorico
        a meno di un sigma.}

    \item{\textbf{Gruppo 2.} Il valore ottenuto non è compatibile con quello teorico, infatti
        
        \begin{equation}
            R = |T_0 - \theta_0| \simeq \SI{39}{\celsius} \qquad \text{e} \qquad k\delta \theta_0 = \SI{9}{\celsius} < R
        \end{equation}

        Se prendiamo per certo il valore teorico (verificato sperimentalmente da fisici sperimentali senza dubbio migliori
        di noi), questo è un ulteriore indizio che l'effetto residuo che ha scombinato i conti era presente solo nel gruppo 2.
        Questa è una conferma dell'ipotesi della condensazione dell'acqua, che prevede, correttamente, effetti solo sulla serie 2.
        Chiaramente non sono escluse altre ipotesi.}

    \item{\textbf{Gruppo 3.} Il valore calcolato è molto lontano e non compatibile (si confronti con il precedente) con il valore
        teorico. Sicuramente qui c'è qualche grosso errore nell'esecuzione dell'esperimento.}
    
    \item{\textbf{Gruppo 4.} In questo caso il discorso è analogo a quello fatto per il gruppo 2.}
\end{itemize}

Concludiamo dicendo che i valori sperimentali dello zero assoluto ottenuti non sono soddisfacenti in quanto tre su quattro
non sono compatibili con il valore teorico, ma soprattutto non sono nemmeno compatibili tra di loro. Il motivo dovrebbe essere
ovvio; i dati che abbiamo raccolto sono di di scarsa qualità poiché l'esperimento è stato affetto da errori sistematici.
