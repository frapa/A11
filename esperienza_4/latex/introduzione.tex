\section{Introduzione}
L'obiettivo di questa esperienza è la misurazione della pressione di un volume costante di aria in una trasformazione isocora in un intervallo di temperature che va da circa 0$^\circ$C a 20$^\circ$C. I valori della pressione del gas verranno ricavati tramite la lettura del dislivello di una colonnina d'acqua in un manometro differenziale ad acqua al variare della temperatura $\theta$.
Infine andremo a ricavare il valore $\theta_0$ dello \emph{zero assoluto} mediante un'estrapolazione eseguita sulla legge che lega la pressione $P$ in funzione della temperatura $\theta$ verso le basse temperature.
Pertanto uno dei cardini principali di questa relazione è quello rappresentato dalla seguente: ''equazione di stato dei gas perfetti'':

\begin{equation}
	P \, V \,=\, n \, R \, \theta \quad\quad \text{dove} \quad\quad R \,=\, 8.31472 \, \frac{\text{J}}{\text{mol} \, K}
	\label{eq:legge_stato_gas}
\end{equation}
%
dove $P$ indica la pressione, $V$ rappresenta il volume, $\theta$ la temperatura del gas e $n$ simboleggia il numero di molecole del gas. Invece $R$ è la costante universale dei gas.\\

Questa equazione rappresenta una generalizzazione delle leggi empiriche osservate dai fisici Boyle, Avogadro e Charles. Inoltre ricordiamo che l'equazione di stato dei gas perfetti descrive bene il comportamento dei gas reali per pressioni non troppo elevate e per temperature non troppo vicine alla temperatura di liquefazione del gas. In questi casi, una migliore descrizione del comportamento del gas è dato dall'equazione di stato di van der Waals.
