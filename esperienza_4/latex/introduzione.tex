\section{Introduzione}
L'obiettivo di questa esperienza è la misurazione della pressione di un volume costante di aria in una trasformazione isocora in un intervallo di temperature da circa 0$^\circ$C a circa 20$^\circ$C. I valori della pressione del gas verranno ricavati tramite la lettura del dislivello di una colonnina d'acqua in un manometro differenziale ad acqua a variare della temperatura $\theta$.
Infine si otterrà il valore $\theta_0$ dello \emph{zero assoluto} estrapolando l'andamento di $P$ in funzione di $\theta$ verso le basse temperature.
