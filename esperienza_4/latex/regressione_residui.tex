\subsubsection{Cause degli andamenti residui}

Nel paragrafo precedenti si sono riscontrati degli evidenti andamenti residui. Come conseguenza la legge lineare ottenuta
si adatta male ai dati, il $\chi^2$ è molto alto, come alte sono le correzioni da apportare alle incertezze affinché 
il valore del chi quadro risulti corretto. Certo, l'incertezza stimata inizialmente, che tiene conto solo degli errori di risoluzione
è sicuramente sottostimata e va corretta, ma portare l'incertezza sulla temperatura sino a \SI{0.3}{\celsius} non è accettabile.
Inoltre gli andamenti residui indicano che ci sono stati dei processi di cui non si è tenuto conto durante l'esecuzione dell'esperimento.

Ma quali sono i processi che hanno provocato tali andamenti? Dobbiamo fare qualche ipotesi e constatazione. Innanzitutto,
facciamo notare che gli andamenti residui per le due serie di dati sono di diversa natura. Le figure \ref{fig:fit1} e
\ref{fig:fit2} mostrano residui opposti rispetto alle rette calcolate nel paragrafo \ref{reg_1}. Nel caso della prima
serie di dati (figura \ref{fig:fit1}), la pendenza dell'andamento residuo è minore a quella della retta di fit a
temperature alte e maggiore a temperature basse, mentre nel caso della seconda serie di dati (figura \ref{fig:fit2}) è
vero il contrario.

In secondo luogo, i residui sono ben rappresentati da rette spezzate, seppur con qualche imprecisione.

Nel caso della prima serie di dati, l'ipotesi che ci sembra spiegare meglio l'andamento residuo osservato è la condenazione
del vapor acqueo, che era presente all'interno del vaso. Probabilmente attorno ai \SI{15}{\celsius}, punto in cui la linea
è spezzata, il vapore acqueo ha iniziato a condensare, togliendo molecole al gas e cambiando la correlazione tra dislivello e
temperatura. Secondo questa ipotesi la pressione (dislivello) a temperature più basse dovrebbe essere minore di quella prevista
con il residuo dei dati a temperature alte, che è quello che il grafico suggerisce.
Affinché questa ipotesi risulti valida occorre anche ipotizzare che la condensazione abbia proceduto ad un ritmo circa costante,
poiché la retta dovrebbe diventare meno pendente in presenza di meno gas, mentre quello che si osserva è che la pendenza aumenta.
Ipotizzando che la condensazione (e chiaramente anche l'evaporazione durante la salita) sia stata costante si può spiegare 
l'andamento residuo. 

Nel caso della seconda serie di dati, notiamo che ai punti registrati durante la ``discesa'' verso temperature più basse
corrispondono dislivelli maggiori (e quindi pressioni minori) che ai punti raccolti durante la salita. Pensiamo che quando
la temperatura del vaso era bassa, e quindi la differenza di pressione tra esterno ed interno maggiore, si sia verificata una perdita d'aria
che ha aumentato la quantità di aria presente nel contenitore. La salita è avvenuta dunque a pressioni maggiori, come
ben evidenzia il grafico in figura \ref{fig:fit1}. Tuttavia questo non spiega la forma a V del residuo, ma soltanto perché i punti
di salita/discesa sono così distaccati. Non abbiamo ipotesi che riescono a spiegare questo andamento. Alcune delle cause possibili
includono nuovamente la condensazione dell'acqua e la formazione di bolle di aria non termalizzata all'interno della bottiglia.

In entrambi i casi siamo costretti a dichiarare la nostra ignoranza sui fenomeni che hanno causato questi residui. L'unica cosa che
possiamo fare è ipotizzare cos'è successo, in assenza di una risposta certa. Nel prossimo paragrafo raggrupperemo i dati
in gruppi in base agli andamenti residui e procederemo ad analizzare ogni gruppo separatamente.

\begin{SCtable}[][t]
    \centering
    \small
    \begin{tabular}{c c @{\hspace{0.8cm}} c c @{\hspace{0.8cm}} c c @{\hspace{0.8cm}} c c}
        \multicolumn{8}{c}{\textbf{Gruppi di dati}} \\
        \toprule
        \multicolumn{2}{c}{Gruppo 1} & \multicolumn{2}{l}{Gruppo 2} &\multicolumn{2}{l}{Gruppo 3} & \multicolumn{2}{l}{Gruppo 4} \\
        \midrule
        $\theta$ [\si{\celsius}] & $d$ [\si{m}] & $\theta$ [\si{\celsius}] & $d$ [\si{m}] &
        $\theta$ [\si{\celsius}] & $d$ [\si{m}] & $\theta$ [\si{\celsius}] & $d$ [\si{m}] \\ 
        \midrule
        20.81 &  0.000 & 13.77 & -0.253 & 23.20 &  0.000 & 13.27 & -0.449 \\
        18.14 & -0.088 & 12.44 & -0.299 & 21.45 & -0.083 & 11.57 & -0.517 \\
        16.05 & -0.155 & 10.29 & -0.381 & 19.73 & -0.166 &  9.38 & -0.603 \\
        15.01 & -0.194 &  8.51 & -0.451 & 17.57 & -0.259 &  7.50 & -0.677 \\
        17.03 & -0.123 &  6.70 & -0.526 & 15.58 & -0.348 &  5.76 & -0.738 \\
        19.06 & -0.053 &  2.26 & -0.695 & 14.30 & -0.391 &  3.10 & -0.836 \\
        21.03 &  0.011 &  0.05 & -0.771 & 16.24 & -0.307 &  2.50 & -0.864  \\
              &        &  1.89 & -0.709 & 18.65 & -0.197 &  4.64 & -0.779 \\
              &        &  3.62 & -0.636 & 20.50 & -0.112 &  6.48 & -0.713 \\
              &        &  5.55 & -0.564 & 22.42 & -0.027 &  8.80 & -0.616 \\
              &        &  7.68 & -0.475 & 23.95 &  0.050 & 10.56 & -0.545 \\
              &        &  9.38 & -0.410 &       &        & 12.44 & -0.470 \\
              &        & 11.35 & -0.335 &       &        &       &        \\
              &        & 13.00 & -0.272 &       &        &       &        \\
        \bottomrule
    \end{tabular}
    \caption{I dati sono stati divisi in quattro gruppi distinti in base agli andamenti residui.
    I gruppi 1 e 2 sono ricavati dalla prima serie di dati, metre gli altri due dalla
    seconda. Il gruppo 1 comprende tutte le misure prese a temperatura maggiore di \SI{15}{\celsius},
    il gruppo 2 raggruppa le restanti. Il gruppo 3 contiene i punti al di sopra di \SI{14}{\celsius}, e il gruppo 4 quelli al di sotto
    di tale temperatura. Le cifre riportate ricalcano la risoluzione degli stumenti.}
    \label{tab:gruppi}
\end{SCtable}

\subsubsection{Regressione residui}

Poiché gli andamenti residui sono ben rappresentati da rette spezzate, abbiamo pensato di ovviare a tutti questi problemi suddividendo i
dati in vari gruppi in base agli andamenti residui, in modo da poter eseguire la regressione lineare per ogni gruppo di dati.
Abbiamo fatto quanto segue:

\label{sottoserie}
\begin{itemize}
    \item{Innanzitutto abbiamo eliminato un dato nella prima serie di misure, in quanto era probabilmente il frutto di un
        errore di misura. Questo dato, visibile in figura \ref{fig:fit1} a fianco della legenda,
        contribuiva molto al valore del $\chi^2$, ed era completamente scollegato dal resto della serie.}
    \item{Dalla prima serie di misure sono stati estratti due gruppi di dati, il primo formato da tutte le misure fatte
        a temperatura superiore a \SI{15}{\celsius} e il secondo da tutte quelle fatte a temperatura inferiore, eccetto
        il dato che è stato tolto.}
    \item{Dalla seconda serie, abbiamo estratto altri due gruppi di dati, uno con i dati rilevati a temperature maggiori di
        \SI{14}{\celsius}, e l'altro con i dati rimanenti.}
\end{itemize}

La Tabella \ref{tab:gruppi}, estratta dalla Tabella \ref{tab:dati} riporta i vari gruppi in cui sono stati suddivisi i dati.

Dopo aver suddiviso i dati in questo modo abbiamo calcolato la retta di fit per ogni gruppo di dati, in modo da ottenere
4 rette. Come nei paragrafi precedenti abbiamo seguito la seguenti procedura:

\begin{itemize}
    \item{Si è eseguito un fit preliminare al fine di calcolare il parametro $B'$ e $\delta B'$ per trasferire l'incertezza.}
    \item{È stato effettuato il trasferimento dell'incertezza dalla temperatura al dislivello.}
    \item{Si è eseguito una nuova regressione con l'incertezza totale (dislivello più quella trasferita).}
\end{itemize}

Per ulteriori informazioni sui calcoli e sulle formule usate si veda il paragrafo \ref{reg_1} dove la procedura è spiegata
nel dettaglio.

I risultati del fit preliminare ed i risultati della regressione finale sono riportati nella seguente tabella.
Ogni riga è riferita ad un gruppo, in ordine dal primo al quarto. I dati sono arrotondati alla prima cifra dell'incertezza.

\begin{center}
    \begin{tabular}{c c | c c c c}
        \multicolumn{6}{c}{\textbf{Parametri della regressione}} \\
        \toprule
        \multicolumn{2}{c|}{Fit preliminare} & \multicolumn{4}{c}{Regressione finale} \\
        \midrule
        $B'$ & $\delta B'$ & A & $\delta A$ & B & $\delta B$ \\
        $[\si{\centi\meter\per\celsius}]$ & $[\si{\centi\meter\per\celsius}]$ &
        [cm] & [cm] & $[\si{\centi\meter\per\celsius}]$ & [\si{\centi\metre\per\celsius}] \\
        \midrule
        3.353 & 0.007 & -69.5  & 0.1 & 3.353 & 0.007 \\
        3.890 & 0.003 & -78.15  & 0.02 & 3.890 & 0.003 \\
        4.550 & 0.004 & -105.08 & 0.08 & 4.550 & 0.004 \\
        3.877 & 0.003 & -96.07  & 0.03 & 3.877 & 0.004 \\
        \bottomrule
    \end{tabular}
\end{center}

È evidente la compatibilità di tutti i valori $B'$, ottenuti dalle regressioni preliminari, con i rispettivi valori $B$ ottenuti dai
fit finali. I valori di $B'$ e $B$ sono infatti uguali entro l'incertezza in tutti e 4 i casi.
